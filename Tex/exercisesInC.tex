%----------------------%

%%%%%%%%%%%%%%%%%%%%%%%%%%%%%%%%%%%%%%%%%
% The Legrand Orange Book
% LaTeX Template
% Version 1.4 (12/4/14)
%
% This template has been downloaded from:
% http://www.LaTeXTemplates.com
%
% Original author:
% Mathias Legrand (legrand.mathias@gmail.com)
%
% License:
% CC BY-NC-SA 3.0 (http://creativecommons.org/licenses/by-nc-sa/3.0/)
%
% Compiling this template:
% This template uses biber for its bibliography and makeindex for its index.
% When you first open the template, compile it from the command line with the 
% commands below to make sure your LaTeX distribution is configured correctly:
%
% 1) pdflatex main
% 2) makeindex main.idx -s StyleInd.ist
% 3) biber main
% 4) pdflatex main x 2
%
% After this, when you wish to update the bibliography/index use the appropriate
% command above and make sure to compile with pdflatex several times 
% afterwards to propagate your changes to the document.
%
% This template also uses a number of packages which may need to be
% updated to the newest versions for the template to compile. It is strongly
% recommended you update your LaTeX distribution if you have any
% compilation errors.
%
% Important note:
% Chapter heading images should have a 2:1 width:height ratio,
% e.g. 920px width and 460px height.
%
%%%%%%%%%%%%%%%%%%%%%%%%%%%%%%%%%%%%%%%%%

%----------------------------------------------------------------------------------------
%	PACKAGES AND OTHER DOCUMENT CONFIGURATIONS
%----------------------------------------------------------------------------------------

\documentclass[11pt,fleqn]{book} % Default font size and left-justified equations

\usepackage[top=3cm,bottom=3cm,left=3.2cm,right=3.2cm,headsep=10pt,a4paper]{geometry} % Page margins

\usepackage{xcolor} % Required for specifying colors by name
\definecolor{ocre}{RGB}{243,102,25} % Define the orange color used for highlighting throughout the book
%\definecolor{ocre}{RGB}{89,245,50} % Define the orange color used for highlighting throughout the book

% Font Settings
\usepackage{avant} % Use the Avantgarde font for headings
%\usepackage{times} % Use the Times font for headings
\usepackage{mathptmx} % Use the Adobe Times Roman as the default text font together with math symbols from the Sym­bol, Chancery and Com­puter Modern fonts
\usepackage{gensymb}

% Cut-and-Pastable .pdf output ?
\usepackage{cmap}

\usepackage{epsdice} % Dice symbols (texlive-fonts-extra)
\usepackage{chessfss} % Chess Symbols (texlive-games)

\usepackage{microtype} % Slightly tweak font spacing for aesthetics
\usepackage[utf8]{inputenc} % Required for including letters with accents
\usepackage[T1]{fontenc} % Use 8-bit encoding that has 256 glyphs

\usepackage{xfrac} % Running fractions in Farey etc.

\usepackage{pmboxdraw} % textblock in mode7 etc.

\usepackage{cwpuzzle}

% Bibliography
\usepackage[style=alphabetic,sorting=nyt,sortcites=true,autopunct=true,babel=hyphen,hyperref=true,abbreviate=false,backref=true,backend=biber]{biblatex}
\addbibresource{bibliography.bib} % BibTeX bibliography file
\defbibheading{bibempty}{}

% Index
\usepackage{calc} % For simpler calculation - used for spacing the index letter headings correctly
\usepackage{makeidx} % Required to make an index
\makeindex % Tells LaTeX to create the files required for indexing

%----------------------------------------------------------------------------------------

%----------------------------------------------------------------------------------------
%	VARIOUS REQUIRED PACKAGES
%----------------------------------------------------------------------------------------

\usepackage{titlesec} % Allows customization of titles

\usepackage{graphicx} % Required for including pictures
\graphicspath{{../Pictures/}} % Specifies the directory where pictures are stored

\usepackage{lipsum} % Inserts dummy text

\usepackage{tikz} % Required for drawing custom shapes
\usepackage{forest} % Better trees
\usetikzlibrary{matrix} % Some 2D grid stuff
\usetikzlibrary{shapes.symbols}
\usetikzlibrary{shapes.geometric}
\usetikzlibrary{positioning}
\usetikzlibrary{arrows.meta}
\usetikzlibrary{fit}

\usepackage[british]{babel} % English language/hyphenation

\usepackage{enumitem} % Customize lists


\setlist{nolistsep} % Reduce spacing between bullet points and numbered lists

\usepackage{booktabs} % Required for nicer horizontal rules in tables

\usepackage{eso-pic} % Required for specifying an image background in the title page

% Split year iinto four digits, so as to create 2, 0, 1, 5 etc in neon.tex
\def\Year{\expandafter\YEAR\the\year}
\def\YEAR#1#2#3#4{$#1$,$#2$,$#3$ and $#4$}

%----------------------------------------------------------------------------------------
%	MAIN TABLE OF CONTENTS
%----------------------------------------------------------------------------------------

\usepackage{titletoc} % Required for manipulating the table of contents

\contentsmargin{0cm} % Removes the default margin
% Chapter text styling
\titlecontents{chapter}[1.25cm] % Indentation
{\addvspace{15pt}\large\sffamily\bfseries} % Spacing and font options for chapters
{\color{ocre!60}\contentslabel[\Large\thecontentslabel]{1.25cm}\color{ocre}} % Chapter number
{}  
{\color{ocre!60}\normalsize\sffamily\bfseries\;\titlerule*[.5pc]{.}\;\thecontentspage} % Page number
% Section text styling
\titlecontents{section}[1.25cm] % Indentation
{\addvspace{5pt}\sffamily\bfseries} % Spacing and font options for sections
{\contentslabel[\thecontentslabel]{1.25cm}} % Section number
{}
{\sffamily\hfill\color{black}\thecontentspage} % Page number
[]
% Subsection text styling
\titlecontents{subsection}[1.25cm] % Indentation
{\addvspace{1pt}\sffamily\small} % Spacing and font options for subsections
{\contentslabel[\thecontentslabel]{1.25cm}} % Subsection number
{}
{\sffamily\;\titlerule*[.5pc]{.}\;\thecontentspage} % Page number
[] 

%----------------------------------------------------------------------------------------
%	MINI TABLE OF CONTENTS IN CHAPTER HEADS
%----------------------------------------------------------------------------------------

% Section text styling
\titlecontents{lsection}[0em] % Indendating
{\footnotesize\sffamily} % Font settings
{}
{}
{}

% Subsection text styling
\titlecontents{lsubsection}[.5em] % Indentation
{\normalfont\footnotesize\sffamily} % Font settings
{}
{}
{}
 
%----------------------------------------------------------------------------------------
%	PAGE HEADERS
%----------------------------------------------------------------------------------------

\usepackage{fancyhdr} % Required for header and footer configuration

\pagestyle{fancy}
\renewcommand{\chaptermark}[1]{\markboth{\sffamily\normalsize\bfseries\chaptername\ \thechapter.\ #1}{}} % Chapter text font settings
\renewcommand{\sectionmark}[1]{\markright{\sffamily\normalsize\thesection\hspace{5pt}#1}{}} % Section text font settings
\fancyhf{} \fancyhead[LE,RO]{\sffamily\normalsize\thepage} % Font setting for the page number in the header
\fancyhead[LO]{\rightmark} % Print the nearest section name on the left side of odd pages
\fancyhead[RE]{\leftmark} % Print the current chapter name on the right side of even pages
\renewcommand{\headrulewidth}{0.5pt} % Width of the rule under the header
\addtolength{\headheight}{2.5pt} % Increase the spacing around the header slightly
\renewcommand{\footrulewidth}{0pt} % Removes the rule in the footer
\fancypagestyle{plain}{\fancyhead{}\renewcommand{\headrulewidth}{0pt}} % Style for when a plain pagestyle is specified

% Removes the header from odd empty pages at the end of chapters
\makeatletter
\renewcommand{\cleardoublepage}{
\clearpage\ifodd\c@page\else
\hbox{}
\vspace*{\fill}
\thispagestyle{empty}
\newpage
\fi}

%----------------------------------------------------------------------------------------
%	THEOREM STYLES
%----------------------------------------------------------------------------------------

\usepackage{amsmath,amsfonts,amssymb,amsthm} % For math equations, theorems, symbols, etc

\newcommand{\intoo}[2]{\mathopen{]}#1\,;#2\mathclose{[}}
\newcommand{\ud}{\mathop{\mathrm{{}d}}\mathopen{}}
\newcommand{\intff}[2]{\mathopen{[}#1\,;#2\mathclose{]}}
\newtheorem{notation}{Notation}[chapter]

%%%%%%%%%%%%%%%%%%%%%%%%%%%%%%%%%%%%%%%%%%%%%%%%%%%%%%%%%%%%%%%%%%%%%%%%%%%
%%%%%%%%%%%%%%%%%%%% dedicated to boxed/framed environements %%%%%%%%%%%%%%
%%%%%%%%%%%%%%%%%%%%%%%%%%%%%%%%%%%%%%%%%%%%%%%%%%%%%%%%%%%%%%%%%%%%%%%%%%%
\newtheoremstyle{ocrenumbox}% % Theorem style name
{0pt}% Space above
{0pt}% Space below
{\normalfont}% % Body font
{}% Indent amount
{\small\bf\sffamily\color{ocre}}% % Theorem head font
{\;}% Punctuation after theorem head
{0.25em}% Space after theorem head
{\small\sffamily\color{ocre}\thmname{#1}\nobreakspace\thmnumber{\@ifnotempty{#1}{}\@upn{#2}}% Theorem text (e.g. Theorem 2.1)
\thmnote{\nobreakspace\the\thm@notefont\sffamily\bfseries\color{black}---\nobreakspace#3.}} % Optional theorem note
\renewcommand{\qedsymbol}{$\blacksquare$}% Optional qed square

\newtheoremstyle{blacknumex}% Theorem style name
{5pt}% Space above
{5pt}% Space below
{\normalfont}% Body font
{} % Indent amount
{\small\bf\sffamily}% Theorem head font
{\;}% Punctuation after theorem head
{0.25em}% Space after theorem head
{\small\sffamily{\tiny\ensuremath{\blacksquare}}\nobreakspace\thmname{#1}\nobreakspace\thmnumber{\@ifnotempty{#1}{}\@upn{#2}}% Theorem text (e.g. Theorem 2.1)
\thmnote{\nobreakspace\the\thm@notefont\sffamily\bfseries---\nobreakspace#3.}}% Optional theorem note

\newtheoremstyle{blacknumbox} % Theorem style name
{0pt}% Space above
{0pt}% Space below
{\normalfont}% Body font
{}% Indent amount
{\small\bf\sffamily}% Theorem head font
{\;}% Punctuation after theorem head
{0.25em}% Space after theorem head
{\small\sffamily\thmname{#1}\nobreakspace\thmnumber{\@ifnotempty{#1}{}\@upn{#2}}% Theorem text (e.g. Theorem 2.1)
\thmnote{\nobreakspace\the\thm@notefont\sffamily\bfseries---\nobreakspace#3.}}% Optional theorem note

%%%%%%%%%%%%%%%%%%%%%%%%%%%%%%%%%%%%%%%%%%%%%%%%%%%%%%%%%%%%%%%%%%%%%%%%%%%
%%%%%%%%%%%%% dedicated to non-boxed/non-framed environements %%%%%%%%%%%%%
%%%%%%%%%%%%%%%%%%%%%%%%%%%%%%%%%%%%%%%%%%%%%%%%%%%%%%%%%%%%%%%%%%%%%%%%%%%
\newtheoremstyle{ocrenum}% % Theorem style name
{5pt}% Space above
{5pt}% Space below
{\normalfont}% % Body font
{}% Indent amount
{\small\bf\sffamily\color{ocre}}% % Theorem head font
{\;}% Punctuation after theorem head
{0.25em}% Space after theorem head
{\small\sffamily\color{ocre}\thmname{#1}\nobreakspace\thmnumber{\@ifnotempty{#1}{}\@upn{#2}}% Theorem text (e.g. Theorem 2.1)
\thmnote{\nobreakspace\the\thm@notefont\sffamily\bfseries\color{black}---\nobreakspace#3.}} % Optional theorem note
\renewcommand{\qedsymbol}{$\blacksquare$}% Optional qed square
\makeatother

% Defines the theorem text style for each type of theorem to one of the three styles above
\newcounter{dummy} 
\numberwithin{dummy}{section}
\theoremstyle{ocrenumbox}
\newtheorem{theoremeT}[dummy]{Theorem}
\newtheorem{problem}{Problem}[chapter]
%% NWC : Chapter - > Section
\newtheorem{exerciseT}{Exercise}[section]
\theoremstyle{blacknumex}
\newtheorem{exampleT}{Example}[chapter]
\theoremstyle{blacknumbox}
\newtheorem{vocabulary}{Vocabulary}[chapter]
\newtheorem{definitionT}{Definition}[section]
\newtheorem{corollaryT}[dummy]{Corollary}
\theoremstyle{ocrenum}
\newtheorem{proposition}[dummy]{Proposition}

%----------------------------------------------------------------------------------------
%	DEFINITION OF COLORED BOXES
%----------------------------------------------------------------------------------------

\RequirePackage[framemethod=default]{mdframed} % Required for creating the theorem, definition, exercise and corollary boxes

% Theorem box
\newmdenv[skipabove=7pt,
skipbelow=7pt,
backgroundcolor=black!5,
linecolor=ocre,
innerleftmargin=5pt,
innerrightmargin=5pt,
innertopmargin=5pt,
leftmargin=0cm,
rightmargin=0cm,
innerbottommargin=5pt]{tBox}

% Exercise box	  
\newmdenv[skipabove=7pt,
skipbelow=7pt,
rightline=false,
leftline=true,
topline=false,
bottomline=false,
backgroundcolor=ocre!10,
linecolor=ocre,
innerleftmargin=5pt,
innerrightmargin=5pt,
innertopmargin=5pt,
innerbottommargin=5pt,
leftmargin=0cm,
rightmargin=0cm,
linewidth=4pt]{eBox}	

% Definition box
\newmdenv[skipabove=7pt,
skipbelow=7pt,
rightline=false,
leftline=true,
topline=false,
bottomline=false,
linecolor=ocre,
innerleftmargin=5pt,
innerrightmargin=5pt,
innertopmargin=0pt,
leftmargin=0cm,
rightmargin=0cm,
linewidth=4pt,
innerbottommargin=0pt]{dBox}	

% Corollary box
\newmdenv[skipabove=7pt,
skipbelow=7pt,
rightline=false,
leftline=true,
topline=false,
bottomline=false,
linecolor=gray,
backgroundcolor=black!5,
innerleftmargin=5pt,
innerrightmargin=5pt,
innertopmargin=5pt,
leftmargin=0cm,
rightmargin=0cm,
linewidth=4pt,
innerbottommargin=5pt]{cBox}

% Creates an environment for each type of theorem and assigns it a theorem text style from the "Theorem Styles" section above and a colored box from above
\newenvironment{theorem}{\begin{tBox}\begin{theoremeT}}{\end{theoremeT}\end{tBox}}
\newenvironment{exercise}{\begin{eBox}\begin{exerciseT}}{\hfill{\color{ocre}\tiny\ensuremath{\blacksquare}}\end{exerciseT}\end{eBox}}				  
\newenvironment{definition}{\begin{dBox}\begin{definitionT}}{\end{definitionT}\end{dBox}}	
\newenvironment{example}{\begin{exampleT}}{\hfill{\tiny\ensuremath{\blacksquare}}\end{exampleT}}		
\newenvironment{corollary}{\begin{cBox}\begin{corollaryT}}{\end{corollaryT}\end{cBox}}	

%----------------------------------------------------------------------------------------
%	WWW ENVIRONMENT
%----------------------------------------------------------------------------------------

\newenvironment{www}{\par\vspace{10pt}\small % Vertical white space above the remark and smaller font size
\begin{list}{}{
\leftmargin=5pt % Indentation on the left
\rightmargin=25pt}\item\ignorespaces % Indentation on the right
\makebox[-2.5pt]{\begin{tikzpicture}[overlay]
\node[draw=ocre!60,line width=1pt,circle,fill=ocre!25,font=\sffamily\bfseries,inner sep=2pt,outer sep=0pt] at (-25pt,0pt){\textcolor{ocre}{www}};\end{tikzpicture}} % Orange WWW in a circle
\advance\baselineskip -1pt}{\end{list}\vskip5pt} % Tighter line spacing and white space after remark
\newcommand{\wwwurl}[1]{\begin{www}\url{#1}\end{www}}

\newenvironment{newex}{\par\vspace{10pt}\small % Vertical white space above the remark and smaller font size
\begin{list}{}{
\leftmargin=5pt % Indentation on the left
\rightmargin=25pt}\item\ignorespaces % Indentation on the right
\makebox[-2.5pt]{\begin{tikzpicture}[overlay]
\node[draw=ocre!60,line width=1pt,circle,fill=ocre!25,font=\sffamily\bfseries,inner sep=2pt,outer sep=0pt] at (-25pt,0pt){\textcolor{ocre}{New}};\end{tikzpicture}} % Orange WWW in a circle
\advance\baselineskip -1pt}{\end{list}\vskip5pt} % Tighter line spacing and white space after remark
\newcommand{\newexercise}[1]{\begin{newex}This exercise is new for \url{#1}\end{newex}}

%%% Warning Sign
\newenvironment{hardone}{\par\vspace{10pt}\small % Vertical white space above the remark and smaller font size
\begin{list}{}{
\leftmargin=5pt % Indentation on the left
\rightmargin=25pt}\item\ignorespaces % Indentation on the right
\makebox[-2.5pt]{\begin{tikzpicture}[overlay]
\filldraw[draw=red,fill=ocre] (-40pt,-10pt) -- (-10pt,-10pt) -- (-25pt,15pt);
\node at (-25pt,0pt){\textcolor{black}{\Huge\bfseries\sffamily !}};
\node[align=left, text width=0.90\textwidth, minimum height=3ex, ocre] at (190pt,0) {Experience shows that the next exercise is trickier than most - you have been warned!};
\end{tikzpicture}}% 
\advance\baselineskip -1pt}{\end{list}\vskip5pt} % Tighter line spacing and white space after remark
\newcommand{\toohard}{\vspace*{3ex}\begin{hardone}\strut\end{hardone}\vspace*{3ex}}

% ---------------------------------------
% HACK ENVIRONMENT
% ---------------------------------------

%\newenvironment{hack}{\par\vspace{10pt}
%\begin{list}{}{
%\leftmargin=5pt % Indentation on the left
%\rightmargin=25pt}\item\ignorespaces % Indentation on the right
%\makebox[-5.5pt]{\begin{tikzpicture}[overlay]
%\node[draw=ocre,line width=2pt,shape=regular polygon, regular polygon sides=3,fill=ocre!25,font=\sffamily\bfseries,anchor=north,inner sep=0pt,outer sep=0pt] at (-50pt,0){\Huge\textcolor{ocre}{!}};\end{tikzpicture}}%
%\advance\baselineskip -1pt}{\end{list}\vskip5pt} % Tighter line spacing and white space after remark

\newenvironment{hack}{\par\noindent
\makebox[-5.5pt]{\begin{tikzpicture}[overlay]
\node[draw=ocre,line width=2pt,shape=regular polygon, regular polygon sides=3,fill=ocre!25,font=\sffamily\bfseries,anchor=north,inner sep=0pt,outer sep=0pt] at (-50pt,0){\Huge\textcolor{ocre}{!}};\end{tikzpicture}}%
}{}%


%----------------------------------------------------------------------------------------
%	SECTION NUMBERING IN THE MARGIN
%----------------------------------------------------------------------------------------

\makeatletter
\renewcommand{\@seccntformat}[1]{\llap{\textcolor{ocre}{\csname the#1\endcsname}\hspace{1em}}}                    
\renewcommand{\section}{\@startsection{section}{1}{\z@}
{-4ex \@plus -1ex \@minus -.4ex}
{1ex \@plus.2ex }
{\normalfont\large\sffamily\bfseries}}
\renewcommand{\subsection}{\@startsection {subsection}{2}{\z@}
{-3ex \@plus -0.1ex \@minus -.4ex}
{0.5ex \@plus.2ex }
{\normalfont\sffamily\bfseries}}
\renewcommand{\subsubsection}{\@startsection {subsubsection}{3}{\z@}
{-2ex \@plus -0.1ex \@minus -.2ex}
{.2ex \@plus.2ex }
{\normalfont\small\sffamily\bfseries}}                        
\renewcommand\paragraph{\@startsection{paragraph}{4}{\z@}
{-2ex \@plus-.2ex \@minus .2ex}
{.1ex}
{\normalfont\small\sffamily\bfseries}}


%----------------------------------------------------------
%   PROGRAM LISTINGS
%----------------------------------------------------------

\usepackage{listings} % Program files
%\usepackage{courier}
\usepackage{caption}
\lstset{ language=C, basicstyle=\small\rmfamily, numbers=left, numberstyle=\tiny, frame=tb, stringstyle=\color{darkgray},columns=fullflexible, backgroundcolor=\color{ocre!30},xleftmargin=17pt, framexleftmargin=17pt, framexrightmargin=5pt, framexbottommargin=4pt,showstringspaces=false,captionpos=b}
\DeclareCaptionFont{black}{\color{black}}
\DeclareCaptionFormat{listing}{\colorbox{ocre}{\parbox{\textwidth}{#1#2#3}}}
\captionsetup[lstlisting]{format=listing,labelfont=black,textfont=black}

\lstnewenvironment{terminaloutput}{
\lstset{language=bash,
           columns=fixed,
           basicstyle=\small\ttfamily\color{ocre},
           numbers=none,
           backgroundcolor=\color{darkgray},
          }
}{}

\lstnewenvironment{codesnippet}{
\lstset{ language=C, basicstyle=\small\rmfamily, numbers=none, frame=none, stringstyle=\color{darkgray},columns=fullflexible, backgroundcolor=\color{ocre!30},xleftmargin=17pt, framexleftmargin=17pt, framexrightmargin=5pt, framexbottommargin=4pt,showstringspaces=false}
}{}


% Ugly hack to allow nodes with multiple lines
\newcommand{\ml}[1]{
\begin{tabular}{@{}l@{}}#1\vspace{-2pt}\end{tabular}
}

\usepackage{currfile}
\newcommand{\nsection}[1]{\section{#1}\label{\currfilebase:sec}}

%% The "Under Construction" Warning
\newenvironment{notyet}{\par\vspace{10pt}\large % Vertical white space above the remark and smaller font size
\begin{list}{}{
\leftmargin=5pt % Indentation on the left
\rightmargin=25pt}\item\ignorespaces % Indentation on the right
\makebox[-2.5pt]{
\begin{tikzpicture}[overlay,scale=0.5,limb/.style={line cap=round,line width=1.5mm,line join=bevel}]
\draw[line width=2mm,rounded corners,fill=orange] (-2,0) -- (0,-2) -- (2,0) -- (0,2) -- cycle;
\fill (1.5mm,7mm) circle (1.5mm);
\fill(0,-7.5mm) -- ++(10mm,0mm) -- ++(120:2mm)--++(100:1mm)--++(150:2mm) arc (70:170:2.5mm and 1mm);
\draw[limb] (-7.5mm,-6.5mm)--++(70:4mm)--++(85:4mm) coordinate(a)--++(-45:5mm)--(-2.5mm,-6.5mm);
\fill[rotate around={45:(a)}] ([shift={(-0.5mm,0.55mm)}]a) --++(0mm,-3mm)--++
        (7mm,-0.5mm)coordinate(b)--++(0mm,4mm)coordinate(c)--cycle;
\draw[limb] ([shift={(-0.6mm,-0.4mm)}]b) --++(-120:5mm) ([shift={(-0.5mm,-0.5mm)}]c) --++
        (-3mm,0mm)--++(-100:3mm)coordinate (d);
\draw[ultra thick] (d) -- ++(-45:1.25cm);
\node[draw,rectangle,align=left, text width=0.75\textwidth, minimum height=3ex, ocre] at (150mm,0) {\hspace*{0mm}UNDER CONSTRUCTION - Please don't attempt this yet.\\We'll have this new exciting exercise ready for you soon!};
\end{tikzpicture}}% 
\advance\baselineskip -1pt}{\end{list}\vskip5pt} % Tighter line spacing and white space after remark

\newcommand{\undercons}{\vspace*{3ex}\begin{notyet}\strut\end{notyet}\vspace*{3ex}}%

\newcommand{\readchapter}[2]{
\nsection{Lecture Notes Chapter #1}
\begin{exercise}
After you've studied Chapter~#1 ({\em #2}), compile and run the examples given in the lecture notes.
\end{exercise} }%

\newcommand{\nopartorder}{
Note that these exercises are in {\bf NO} particular order - try the
ones you find more straightforwards before attempting complex ones.
Remember to assert {\texttt test()} all of your functions.}%

%----------------------------------------------------------------------------------------
%	HYPERLINKS IN THE DOCUMENTS
%----------------------------------------------------------------------------------------

% For an unclear reason, the package should be loaded now and not later
\usepackage{hyperref}
\hypersetup{hidelinks,backref=true,pagebackref=true,hyperindex=true,colorlinks=false,breaklinks=true,urlcolor= ocre,bookmarks=true,bookmarksopen=false,pdftitle={Title},pdfauthor={Author}}

%-----------------
% Chessboards
%-----------------
\usepackage{chessboard}
\storechessboardstyle{8x8}{%
  maxfield=h8,
  borderwidth=0mm,
  boardfontsize=10pt,
  %boardfontencoding=LSBC3,
  color=white,
  colorwhitebackfields,
  color=black,
  colorblackbackfields,
  blackfieldmaskcolor=black,
  whitepiececolor=ocre,
  %whitepiecemaskcolor=ocre,
  addfontcolors,
  pgfstyle=border,
  color=white,
  %markregion=a1-h8,
}

%----------------------------------------------------------------------------------------
%	CHAPTER HEADINGS
%----------------------------------------------------------------------------------------

% The set-up below should be (sadly) manually adapted to the overall margin page septup controlled by the geometry package loaded in the main.tex document. It is possible to implement below the dimensions used in the goemetry package (top,bottom,left,right)... TO BE DONE

\newcommand{\thechapterimage}{}
\newcommand{\chapterimage}[1]{\renewcommand{\thechapterimage}{#1}}

% Numbered chapters with mini tableofcontents
\def\thechapter{\arabic{chapter}}
\def\@makechapterhead#1{
\thispagestyle{empty}
{\centering \normalfont\sffamily
\ifnum \c@secnumdepth >\m@ne
\if@mainmatter
\startcontents
\begin{tikzpicture}[remember picture,overlay]
\node at (current page.north west)
{\begin{tikzpicture}[remember picture,overlay]
\node[anchor=north west,inner sep=0pt] at (0,0) {\includegraphics[width=\paperwidth]{\thechapterimage}};
%%%%%%%%%%%%%%%%%%%%%%%%%%%%%%%%%%%%%%%%%%%%%%%%%%%%%%%%%%%%%%%%%%%%%%%%%%%%%%%%%%%%%
% Commenting the 3 lines below removes the small contents box in the chapter heading
\fill[color=ocre!10!white,opacity=.6] (1cm,0) rectangle (8cm,-7cm);
\node[anchor=north west] at (1.1cm,.35cm) {\parbox[t][8cm][t]{6.5cm}{\huge\bfseries\flushleft \printcontents{l}{1}{\setcounter{tocdepth}{2}}}};
\draw[anchor=west] (5cm,-9cm) node [rounded corners=20pt,fill=ocre!10!white,text opacity=1,draw=ocre,draw opacity=1,line width=1.5pt,fill opacity=.6,inner sep=12pt]{\huge\sffamily\bfseries\textcolor{black}{\thechapter. #1\strut\makebox[22cm]{}}};
%%%%%%%%%%%%%%%%%%%%%%%%%%%%%%%%%%%%%%%%%%%%%%%%%%%%%%%%%%%%%%%%%%%%%%%%%%%%%%%%%%%%%
\end{tikzpicture}};
\end{tikzpicture}}
\par\vspace*{230\p@}
\fi
\fi}

% Unnumbered chapters without mini tableofcontents (could be added though) 
\def\@makeschapterhead#1{
\thispagestyle{empty}
{\centering \normalfont\sffamily
\ifnum \c@secnumdepth >\m@ne
\if@mainmatter
\begin{tikzpicture}[remember picture,overlay]
\node at (current page.north west)
{\begin{tikzpicture}[remember picture,overlay]
\node[anchor=north west,inner sep=0pt] at (0,0) {\includegraphics[width=\paperwidth]{\thechapterimage}};
\draw[anchor=west] (5cm,-9cm) node [rounded corners=20pt,fill=ocre!10!white,fill opacity=.6,inner sep=12pt,text opacity=1,draw=ocre,draw opacity=1,line width=1.5pt]{\huge\sffamily\bfseries\textcolor{black}{#1\strut\makebox[22cm]{}}};
\end{tikzpicture}};
\end{tikzpicture}}
\par\vspace*{230\p@}
\fi
\fi
}
\makeatother


\usepackage[title]{appendix}

\begin{document}

\begingroup
\thispagestyle{empty} % Suppress headers and footers on the title page
\begin{tikzpicture}[remember picture,overlay]
\draw (current page.center) node [fill=ocre!30!white,fill opacity=0.6,text opacity=1,inner sep=1cm]{\Huge\centering\bfseries\sffamily\parbox[c][][t]{\paperwidth}{\centering COMSM1201 : Exercises in C\\[15pt] % Book title
{\Large Neill Campbell}\\[20pt] % Subtitle
{\huge Department of Computer Science, University of Bristol}}}; % Author name
\end{tikzpicture}
\vfill
\endgroup

%----------------------------------------------------------------------------------------
%	COPYRIGHT PAGE
%----------------------------------------------------------------------------------------

\newpage
~\vfill
\thispagestyle{empty}

\noindent Copyright \copyright\ 2022 Neill Campbell\\ % Copyright notice

\noindent Formatted in \LaTeX, based on the Legrand Orange Book from \textsc{book-website.com}\\

\noindent This work is licensed under a Creative Commons Attribution-NonCommercial-ShareAlike 4.0 International License.

\noindent \textit{\date} % Printing/edition date

%----------------------------------------------------------------------------------------
%	TABLE OF CONTENTS
%----------------------------------------------------------------------------------------

%\usechapterimagefalse % If you don't want to include a chapter image, use this to toggle images off - it can be enabled later with \usechapterimagetrue

\chapterimage{chapter_head_1.pdf} % Table of contents heading image

\pagestyle{empty} % Disable headers and footers for the following pages

\tableofcontents % Print the table of contents itself

\cleardoublepage % Forces the first chapter to start on an odd page so it's on the right side of the book

\pagestyle{fancy} % Enable headers and footers again





%%%%%%%%%%%%%%%%%%%%%%%%%%%% WEEK 0/1 %%%%%%%%%%%%%%%%%%%%%%%%%%%%%%%%%%%%
\chapterimage{../Pictures/hello-world.png}
\chapter{Hello World}

Some of the exercises in this Chapter are taken from the book "C by Dissection".

\readchapter{B}{Hello World!}
\input{twice}
\input{letterc}

\readchapter{C}{Grammar}
\input{aplusplus}
\input{randomness}

\readchapter{D}{Flow Control}
\input{findmax}
\input{loveodd}
\input{lcg}
\input{higherlower}
\input{atm}
\input{trianglenums}

\readchapter{E}{Functions}
\input{hailstone_basic}
\input{hailstone_seq}
\input{primes_basic}
\input{whichtriangle}
\input{timeflies}
\input{dirichlet}

%%%%%%%%%%%%%%%%%%%%%%%%%%%% WEEK 2 %%%%%%%%%%%%%%%%%%%%%%%%%%%%%%%%%%%%%
\chapterimage{../Pictures/pet.png}
\chapter{Mathematics \& Characters}

\nopartorder
\readchapter{F}{Mathematics \& Characters}
\input{unitcircle}
\input{montepi}
\input{leibniz}
\input{irrational}
\input{fibword_phi}
\input{vowelness}
\input{planettrium}
\input{bob}
\input{secretcodes}
\readchapter{G}{Prettifying}
\input{roulette}

%%%%%%%%%%%%%%%%%%%%%%%%%%%% WEEK 3 %%%%%%%%%%%%%%%%%%%%%%%%%%%%%%%%%%%%%
\chapterimage{../Pictures/strings.pdf}
\chapter{$1D$ Arrays \& Strings}

\nopartorder
\readchapter{H}{$1D$ Arrays}
\input{microwave}
\input{ipod}
\newcommand{\fivedice}[5]{\epsdice[white]{#1}\epsdice[white]{#2}\epsdice[white]{#3}\epsdice[white]{#4}\epsdice[white]{#5}}
\nsection{Yahtzee}


The game of Yahtzee is a game played with five dice, and you
try to obtain certain `hands'.
In a similar way to poker, these hands could include a {\it Full House} (two dice are the same, and another three are the same), e.g.:

\fivedice{6}{6}{1}{6}{1}
or  
\fivedice{4}{1}{1}{1}{4}

\noindent or another possible hand is {\it Four-of-a-Kind}, e.g.:
\fivedice{3}{3}{1}{3}{3}
or 
\fivedice{5}{5}{5}{2}{5}
(but not
\fivedice{3}{3}{3}{3}{3}
which is {\it Five-of-a-Kind})

A little mathematics tells use that the probability of these two hands should be $3.85\%$ and $1.93\%$ respectively.
\begin{exercise}

Complete the program \verb^yahtzee.c^ (which is in the usual place
online), by analysing a large number of random dice rolls, the probabilty
of each of these two hands.  The five dice of the hand are stored in an
array, and to facilitate deciding which hand you've got, a histogram is
computed to say how often a \epsdice[white]{1} occurs in the hand, how
often a \epsdice[white]{2} occurs and so one. A {\it Full-House} occurs
when both a $2$ and a $3$ occurs in the histogram; a {\it Four-of-a-Kind}
occurs when there is a $4$ somewhere in the histogram.

\end{exercise}

\input{rule110}
\input{devils}
\input{countingsort}
\input{fibword_subs}
\readchapter{I}{Strings}
\input{palindrome}
\nsection{Int to String}

\begin{exercise}
Write a function that converts an integer
to a string, using the skeleton code in the usual place online.
works correctly:

The integer may be signed (i.e.\ be positive or negative)
and you may assume it is in base-10.
Avoid using any of the built-in string-handling functions
to do this (e.g. \verb^snprintf()^)
including all those in \verb^string.h^.
\end{exercise}


\input{roman}
\nsection{Soundex Coding}

First applied to the 1880 census, Soundex is a phonetic index, not a
strictly alphabetical one. Its key feature is that it codes surnames
(last names) based on the way a name sounds rather than on how it is
spelled. For example, surnames that sound the same but are spelled
differently, like Smith and Smyth, have the same code and are indexed
together.  The intent was to help researchers find a surname quickly
even though it may have received different spellings. If a name like
Cook, though, is spelled Koch or Faust is Phaust, a search for a
different set of Soundex codes and cards based on the variation of the
surname's first letter is necessary.

To use Soundex, researchers must first code the surname of the person
or family in which they are interested. Every Soundex code consists of
a letter and three numbers, such as B536, representing names such as
Bender. The letter is always the first letter of the surname, whether
it is a vowel or a consonant.

The detailed description of the algorithm may be found at~:\\
\wwwurl{http://www.highprogrammer.com/alan/numbers/soundex.html}

\begin{quote}
{\it 
The first letter is simply the first letter in the word. The remaining numbers range from 1 to 6, indicating different categories of sounds created by consonants following the first letter. If the word is too short to generate 3 numbers, 0 is added as needed. If the generated code is longer than 3 numbers, the extra are thrown away.

\begin{center}
\begin{tabular}{|l|l|l|}\hline
Code& 	Letters	Description \\ \hline
1	& B, F, P, V	Labial \\ \hline
2	& C, G, J, K, Q, S, X, Z	Gutterals and sibilants \\ \hline
3	& D, T	Dental \\ \hline
4	& L	Long liquid \\ \hline
5	& M, N	Nasal \\ \hline
6	& R	Short liquid \\ \hline
SKIP	& A, E, H, I, O, U, W, Y	Vowels (and H, W, and Y) are skipped \\ \hline
\end{tabular}
\end{center}

There are several special cases when calculating a soundex code:

\begin{itemize}
\item Letters with the same soundex number that are immediately next to each other are discarded. So Pfizer becomes Pizer, Sack becomes Sac, Czar becomes Car, Collins becomes Colins, and Mroczak becomes Mrocak.
\item If two letters with the same soundex number seperated by "H" or "W", only use the first letter. So Ashcroft is treated as Ashroft.
\end{itemize}

Sample Soundex codes:

\begin{center}
\begin{tabular}{|l|l|} \hline
Word	 	& Soundex \\ \hline
Washington	& W252 \\ \hline
Wu	 	& W000 \\ \hline
DeSmet	 	& D253 \\ \hline
Gutierrez	& G362 \\ \hline
Pfister	 	& P236 \\ \hline
Jackson	 	& J250 \\ \hline
Tymczak	 	& T522 \\ \hline
Ashcraft	& A261 \\ \hline
\end{tabular}
\end{center}
}
\end{quote}

\begin{exercise}

Write a program that contains a function which is passed a name as a
string, and returns the soundex code for it.  Using \verb^assert()^ tests,
check it works correctly for, amongst others, all the examples above.

\end{exercise}

\input{fibword_strings}
\nsection{Merging Strings}

\toohard 

\begin{exercise}
Write the function \verb^strmerge()^ which concatenetes \verb^s1^ and \verb^s2^ into \verb^s3^,
discarding any overlapping characters which are common to the end (tail) of \verb^s1^ and the 
beginning (head) of \verb^s2^.
To help, I've put some skeleton code in the usual place online.

Hint : Functions such as \verb^strncmp()^ and \verb^strcat^ might be useful.
\end{exercise}

\nsection{Rot18}



ROT18 is an encryption algorithm made of the combination of ROT13 and
ROT5. ROT13 replaces each letter in a string with the letter 13 places
later in the alphabet, wrapping back to the beginning of the alphabet
where necessary e.g. 'a' becomes 'n', 'b' becomes 'o', and 'n' becomes
'a'. ROT5 replaces each digit in a string with the digit 5 greater,
wrapping back to 0 where necessary e.g. '0' becomes '5', '1' becomes
'6', and '5' becomes '0'.

\wwwurl{https://www.boxentriq.com/code-breaking/rot13}

\begin{exercise}

Write the function whose "top-line" is:
\begin{codesnippet}
void rot(char str[])
\end{codesnippet}

\noindent that takes a string as input and encrypts it using the ROT18
cipher. The function must preserve the case of any letters, and any
characters that aren't letters or digits must be unaffected.

When the string \verb^"Hello, World!"^ is encrypted, it becomes
\verb^"Uryyb, Jbeyq!"^ and when ecrypted a second time, reverts to the
original string.


\begin{verbatim}
Erzrzore:  fubeg shapgvbaf; ubhfr-fglyr ehyrf :-)
\end{verbatim}

\end{exercise}



%%%%%%%%%%%%%%%%%%%%%%%%%%%% WEEK 4 %%%%%%%%%%%%%%%%%%%%%%%%%%%%%%%%%%%%%
\chapterimage{../Pictures/grids.pdf}
\chapter{$2D$ Arrays}

\nopartorder
\input{crossword}
\input{binarygrid}
\nsection{The Game of Life}

\input{tikzsets}

The Game of Life was developed by British mathematician
John Horton Conway. In Life, a board represents the world
and each cell a single location. A cell may be either empty
or inhabited. The game has three simple rules, which relate to the
cell's eight nearest neighbours~:
\begin{enumerate}
\item {\bf Survival} An inhabited cell remains inhabited if
exactly $2$ or $3$ of its neighbouring cells are inhabited.
\item {\bf Death} An inhabited cell becomes uninhabited if 
fewer than $2$, or more than $3$ of its neighbours are inhabited.
\item {\bf Birth} An uninhabited cell becomes inhabited if exactly
$3$ of its neighbours are inhabited.
\end{enumerate}

The next board is derived solely from the current one. The current board
remains unchanged while computing the next board.  In the simple case
shown here, the boards alternate infinitely between these two states.

\begin{center}
\begin{tikzpicture}
\matrix [noughtsone board]
{
 & & & &  \\
 & &1& &  \\
 & &1& &  \\
 & &1& &  \\
 & & & &  \\
};
\end{tikzpicture}
\begin{tikzpicture}
\matrix [noughtsone board]
{
 & & & &  \\
 & & & &  \\
 &1&1&1&  \\
 & & & &  \\
 & & & &  \\
};
\end{tikzpicture}
\end{center}

\subsection*{The 1.06 format}

A general purpose way of encoding the input board is
called the Life 1.06 format~:
\wwwurl{http://conwaylife.com/wiki/Life_1.06}
This format has comments indicated by a hash in the first column,
and the first line is always:
\begin{terminaloutput}
#Life 1.06
\end{terminaloutput}
Every line specifies an $x$ and $y$ coordinate of a live cell;
such files can be quite long.
The coordinates specified are relative to the middle of the board,
so~:
\begin{verbatim}
0  -1
\end{verbatim}
means the middle row, one cell to the left of the centre.

There are hundreds of interesting patterns stored like this
on the above site.

\begin{exercise}
\label{ex:life106}
Write a program which is run using the \verb^argc^ and
\verb^argv^ parameters to \verb^main^. The usage is
as follows~:
\begin{terminaloutput}
% life file1.lif 10
\end{terminaloutput}
where \verb^file1.lif^ is a file specifying the initial
state of the board, and \verb^10^ specifies that ten
iterations are required.

Display the output to screen every iteration using plain text,
you may assume that the board is $150$ cells wide and $90$ cells tall.
\end{exercise}

\subsection*{Alternative Rules for The Game of Life}

The rules for life could also be phrased in a different manner, that
is, give birth if there are two neighbours around an empty cell (B2)
and allow an `alive' cell to survive only if surrounded by 2 or 3 cells (S23).
Other rules which are {\it life-like} exist,
for instance {\it 34 Life} (B34/S34), {\it Life Without Death} (B3/S012345678)
and {HighLife} (B36/S23).
\wwwurl{http://en.wikipedia.org/wiki/Life-like_cellular_automaton}

\begin{exercise}
Write a program that allows the user to input life-like rules e.g.~:
\begin{terminaloutput}
life B34/S34 lifeboard.lif
\end{terminaloutput}
or
\begin{terminaloutput}
life B2/S lifeboard.lif
\end{terminaloutput}
and display generations of boards, beginning with the inital board in
the input file.
\end{exercise}

\input{lifewars}
\input{wireworld}
\input{forestfire}
\nsection{Diffusion Limited Aggregation}

\tikzset{twocolour board/.style={
  matrix of nodes
  , execute at empty cell={\node[text=white,fill=white]{+};}
  , nodes in empty cells=false
  , nodes={draw=gray,fill=ocre,minimum width=#1,minimum height=#1,outer sep=0pt,align=center,inner sep=0pt,font=\tiny}
  , text=ocre
  , row sep={#1,between origins}
  , column sep={#1,between origins}
},
  twocolour board/.default=0.2cm
}

\wwwurl{https://en.wikipedia.org/wiki/Diffusion-limited_aggregation}

\begin{quote}
In its simplest form, DLA occurs on a grid of square
cells.  The cell at the center of the grid is the location of the
seed point, a particle stuck at that square.  Now pick a square on the
perimeter of the grid and place a wandering particle on that square.
At each iteration, this particle moves to one of the four
adjacent squares, left, right, above, or below.
When a wandering particle arrives at one of the four squares adjacent
to the seed, it sticks there forming a cluster of two particles, and
another edge particle is released.  When a moving particle arrives at
one of the squares adjacent to the cluster, it sticks there.
\end{quote}

\begin{center}
\begin{tikzpicture}
\matrix [twocolour board]
{
&&&&&&&&&&&&&&&&&&&&&&&&&&&&&&&&&&&&&&&&&&&&&&&&&&\\
&&&&&&&&&&&&&&&&&&&&&&&@&&&&&&&&&&&&&&&&&&&&&&&&&&&\\
&&&&&&&&&&&&&&&&&&&&&&@&@&@&@&@&@&&&&&&&&&&&&&&&&&&&&&&&\\
&&&&&&&&&&&&&&&&&&&&&&&&&@&&&&&&&&&&&&&&&&&&&&&&&&&\\
&&&&&&&&&&&&&&&&&&&&&&&&&@&&&&@&&&&&&&&&&&&&&&&&&&&&\\
&&&&&&&&&&&&&&&&&&&&&&&&&@&@&@&&@&@&&&&&&&&&&&&&&&&&&&&\\
&&&&&&&&&&&&&&&&&&&&&@&&&&&@&@&@&@&&&&&&&&&&&&&&&&&&&&&\\
&&&&&&&&&&&&&&&&&&&&@&@&&&&&&@&&&&@&&&&&&&&&&&&&&&&&&&\\
&&&&&&&&&&&&&&&&&&&&@&@&&&&&&@&&@&@&@&&&&&&&&&&&&&&&&&&&\\
&&&&&&&&&&&&&&&&&&&&&@&&&&&@&@&&@&&&&&&&&&&&&&&&&&&&&&\\
&&&&&&&&&&&&&&&&&&&&&@&@&@&&&&@&@&@&@&&&&&&&&&&&&&&&&&&&&\\
&&&&&&&&&&&&&&&&&&&&&&&@&@&@&&@&@&&&&@&&@&&&&&&&&&&&&&&&&\\
&&&&&&&&&&&&&&&&&&&&@&@&@&@&&@&@&@&@&@&@&@&@&@&@&&&&&&&&&&&&&&&&\\
&&&&&&&&&&&&&&&&&&&&&&&&&&@&&&&@&&@&&&&&&&&&&&&&&&&&&\\
&&&&&&&&&&&&&&&&&&&&&&&&&&@&&&&&&@&&&&&&&&&&&&&&&&&&\\
&&&&&&&&&&&&@&&&&&&&@&@&&&&&&@&@&&&&&&&&&&&&&&&&&&&&&&&\\
&&&&&&&&&&&&@&&&&&&&&@&&@&&@&&@&&&&&&&&&&&&&&&&&&&&&&&&\\
&&&&&&&&&&&@&@&&&&&&@&@&@&@&@&@&@&@&@&@&&&&&&&&&&&&&&&&&&&&&&&\\
&&&&&&&&&&&&@&@&&&&&&&&&&&&@&@&&&&&&&&&&&&&&&&&&&&&&&&\\
&&&&&&&&&&&&&@&&@&@&@&@&@&@&@&&&@&@&@&&&&&&&&&&&&&&&&&&&&&&&&\\
&&&&&&&&&@&@&@&@&@&@&@&@&@&&@&@&@&@&@&@&@&@&@&@&@&@&@&&&&&&&&&&&&&&&&&&&\\
&&&&&&&&@&@&&@&&@&&&&@&&&&&&&&@&&&&&@&&&&&&&&&&&&&&&&&&&&\\
&&&&&&&&&&&&&&&&@&@&&&&&&&&@&@&&&&&&&&&&&&&&&&&&&&&&&&\\
&&&&&&&&&&&&&&&&&&&&&&&&&@&&&&&&&@&&&&&&&&&&&&&&&&&&\\
&&&&&&&&&&&&&&&&&&&&&&&&&@&&&&&&@&@&&&&&&&&&&&&&&&&&&\\
&&&&&&&&&&&&&&&&&&&&@&&&&@&@&&&@&@&@&@&&&&&&&&&&&&&&&&&&&\\
&&&&&&&&&&&&&&&&@&&&&@&@&@&@&@&@&@&@&@&@&&@&&&&&&@&@&&&&&&&&&&&&\\
&&&&&&&&&&&&&&&&@&@&@&@&@&@&&&@&&&@&&&&&&&&&&@&@&&&&&&&&&&&&\\
&&&&&&&&&&&&&&&&&&@&&&&&@&@&@&&@&@&&&&&&&&@&@&&&&&&&&&&&&&\\
&&&&&&&&&&&&&&&&&&@&&&&&@&&@&&&@&@&&&@&&&&@&&&&&&&&&&&&&&\\
&&&&&&&&&&&&&&&@&@&@&@&&&&&&&@&&&@&@&&&@&&&@&@&&&&&&&&&&&&&&\\
&&&&&&&&&&&&&&&&&&@&&&&&&&&&&@&@&@&@&@&@&&@&@&&@&&@&&&&&&&&&&\\
&&&&&&&&&&&&&&&&&@&@&@&&&&&&&&&@&&&@&&@&@&@&@&@&@&@&@&@&&&&&&&&&\\
&&&&&&&&&&&&&&&&&&@&@&&&&&&&&&@&@&@&@&&&@&@&&@&&@&&@&@&@&&&&&&&\\
&&&&&&&&&&&&&&&&&&&@&&&&&&&@&@&@&&@&&&&@&&&@&&@&&@&@&@&@&&&&&&\\
&&&&&&&&&&&&&&&&&&&@&&&&&&&&&&&&&&&&&&&&&&&@&&&&&&&&\\
&&&&&&&&&&&&&&&&&&&&&&&&&&&&&&&&&&&&&&&&&&@&&&&&&&&\\
&&&&&&&&&&&&&&&&&&&&&&&&&&&&&&&&&&&&&&&&&&&&&&&&&&\\
};
\end{tikzpicture}
\end{center}

\noindent Note that the grid is toroidal - this means that if a particle goes off the top of the grid, it reappears at the bottom. Likewise, if it goes off the edge, it will appear on the other side.

\begin{exercise}
\label{ex:dla}
Using the algorithm outlined above, using a $50 \times 50$ board,
output each of $250$ iterations
(here iteration means when each particle has become `stuck').
The output should be in plain text.
\end{exercise}


In the basic version of a DLA, a particle `sticks' to the seed points when
it gets adjacent to one. Here we introduce the probability of stickiness,
a number between $0$ and $1$, called $p_s$. The particle only sticks
to a seed point with a probability of $p_s$; if
it doesn't stick it continues wandering. This allows for the patterns
that emerge to be more hairy and solid:
\wwwurl{http://paulbourke.net/fractals/dla/}

\begin{exercise}
Implement this stickiness concept by extending the program written for Exercise~\ref{ex:dla}, allowing the user to specify a `stickiness' probability via the command line using argv[1]. Here a value of $1.0$ means that the particle will always stick, and $0.5$ means that it will stick one time in $2$.
\end{exercise}
\noindent Here's an example when setting $p_s=0.25$:
\begin{center}
\begin{tikzpicture}
\matrix [twocolour board]
{
&&&&&&&&&&&&&&&&&&&&&&&&&&&&&&&&&&&&&&&&&&&&&&&&&&\\
&&&&&&&&&&&&&&&&&&&&&&&&&&&&&&&&&&&&&&&&&&&&&&&&&&\\
&&&&&&&&&&&&&&&&&&&&&&&&&&&&&&&&&&&&&&&&&&&&&&&&&&\\
&&&&&&&&&&&&&&&&&&&&&&&&&&&&&&&&&&&&&&&&&&&&&&&&&&\\
&&&&&&&&&&&&&&&&&&&&&&&&&&&&&&&&&&&&&&&&&&&&&&&&&&\\
&&&&&&&&&&&&&&&&&&&&&&&&&&&&&&&&&&&&&&&&&&&&&&&&&&\\
&&&&&&&&&&&&&&&&&&&&&&&&&&&&&&&&&&&&&&&&&&&&&&&&&&\\
&&&&&&&&&&&&&&&&&&&&&&&&&&&&&&&&&&&&&&&&&&&&&&&&&&\\
&&&&&&&&&&&&&&&&&&&&&&&&&&&&&&&&&&&&&&&&&&&&&&&&&&\\
&&&&&&&&&&&&&&&&&&&&&&&&&&&&&&&&&&&&&&&&&&&&&&&&&&\\
&&&&&&&&&&&&&&&&&&&&&&&&&&&&&&&&&&&&&&&&&&&&&&&&&&\\
&&&&&&&&&&&&&&&&&&&&&&&&&&&&&&&&&&&&&&&&&&&&&&&&&&\\
&&&&&&&&&&&&&&&&&&&&&&&&&&&&&&&&&&&&&&&&&&&&&&&&&&\\
&&&&&&&&&&&&&&&&&&&&&&&&&&&&&&&&&&&&&&&&&&&&&&&&&&\\
&&&&&&&&&&&&&&&&&&&&&&&&&&&&&&&&&&&&&&&&&&&&&&&&&&\\
&&&&&&&&&&&&&&&&&&&&&&&&&&&&&&@&@&&&&&&&&&&&&&&&&&&&\\
&&&&&&&&&&&&&&&&&&&&&&&&@&&@&&@&@&@&@&@&&&@&&&&&&&&&&&&&&&\\
&&&&&&&&&&&&&&&&&&&&&&&&@&@&@&&&&@&@&&&&@&@&@&&&&&&&&&&&&&\\
&&&&&&&&&&&&&&&&&&&&&&&&&&@&@&&&@&@&@&@&@&@&@&&&&&&&&&&&&&&\\
&&&&&&&&&&&&&&&&&&&&&&&&&&@&@&&@&@&@&@&&@&&@&&&&&&&&&&&&&&\\
&&&&&&&&&&&&&&&&&&&&&&&&&&@&@&@&@&&@&&&&&&&&&&&&&&&&&&&\\
&&&&&&&&&&&&&&&&&&&&&&@&@&&&@&@&&&&&&&@&&&&&&&&&&&&&&&&\\
&&&&&&&&&&&&&&&&&&&&&&@&@&@&&@&@&@&&&&@&&@&&&&&&&&&&&&&&&&\\
&&&&&&&&&&&&&&&&&&&&&&@&@&@&&@&&@&@&@&@&@&@&@&&&&@&&&&&&&&&&&&\\
&&&&&&&&&&&&&&&&&&&&&@&@&@&@&@&@&@&@&&&&&&@&&&@&@&&&&&&&&&&&&\\
&&&&&&&&&&&&&&&&&&&&&@&@&@&@&@&&&&&&&&&@&@&@&@&&&&&&&&&&&&&\\
&&&&&&&&&&&&&&&&&&&&&&&&@&@&&&@&&&&&&@&&&&&&&&&&&&&&&&\\
&&&&&&&&&&&&&&&&&&&&&&&&&@&&&@&@&&&&&@&@&&&&&&&&&&&&&&&\\
&&&&&&&&&&&&&&&&&&@&@&@&&&&&@&&@&@&&&&&&&&&&&&&&&&&&&&&&\\
&&&&&&&&&&&&&&&&&&@&@&@&@&@&@&@&@&@&@&@&&&&&&&&&&&&&&&&&&&&&&\\
&&&&&&&&&&&&&&&&&&&&@&&@&@&@&@&&&&&&&&&&&&&&&&&&&&&&&&&\\
&&&&&&&&&&&&&&&@&&&&@&@&@&&@&&@&@&@&@&&&&&&&&&&&&&&&&&&&&&&\\
&&&&&&&&&&&&&&&@&&&&@&@&@&@&@&&@&@&&&&&&&&&&&&&&&&&&&&&&&&\\
&&&&&&&&&&&&&&@&@&&&@&@&@&@&@&&&&@&@&&&&&&&&&&&&&&&&&&&&&&&\\
&&&&&&&&&&&&@&@&@&@&@&@&@&@&&@&&&&&@&@&&&&&&&&&&&&&&&&&&&&&&&\\
&&&&&&&&&&&&&@&&&@&@&&&@&@&&&&&&@&@&&&&&&&&&&&&&&&&&&&&&&\\
&&&&&&&&&&&&&&&&@&&&&&@&@&&&&@&@&@&@&&&&&&&&&&&&&&&&&&&&&\\
&&&&&&&&&&&&&&&&@&&&&&@&&&&@&@&&@&@&&&&&&&&&&&&&&&&&&&&&\\
&&&&&&&&&&&&&&&&&&&&&&&&&@&&&@&@&&&&&&&&&&&&&&&&&&&&&\\
&&&&&&&&&&&&&&&&&&&&&&&&&@&&&&@&@&&&&&&&&&&&&&&&&&&&&\\
&&&&&&&&&&&&&&&&&&&&&&&&&&&&&&@&@&@&&&&&&&&&&&&&&&&&&\\
&&&&&&&&&&&&&&&&&&&&&&&&&&&&&&&@&&&&&&&&&&&&&&&&&&&\\
&&&&&&&&&&&&&&&&&&&&&&&&&&&&&&&&&&&&&&&&&&&&&&&&&&\\
&&&&&&&&&&&&&&&&&&&&&&&&&&&&&&&&&&&&&&&&&&&&&&&&&&\\
&&&&&&&&&&&&&&&&&&&&&&&&&&&&&&&&&&&&&&&&&&&&&&&&&&\\
&&&&&&&&&&&&&&&&&&&&&&&&&&&&&&&&&&&&&&&&&&&&&&&&&&\\
&&&&&&&&&&&&&&&&&&&&&&&&&&&&&&&&&&&&&&&&&&&&&&&&&&\\
&&&&&&&&&&&&&&&&&&&&&&&&&&&&&&&&&&&&&&&&&&&&&&&&&&\\
&&&&&&&&&&&&&&&&&&&&&&&&&&&&&&&&&&&&&&&&&&&&&&&&&&\\
&&&&&&&&&&&&&&&&&&&&&&&&&&&&&&&&&&&&&&&&&&&&&&&&&&\\
};
\end{tikzpicture}
\end{center}

Using different planar walks. e.g. in base-4:
\verb^http://demonstrations.wolfram.com/UsingIrrationalSquareRootsToCreatePlanarWalks/^

\input{langton}
\nsection{Minesweeper}

The game {\it Minesweeper}
\wwwurl{https://en.wikipedia.org/wiki/Minesweeper_(video_game)}
is a logic puzzle game, played on a two-dimensional
grid of squares. There are `mines' hidden in the grid, and other, numbered squares,
tell you how many mines there are in that square's (eight-count) Moore neighbourhood.
\wwwurl{https://en.wikipedia.org/wiki/Moore_neighborhood}

You will know in advance the width and height of the grid, and also the
total number of mines in the completed (solved) grid.  In our version of
the game, we'll be using just two rules to `solve' the grid by working
out the unknown squares.
\\[1em]
{\bf Rule $1$ (Count the Mines)}: If we've discovered all the mines on the board already, then any unknown cell
can simply be numbered with the count of mines in its Moore neighbourhood.

In the following grid,
if we somehow know in advance that the total number of mines in the grid is five, and if these have all been found:\\
\begin{center}
\begin{tikzpicture}
\matrix[matrix of nodes,nodes={draw=black, anchor=center, minimum size=.6cm,fill=gray!10}, column sep=-\pgflinewidth, row sep=-\pgflinewidth, , execute at empty cell={\node[draw=black,text=black,fill=ocre!40]{?};} ] (A) {
0&1&1& &0\\
1&3&\textcolor{red}{X}&3&1\\
1&\textcolor{red}{X}&\textcolor{red}{X}&\textcolor{red}{X}&1\\
1&3&\textcolor{red}{X}&3&1\\
0&1&1&1&0\\
};
\end{tikzpicture}
\end{center}

\noindent then the solution to the unknown square must be:\\
\begin{center}
\begin{tikzpicture}
\matrix[matrix of nodes,nodes={draw=black, anchor=center, minimum size=.6cm,fill=gray!10}, column sep=-\pgflinewidth, row sep=-\pgflinewidth, , execute at empty cell={\node[draw=black,text=black,fill=ocre!40]{.};} ] (A) {
0&1&1&1&0\\
1&3&\textcolor{red}{X}&3&1\\
1&\textcolor{red}{X}&\textcolor{red}{X}&\textcolor{red}{X}&1\\
1&3&\textcolor{red}{X}&3&1\\
0&1&1&1&0\\
};
\end{tikzpicture}
\end{center}

{\bf Rule $2$ (Unknowns are Mines)}: For a known square having the number $n$, with $u$ unknown and $m$ known mines in its
Moore neighbourhood, if $n = m + u$ and $u > 0$ then all $u$ unknown squares should be made mines.

\noindent Applying Rule $2$ to this grid:\\
\begin{center}
\begin{tikzpicture}
\matrix[matrix of nodes,nodes={draw=black, anchor=center, minimum size=.6cm,fill=gray!10}, column sep=-\pgflinewidth, row sep=-\pgflinewidth, , execute at empty cell={\node[draw=black,text=black,fill=ocre!40]{?};} ] (A) {
0&1&1&1&0\\
1&3&\textcolor{red}{X}&3&1\\
1&\textcolor{red}{X}&\textcolor{red}{X}&\textcolor{red}{X}&1\\
1&3& &3&1\\
0&1&1&1&0\\
};
\end{tikzpicture}
\end{center}

\noindent at the middle square on the bottom row (for instance), yields:\\
\begin{center}
\begin{tikzpicture}
\matrix[matrix of nodes,nodes={draw=black, anchor=center, minimum size=.6cm,fill=gray!10}, column sep=-\pgflinewidth, row sep=-\pgflinewidth, , execute at empty cell={\node[draw=black,text=black,fill=ocre!40]{.};} ] (A) {
0&1&1&1&0\\
1&3&\textcolor{red}{X}&3&1\\
1&\textcolor{red}{X}&\textcolor{red}{X}&\textcolor{red}{X}&1\\
1&3&\textcolor{red}{X}&3&1\\
0&1&1&1&0\\
};
\end{tikzpicture}
\end{center}

\noindent {\bf Repeated} application of these rules will allow some (but not all) boards to be solved.

\begin{exercise}

Code for this exercise can be found in \wwwurl{https://github.com/csnwc/Exercises-In-C}
\noindent then navigate into \verb^Code/Week4/Minesweeper^.

Complete the file {\em ms.c} which, along with my files {\em
ms.h} and {\em drv.c}, allows the puzzles to be solved.

\noindent My file {\em ms.h} contains the function definitions
which you'll have to implement in your {\bf ms.c} file.  My file
{\em drv.c} contains the \verb^main()^ function to act as
a driver to run the code.  Your file will contain many other functions
as well as those specified, so you'll wish to test them as normal using
the \verb^test()^ function.

\noindent If all of these files are in the same directory, you can
compile them using the \verb^Makefile^ given.

\noindent Do not alter or resubmit {\em ms.h} or {\em
drv.c} - my original versions will be used to compile the
{\em ms.c} file that you create. Only submit {\em ms.c}

\end{exercise}

%%%%%%\input{ncurses}

%%%%%%%%%%%%%%%%%%%%%%%%%%%% WEEK 5 %%%%%%%%%%%%%%%%%%%%%%%%%%%%%%%%%%%%%
\chapterimage{../Pictures/pet.png}
\chapter{Files, argc and Graphics}

Data for these exercises are available in the github repository {\em
Data}, including lists of English words in {\em Data/Words}.

\nsection{Anagrams}

An anagram is a collection of letters that
when unscrambled, using all the letters, make
a single word. For instance \verb^magrana^ can
be rearranged to make the word \verb^anagram^.

\begin{exercise}
Using a file of valid words, allow the user
to enter an anagram, and have the answer(s)
printed. For instance~:
\begin{terminaloutput}
 % ./anagram sternaig
angriest
astringe
ganister
gantries
ingrates
rangiest
reasting
stearing
\end{terminaloutput}

An anagram is defined here to be a {\em different} word, so if a valid word is 
entered, it itself is not reported, e.g.~:
\begin{terminaloutput}
 % ./anagram pots
tops
stop
opts
spot
post
\end{terminaloutput}

\end{exercise}

\begin{exercise}
Using a file of valid words, find all
words which are anagrams of each other.
Each word should appear in a maximum of one list.
Output will look something like~:
\begin{terminaloutput}
 % ./selfanagram
.
.
7 merits mister miters mitres remits smiter timers
.
.
.
6 noters stoner tenors tensor toners trones
.
.
.
6 opts post pots spot stop tops
.
.
.
6 restrain retrains strainer terrains trainers transire
.
\end{terminaloutput}
\end{exercise}

If you wished to create ``interesting'' anagrams, rather than simply
a random jumble of letters, you could combine together two
shorter words which are an anagram of a longer one.

\begin{exercise}
Write a program which uses an exhaustive
search of all the possible pairs of short words to make the target word
to be computed.
For instance, a few of the many pairs that
can be used to make an anagram of \verb^compiler^ are~:
\begin{terminaloutput}
 % ./teabreak compiler
LiceRomp
LimeCrop
LimpCore
MileCrop
MoreClip
PermCoil
PromLice
RelicMop
\end{terminaloutput}
\end{exercise}

The name \verb^Campbell^ comes out as \verb^CalmPleb^ which is a
bit harsh. Can't {\bf ever} remember being called calm ... 

\input{ansi}
\nsection{SDL - Intro}

Many programming languages have no inherent graphics capabilities.
To get windows to appear on the screen, or to draw lines and shapes,
you need to make use of an external library. Here we use SDL\footnote{
actually, we are using the most recent version SDL2, which is installed
on all the lab machines}, a cross-platform library providing the user with
(amongst other things) such graphical capabilities.

\wwwurl{https://www.libsdl.org/}

The use of SDL is, unsurprisingly, non-trivial, so some simple wrapper
files have been created (\verb^neillsdl2.c^ and \verb^neillsdl2.h^).
These give you some simple functions to initialise a window, draw
rectangles, wait for the user to press a key etc.

An example program using this functionality is
provided in a file \verb^demo_neillsdl2.c^.

This program initialises a window, then sits in a loop, drawing
randomly positioned and coloured squares, until the
user presses the mouse or a key. 

\begin{exercise}
Using the \verb^Makefile^ provided, compile and run this program.
Now adapt it, so that the colour of the boxes displayed are all
(random) shades of red.

SDL is already installed on lab machines. At home, if you're using a
ubuntu-style linux machine, use: \verb^sudo apt install libsdl2-dev^
to install it.
\end{exercise}

\nsection{A Simple Spelling Checker}

Here, we reads words one at a time
from file and carefully place them in the {\bf correct} part of
our data structure. This has a complexity of $O(n^2)$.

For this purpose, a list of valid words (unsorted) is available
from \verb^github : Data/Words^.

\begin{exercise}
\label{ex:arrayinsertsort}
Write a program which, based on a
fixed-size {\bf array} of strings, reads the words in
one at a time, inserting them into the {\bf correct} part of the array
so that the words are alphabetically sorted.
The name of the file should be passed as \verb^argv[1]^,
and you can assume the array is large enough to hold all words.
How long does it take to build the list~?
\end{exercise}

\begin{exercise}
\label{ex:linearsearch}
Now extend Exercise~\ref{ex:arrayinsertsort}
so that when the user is prompted for a word,
they are told whether this word is present in the array or not.
Use a {\em Linear Search} for this: start at the beginning of the
list and work your way through it one word at a time.
\end{exercise}

\begin{exercise}
\label{ex:binarysearch}
Extend Exercise~\ref{ex:linearsearch} using a
{\em Binary Search} for this.
\end{exercise}

\begin{exercise}
\label{ex:linkedinsertsort}
Write a program which, based on a
linked list data structure, reads the words in
one at a time, inserting them into the {\bf correct} part of the list
so that the words are alphabetically sorted.
The name of the file should be passed as \verb^argv[1]^.
How long does it take to build the list~?\\
How long does it take to do a linear search~?
\end{exercise}

\nsection{Crush It!}

Match-3 tile games have become one of the world's most popular games.
\wwwurl{https://en.wikipedia.org/wiki/Tile-matching_video_game}
Such games use a rectangular grid (board)
containing many tiles, of many different types
(often colour, but here we will use letters). Where there are $3$ or
more tiles of the same type in a line (horizontally or vertically),
they are removed. Once all removals that are possible have occurred,
the tiles above fall down to fill in the gaps. In our version, a large
number of tiles are available above (and hidden from the player) which
cannot be matched until they have fallen down into the playing area.

In our version of the game:
\begin{itemize}
\item Matchable tiles are in the range $(A \ldots Z)$,
although it's is common for only a small
number (e.g. $A \ldots D$) to be used.
\item The width of the board is always five tiles.
\item The `playing' height of the board is six. Other tiles can be above this,
but won't be matched until they have dropped down into one of the bottom six rows.
\item The maximum number of rows the board ever needs to hold is $20$. 
\item Matching (that is finding a horizonal or vertical line of the same tile)
is done in `parallel' - if a tile is shared between two matches
(e.g. the middle tile in a $3\times 3$ `+' pattern) both of these matches are
removed.
\item Given the limited size of board in which matches can be made, the longest
match that can be made is of five tiles horizontally (the width of the board) and
six tiles vertically the `playing' height of the board.
\end{itemize}

An example of this is shown here:
\begin{center}
\begin{tikzpicture}
\matrix[matrix of nodes,nodes={draw=black, anchor=center, minimum size=.6cm,fill=ocre!30}, column sep=-\pgflinewidth, row sep=-\pgflinewidth, , execute at empty cell={\node[draw=black,text=black,fill=gray!20]{.};} ] (A) {
 & & & & \\
\textcolor{red}{B}&\textcolor{red}{B}&\textcolor{red}{B}&D&B\\
C&D&A&A&C\\
D&A&A&B&D\\
A&A&B&C&\textcolor{red}{A}\\
A&B&C&D&\textcolor{red}{A}\\
B&C&\textcolor{red}{A}&\textcolor{red}{A}&\textcolor{red}{A}\\
};
\end{tikzpicture}
\begin{tikzpicture}
\matrix[matrix of nodes,nodes={draw=black, anchor=center, minimum size=.6cm,fill=ocre!30}, column sep=-\pgflinewidth, row sep=-\pgflinewidth, , execute at empty cell={\node[draw=black,text=black,fill=gray!20]{.};} ] (A) {
 & & & & \\
.&.&.&D&B\\
C&D&A&A&C\\
D&A&A&B&D\\
A&A&B&C&.\\
A&B&C&D&.\\
B&C&.&.&.\\
};
\end{tikzpicture}
\begin{tikzpicture}
\matrix[matrix of nodes,nodes={draw=black, anchor=center, minimum size=.6cm,fill=ocre!30}, column sep=-\pgflinewidth, row sep=-\pgflinewidth, , execute at empty cell={\node[draw=black,text=black,fill=gray!20]{.};} ] (A) {
 & & & & \\
.&.&.&.&.\\
C&D&.&D&.\\
D&A&A&A&.\\
A&A&A&B&B\\
A&B&B&C&C\\
B&C&C&D&D\\
};
\end{tikzpicture}
\end{center}
\noindent (Left) Game board in it's initial state. Orange squares show were matches can be made. (Middle) Three matches are made - one for the $3$ horizonal `A' tiles, one for the three vertical `A' tiles and one for the three horizontal `B' tiles. (Right) Game board final state after tiles are dropped down. Not there are now more matches that can be made.


\begin{exercise}

Here we will write some (but not all) of the functionality necessary
for a match-3 tile game.  Skeleton code may be found in~:
\wwwurl{https://github.com/csnwc/Exercises-In-C}
\noindent then navigate into \verb^Code/Week5/CrushIt^.

Complete the files {\bf crushit.c} and {\bf mydefs.h} which, along with
my files {\em crushit.h} and {\em driver.c}, implements some important
functionality necessary for a game of this type.

\noindent My file {\em crushit.h} contains the function definitions
which you'll have to implement in your {\bf crushit.c} file.  My file
{\em driver.c} contains the \verb^main()^ function to act as
a driver to run the code.  Your file will contain many other functions
as well as those specified, so you'll wish to test them as normal using
a \verb^test()^ function.

\noindent If all of these files are in the same directory, you can
compile them using the \verb^Makefile^ given.

\noindent The functions you need to complete include:

\verb^initialise()^ - this takes a pointer to the board state, and
a string.  The string can be a filename, but if this filename can't
be opened, it is assumed to be a list of tiles to fill the board with,
from the top down.  Such a string must contain complete rows of tiles,
with no partial rows.

\verb^match()^ - this takes a pointer to the board state, and removes
all matches of $3$ or more tiles in a vertical or horizontal line.
Removed tiles are replaces with the `.' (empty tile) character.

\verb^dropblocks()^ - this takes a pointer to the board state, and makes
any blocks that are above an empty tile fall down until all holes are
filled in if it is possible to do so.

\verb^tostring()^ - this takes a pointer to the board state, and a string
and copies whole rows of the board into the string from the top downwards.
The whole board isn't copied since most of the characters at the top
are unused (hole) tiles. Therefore, we begin copying at the first row
on which a non-hole tile appears.

\noindent Hints:
\begin{itemize}
\item Do not begin by writing the file handling functionality -
this cannot be tested, so use the string initialising option
instead.
\item To begin with use your own, simpler driver file - mine
makes sense once everything is working, but may seem complex
to begin with.
\item Your \verb^crushit.c^ file should contain many other sub-functions
which are used by the major ones specified. You can put anything in this
file, provided it still compiles as specified.
\item Do not alter or resubmit {\em crushit.h}, {\em Makefile} or
{\em driver.c} - my original versions (or even slightly different
ones) will be used to compile the {\em crushit.c} file that you adapt.
\end{itemize}

\end{exercise}



%%%%%%%%%%%%%%%%%%%%%%%%%%%% WEEK 6 %%%%%%%%%%%%%%%%%%%%%%%%%%%%%%%%%%%%%

%%%%%%%%%%%%%%%%%%%%%%%%%%%% WEEK 7 %%%%%%%%%%%%%%%%%%%%%%%%%%%%%%%%%%%%%
\setcounter{chapter}{6}
\chapterimage{../Pictures/Insertionsort-edited.png}
\chapter{Recursion}

\nsection{Word Ladders}
\input{wordladder_gen}
\input{wordladder_cons}
\input{wordladder_full}
\nsection{Maze}

\newexercise{2023}

\newcommand{\W}{|[fill=ocre,text=white]|\#}
\newcommand{\G}{|[fill=green,text=white]|+}
\newcommand{\R}{|[fill=white,text=gray]|.}

\begin{tikzpicture}[every node/.style={anchor=base,text depth=.5ex,text height=2ex,text width=1em,outer sep=0pt,align=center,inner sep=0pt}]
\matrix [matrix of nodes,draw=white,nodes in empty cells]
{
\W&\W&\W&\W&\W&\W&\W&\W&\W&\W\\
\R&\R&\W&\R&\R&\R&\R&\R&\R&\W\\
\W&\R&\W&\R&\W&\R&\W&\W&\R&\W\\
\W&\R&\W&\R&\W&\W&\W&\W&\R&\W\\
\W&\R&\W&\R&\R&\R&\R&\W&\R&\W\\
\W&\R&\W&\R&\W&\W&\W&\W&\R&\W\\
\W&\R&\W&\R&\R&\R&\R&\W&\R&\W\\
\W&\R&\W&\W&\W&\W&\R&\W&\R&\W\\
\W&\R&\R&\R&\R&\R&\R&\W&\R&\R\\
\W&\W&\W&\W&\W&\W&\W&\W&\W&\W\\
};
\end{tikzpicture}

Escaping from a maze can be done in several ways (ink-blotting, righthand-on-wall etc.)
but here we look at recursion.

\begin{exercise}
\label{ex:maze_rec}
Write a program to read in a maze typed by a user via the filename passed to \verb^argv[1]^.
You can assume the maze will be no larger than $20 \times 20$,
walls are designated by a \verb^#^ and the rest are
spaces. The entrance can be assumed to be the gap in the wall closest to
(but not necessarily exactly at) the top lefthand corner.
The sizes of the maze are given on the first line of the file (width,height).
Write a program that finds the route through a maze, read from this file,
and prints out the solution (if one exists) using full stops.
If the program succeeds it should exit with a status of \verb^0^,
or if no route exists it should exit with a status of \verb^1^.
\end{exercise}


\noindent It becomes obvious that the walls of every maze (having one unique solution)
must consist of two separate sections~:

\renewcommand{\W}{|[fill=ocre,text=white]|B}
\renewcommand{\G}{|[fill=orange,text=white]|A}
\renewcommand{\R}{|[fill=white,text=gray]|.}
\begin{tikzpicture}[every node/.style={anchor=base,text depth=.5ex,text height=2ex,text width=1em,outer sep=0pt,align=center,inner sep=0pt}]
\matrix [matrix of nodes,draw=white,nodes in empty cells]
{
\G&\G&\G&\G&\G&\G&\G&\G&\G&\G\\
\R&\R&\G&\R&\R&\R&\R&\R&\R&\G\\
\W&\R&\G&\R&\W&\R&\W&\W&\R&\G\\
\W&\R&\G&\R&\W&\W&\W&\W&\R&\G\\
\W&\R&\G&\R&\R&\R&\R&\W&\R&\G\\
\W&\R&\G&\R&\W&\W&\W&\W&\R&\G\\
\W&\R&\G&\R&\R&\R&\R&\W&\R&\G\\
\W&\R&\G&\G&\G&\G&\R&\W&\R&\G\\
\W&\R&\R&\R&\R&\R&\R&\W&\R&\R\\
\W&\W&\W&\W&\W&\W&\W&\W&\W&\W\\
};
\end{tikzpicture}

\begin{exercise}
\label{ex:maze_two}
Write a program to read in a maze in the same manner as in Exercise~\ref{ex:maze_rec}, and
then display the two sections using the characters \verb^A^ and \verb^B^.  
\end{exercise}


\input{unlock}
\input {sierpinski_carpet}
\input {sierpinski_triangle}
\input{primefactors}
%\input{mo}

%%%%%%%%%%%%%%%%%%%%%%%%%%%% WEEK 8 %%%%%%%%%%%%%%%%%%%%%%%%%%%%%%%%%%%%%
\chapterimage{../Pictures/bookcase.jpg}
\chapter{Searching Boards}

\input{conway}
\input{tikzsets}
\newcommand{\board}[9]{
\begin{tikzpicture}
\matrix [eighttilestyle board]
{
#1 \& #2 \& #3 \\
#4 \& #5 \& #6 \\
#7 \& #8 \& #9 \\
 };
\end{tikzpicture}
}

\nsection{The 8-Tile Puzzle}

The Chinese 8-Tile Puzzle is a $3 \times 3$ board, with $8$ numbered
tiles in it, and a hole into which neighbouring tiles can move:
\begin{tikzpicture}[every node/.style={anchor=base,text depth=.5ex,text height=2ex,text width=1em,outer sep=0pt,align=center,inner sep=0pt}]
\matrix [matrix of nodes,draw=white,nodes in empty cells]
{
|[fill=ocre,text=black]|1&|[fill=ocre,text=black]|2&|[fill=ocre,text=black]|3 \\
|[fill=ocre,text=black]|4&|[fill=ocre,text=black]|5&|[fill=ocre,text=black]|6 \\
|[fill=ocre,text=black]|7&|[fill=ocre,text=black]|8&|[fill=gray,text=black]|  \\
};
\end{tikzpicture}

\noindent After the next move the board could look like:
\begin{tikzpicture}[every node/.style={anchor=base,text depth=.5ex,text height=2ex,text width=1em,outer sep=0pt,align=center,inner sep=0pt}]
\matrix [matrix of nodes,draw=white,nodes in empty cells]
{
|[fill=ocre,text=black]|1&|[fill=ocre,text=black]|2&|[fill=ocre,text=black]|3 \\
|[fill=ocre,text=black]|4&|[fill=ocre,text=black]|5&|[fill=gray,text=black]|  \\
|[fill=ocre,text=black]|7&|[fill=ocre,text=black]|8&|[fill=ocre,text=black]|6 \\
};
\end{tikzpicture}
or
\begin{tikzpicture}[every node/.style={anchor=base,text depth=.5ex,text height=2ex,text width=1em,outer sep=0pt,align=center,inner sep=0pt}]
\matrix [matrix of nodes,draw=white,nodes in empty cells]
{
|[fill=ocre,text=black]|1&|[fill=ocre,text=black]|2&|[fill=ocre,text=black]|3 \\
|[fill=ocre,text=black]|4&|[fill=ocre,text=black]|5&|[fill=ocre,text=black]|6 \\
|[fill=ocre,text=black]|7&|[fill=gray,text=black]| &|[fill=ocre,text=black]|8  \\
};
\end{tikzpicture}
The problem generally involves the board starting in a random state, and the user
returning the board to the `ordered' $"12345678"$ state.

In this problem, a solution could be found in many different ways; the solution could be recursive,
or you could implement a queue to perform a breadth-first search, or something more complex allowing
a depth-first search to measure `how close' (in some sense) it is to the correct solution. 

\begin{exercise}
\label{ex:basic8tile}
Read in a board using \verb^argv[1]^, e.g.:
\begin{codesnippet}
$ 8tile "513276 48"
\end{codesnippet}

To do this you will use a list of boards. The initial board is
put into this list. Each board in the list is, in turn, read from the
list and all possible moves from that board added into the list. The
next board is taken, and all its resulting boards are added, and so
on.  This is, essentially, a queue.

However, one problem with is that repeated boards may be put into the queue and
`cycles' occur.  This soon creates an explosively large number of
boards (several million).  You can solve this by only adding a board
into the queue if an identical one does not already exist in the queue.
A linear search is acceptable for this task of identifying duplicates.
Each structure in the queue will contain (amongst other things)
a board and a record of its parent board, i.e. the board that it was
created from.

Be advised that a solution requiring as `few' as $20$ moves may take
$10$'s of minutes to compute. If the search is successful, display the solution
to the screen using plain-text.

Use the method described above and only this one. Use a static data structure to achieve
this (arrays) and {\bf not} a dynamic method such as linked-lists; a (large) $1D$ array
of structures is acceptable. Because this array needs to be
so large, it's best to declare it in \verb^main()^ using something like:
\begin{codesnippet}
static boards[NUMBOARDS];
\end{codesnippet}

\end{exercise}

\begin{exercise}
Repeat Exercise~\ref{ex:basic8tile}, but use SDL for the output rather than plain-text.
\end{exercise}

\begin{exercise}
\label{ex:ll8tile}
Repeat Exercise~\ref{ex:basic8tile}, but using a dynamic (linked-list), so that you never have to make any assumptions about the maximum numbers of boards stored.
\end{exercise}

\begin{exercise}
Repeat Exercise~\ref{ex:ll8tile}, but using a $5 \times 5$ board instead.
\end{exercise}

  \newcommand{\K}{|[fill=white,text=black]|K}
\renewcommand{\R}{|[fill=black,text=red]|R}
\renewcommand{\G}{|[fill=black,text=green]|G}
  \newcommand{\Y}{|[fill=black,text=yellow]|Y}
\renewcommand{\B}{|[fill=black,text=blue]|B}
  \newcommand{\M}{|[fill=black,text=magenta]|M}
  \newcommand{\C}{|[fill=black,text=cyan]|C}
\renewcommand{\W}{|[fill=black,text=white]|W}
  \newcommand{\X}{|[fill=black,text=white]|.}

\nsection{Happy Bookcases}

In a quiet part of our building, there are some rather strange bookcases.
They are (like most bookcases) generally happy, but they become unhappy when their books are not arranged correctly (which,
even in a Computer Science Department, is somewhat unusual).  After years
of dedicated research, a team of scientists led by Simon Lock and Sion
Hannuna came to understand the trick to making the bookcases happy again.
It turned out that a bookcase is only happy if~:
\begin{itemize}
\item Each shelf only has books of one colour (or is empty).
\item All books of the same colour are on the same shelf.
\item The only books that exists are black(K), red(R), green(G),
yellow(Y), blue(B), magenta(M), cyan(C) or white(W).
\end{itemize}

\noindent
However, to make things worse, there are some complex rules about how
books may be rearranged~:
\begin{enumerate}
\item You can only move one book at a time.
\item The only book that can move is the rightmost one from each shelf.
\item The book must move to become the rightmost book on its new shelf.
\item You can't put more books on a shelf than its maximum size.
\end{enumerate}

\noindent
So, for instance, the bookcase below has three shelves,
each of which can fit three books;
the first contains only one book, the second has three books,
and the third shelf has two books on it.

\begin{tikzpicture}[every node/.style={anchor=base,text depth=.5ex,text height=2ex,text width=1em,outer sep=0pt,align=center,inner sep=0pt}] \matrix [matrix of nodes,draw=white,nodes in empty cells] {
\Y&\X&\X\\
\B&\B&\Y\\
\Y&\B&\X\\
};
\end{tikzpicture}

\noindent
By following the rules, highly-trained librarians can
rearrange the books to make the bookcase happy again.
One such way of re-arranging the books correctly is shown below~:

\begin{tikzpicture}[every node/.style={anchor=base,text depth=.5ex,text height=2ex,text width=1em,outer sep=0pt,align=center,inner sep=0pt}] \matrix [matrix of nodes,draw=white,nodes in empty cells] {
\Y&\Y&\X\\
\B&\B&\X\\
\Y&\B&\X\\
};
\end{tikzpicture}
\hspace{2em}
\begin{tikzpicture}[every node/.style={anchor=base,text depth=.5ex,text height=2ex,text width=1em,outer sep=0pt,align=center,inner sep=0pt}] \matrix [matrix of nodes,draw=white,nodes in empty cells] {
\Y&\Y&\X\\
\B&\B&\B\\
\Y&\X&\X\\
};
\end{tikzpicture}
\hspace{2em}
\begin{tikzpicture}[every node/.style={anchor=base,text depth=.5ex,text height=2ex,text width=1em,outer sep=0pt,align=center,inner sep=0pt}] \matrix [matrix of nodes,draw=white,nodes in empty cells] {
\Y&\Y&\Y\\
\B&\B&\B\\
\X&\X&\X\\
};
\end{tikzpicture}

\noindent
Here's another example of an unhappy bookcase, and how to rearrange the
books to make it happy~:

\begin{tikzpicture}[every node/.style={anchor=base,text depth=.5ex,text height=2ex,text width=1em,outer sep=0pt,align=center,inner sep=0pt}] \matrix [matrix of nodes,draw=white,nodes in empty cells] {
\R&\G&\X&\X\\
\G&\R&\X&\X\\
\K&\K&\X&\X\\
\K&\K&\X&\X\\
};
\end{tikzpicture}
\hspace*{2em}
\begin{tikzpicture}[every node/.style={anchor=base,text depth=.5ex,text height=2ex,text width=1em,outer sep=0pt,align=center,inner sep=0pt}] \matrix [matrix of nodes,draw=white,nodes in empty cells] {
\R&\X&\X&\X\\
\G&\R&\X&\X\\
\K&\K&\G&\X\\
\K&\K&\X&\X\\
};
\end{tikzpicture}
\hspace*{2em}
\begin{tikzpicture}[every node/.style={anchor=base,text depth=.5ex,text height=2ex,text width=1em,outer sep=0pt,align=center,inner sep=0pt}] \matrix [matrix of nodes,draw=white,nodes in empty cells] {
\R&\R&\X&\X\\
\G&\X&\X&\X\\
\K&\K&\G&\X\\
\K&\K&\X&\X\\
};
\end{tikzpicture}
\hspace*{2em}
\begin{tikzpicture}[every node/.style={anchor=base,text depth=.5ex,text height=2ex,text width=1em,outer sep=0pt,align=center,inner sep=0pt}] \matrix [matrix of nodes,draw=white,nodes in empty cells] {
\R&\R&\X&\X\\
\G&\G&\X&\X\\
\K&\K&\X&\X\\
\K&\K&\X&\X\\
};
\end{tikzpicture}
\hspace*{2em}
\begin{tikzpicture}[every node/.style={anchor=base,text depth=.5ex,text height=2ex,text width=1em,outer sep=0pt,align=center,inner sep=0pt}] \matrix [matrix of nodes,draw=white,nodes in empty cells] {
\R&\R&\X&\X\\
\G&\G&\X&\X\\
\K&\X&\X&\X\\
\K&\K&\K&\X\\
};
\end{tikzpicture}
\hspace*{2em}
\begin{tikzpicture}[every node/.style={anchor=base,text depth=.5ex,text height=2ex,text width=1em,outer sep=0pt,align=center,inner sep=0pt}] \matrix [matrix of nodes,draw=white,nodes in empty cells] {
\R&\R&\X&\X\\
\G&\G&\X&\X\\
\X&\X&\X&\X\\
\K&\K&\K&\K\\
};
\end{tikzpicture}


\begin{exercise}
Write a program that reads in a bookcase definition file (specified on the command line), and shows the `moves' to make the bookcase happy. Such a file looks something like~:
\begin{terminaloutput}
4 3 7
RG.
GR.
CY.
YC.
\end{terminaloutput}

\noindent The first line has two or three numbers on it; the height of
the bookcase (number of shelves), the width (maximum books per shelf)
and an {\bf optional} hint as to the minimum number of bookcases
involved when `solving' this bookcase. (This number
is meaningless for a bookcase that cannot be made happy.) The number
includes the original bookcase, and the final `happy' one in the count.
For the bookcase shown in this file, one possible solution is~:

\begin{tikzpicture}[every node/.style={anchor=base,text depth=.5ex,text height=2ex,text width=1em,outer sep=0pt,align=center,inner sep=0pt}] \matrix [matrix of nodes,draw=white,nodes in empty cells] {
\R&\G&\X\\
\G&\R&\X\\
\C&\Y&\X\\
\Y&\C&\X\\
};
\end{tikzpicture}
\hspace*{2em}
\begin{tikzpicture}[every node/.style={anchor=base,text depth=.5ex,text height=2ex,text width=1em,outer sep=0pt,align=center,inner sep=0pt}] \matrix [matrix of nodes,draw=white,nodes in empty cells] {
\R&\X&\X\\
\G&\R&\X\\
\C&\Y&\G\\
\Y&\C&\X\\
};
\end{tikzpicture}
\hspace*{2em}
\begin{tikzpicture}[every node/.style={anchor=base,text depth=.5ex,text height=2ex,text width=1em,outer sep=0pt,align=center,inner sep=0pt}] \matrix [matrix of nodes,draw=white,nodes in empty cells] {
\R&\R&\X\\
\G&\X&\X\\
\C&\Y&\G\\
\Y&\C&\X\\
};
\end{tikzpicture}
\hspace*{2em}
\begin{tikzpicture}[every node/.style={anchor=base,text depth=.5ex,text height=2ex,text width=1em,outer  sep=0pt,align=center,inner sep=0pt}] \matrix [matrix of nodes,draw=white,nodes in empty cells] {
\R&\R&\X\\
\G&\G&\X\\
\C&\Y&\X\\
\Y&\C&\X\\
};
\end{tikzpicture}
\hspace*{2em}
\begin{tikzpicture}[every node/.style={anchor=base,text depth=.5ex,text height=2ex,text width=1em,outer  sep=0pt,align=center,inner sep=0pt}] \matrix [matrix of nodes,draw=white,nodes in empty cells] {
\R&\R&\Y\\
\G&\G&\X\\
\C&\X&\X\\
\Y&\C&\X\\
};
\end{tikzpicture}
\hspace*{2em}
\begin{tikzpicture}[every node/.style={anchor=base,text depth=.5ex,text height=2ex,text width=1em,outer  sep=0pt,align=center,inner sep=0pt}] \matrix [matrix of nodes,draw=white,nodes in empty cells] {
\R&\R&\Y\\
\G&\G&\X\\
\C&\C&\X\\
\Y&\X&\X\\
};
\end{tikzpicture}
\hspace*{2em}
\begin{tikzpicture}[every node/.style={anchor=base,text depth=.5ex,text height=2ex,text width=1em,outer  sep=0pt,align=center,inner sep=0pt}] \matrix [matrix of nodes,draw=white,nodes in empty cells] {
\R&\R&\X\\
\G&\G&\X\\
\C&\C&\X\\
\Y&\Y&\X\\
};
\end{tikzpicture}

\noindent
In the file, an empty space is defined by a full-stop character.
You may assume that the maximum height and width of a bookcase is $9$.

\noindent
The brute-force algorithm for searching over all moves to make
the bookcase happy goes like this~:
\begin{enumerate}
\item You will use a list of bookcases (here list could either be an array, or a linked list).
\item The initial bookcase is put into the front of this list.
\item Take a bookcase from the {\bf front} of the list.
\item For this (parent) bookcase, find the resulting (child) bookcases
which can be created from all the valid possible single book moves. Put
each of these bookcases into the {\bf end} of the list. There may be
as many as $height \times (height-1)$ of these. If you have found a
happy bookcase, stop. Else, go to $3$.
\end{enumerate}

\noindent
To help with printing out the correct moves when a solution has been
found, each structure in the list will need to contain (amongst other
things) a bookcase and a record of its parent bookcase, i.e. the bookcase
that it was created from. For an array, this could simply be which element 
of the array was the parent, or for a linked list, this will be a pointer.

\noindent
The program reads the name of the bookcase definition file from \verb^argv[1]^.
If it finds a successful way to make the bookcase happy, it prints out
the number of bookcases that would be printed in the solution and {\bf nothing else}, or else exactly the phrase `No Solution?'' if none can be found~:
\begin{terminaloutput}
$ ./bookcase rrggccyy-437.bc
7
$ ./bookcase rrrr-22.bc
No Solution?
$ ./bookcase ccbb-23.bc
1
\end{terminaloutput}

If the `verbose' flag is used (argv[2]), your program will additionally print out the solution (reverse order is fine)~:
\begin{terminaloutput}
$ ./bookcase ccbb-23.bc verbose
1

CC.
BB.

$ ./bookcase rgbrmrykwrrr-3521.bc verbose
No Solution?

$ ./bookcase yby-222.bc verbose
2

Y.
BY

YY
B.

\end{terminaloutput}

\noindent
Your program~:
\begin{itemize}
\item {\bf Must} use the algorithm detailed above (which is similar to a queue and therefore a breadth-first search). Other search algorithms are possible (e.g. best-first, guided, recursive etc.) but the quality of coding is being assessed, not the quality of the algorithm used!
\item {\bf Should} check for invalid bookcase definition files, and report in a graceful way if there is a problem, aborting with \verb^exit(EXIT_FAILURE)^ if so.
\item {\bf May} display the bookcases in colour if you wish - if so use
\verb^neillsimplescreen^ to do so.
\item {\bf Should not} print anything else out to screen after successfully
completing the search, except that which is shown above. Automated checking
may be used, and therefore the output must be precise.
\item {\bf Should} call the function \verb^test()^ to perform any assertion
testing etc.
\end{itemize}

\subsection*{Extension}

Basic assignment = {\Large $90\%$}.
Extension = {\Large $10\%$}.

\noindent
If you'd like to try an extension, make sure to submit {\it extension.c}
and a brief description in a {\it extension.txt} file. This could
involve a faster search technique, better graphical display, user input
or something else of your choosing. The extension will be
marked in the same way as the main assignment.


\end{exercise}

\renewcommand{\A}{|[fill=black,text=white]|0}
\renewcommand{\B}{|[fill=black,text=ocre]|1}

\nsection{Roller-Board}

The puzzle {\it Roller-Board} consists of a $2D$ rectangular grid of
cells, each of which is labelled either `$0$' or `$1$':

\begin{tikzpicture}[every node/.style={anchor=base,text depth=.5ex,text height=2ex,text width=1em,outer sep=0pt,align=center,inner sep=0pt}] \matrix [matrix of nodes,draw=white,nodes in empty cells] {
\A&\A&\A&\A&\A\\
\A&\B&\A&\B&\A\\
\A&\A&\B&\A&\A\\
\A&\B&\A&\B&\A\\
\A&\A&\A&\A&\A\\
};
\end{tikzpicture}

\noindent The challenge is to roll one row or column at a time, so that the board is returned to its `correct' state:

\begin{tikzpicture}[every node/.style={anchor=base,text depth=.5ex,text height=2ex,text width=1em,outer sep=0pt,align=center,inner sep=0pt}] \matrix [matrix of nodes,draw=white,nodes in empty cells] {
\B&\B&\B&\B&\B\\
\A&\A&\A&\A&\A\\
\A&\A&\A&\A&\A\\
\A&\A&\A&\A&\A\\
\A&\A&\A&\A&\A\\
};
\end{tikzpicture}

having all $1s$ on the top row, and every other cell being a $0$.
Each `move' can be either:
\begin{itemize}
\item Roll a column one place up - i.e. the cells in this column all move up one, and the cell at the top `rolls around' and reappears at the bottom of this column.
\item Roll a column one place down - i.e. the cells in this column all move down one, and the cell at the bottom `rolls around' and reappears at the top of this column.
\item Roll a row one place left - i.e. the cells in this row all move left one, and the cell on the left `rolls around' and reappears on the right of this row.
\item Roll a row one place right - i.e. the cells in this row all right one, and the cell on the right `rolls around' and reappears on the left of this row.
\end{itemize}

To solve a $4 \times 4$ board:

\begin{tikzpicture}[every node/.style={anchor=base,text depth=.5ex,text height=2ex,text width=1em,outer sep=0pt,align=center,inner sep=0pt}] \matrix [matrix of nodes,draw=white,nodes in empty cells] {
\B&\A&\A&\A\\
\A&\B&\A&\A\\
\A&\A&\B&\A\\
\A&\A&\A&\B\\
};
\end{tikzpicture}

the best solution might be:

Roll column 1 up~:
\begin{tikzpicture}[every node/.style={anchor=base,text depth=.5ex,text height=2ex,text width=1em,outer sep=0pt,align=center,inner sep=0pt},baseline=(current bounding box.center)] \matrix [matrix of nodes,draw=white,nodes in empty cells] {
\B&\B&\A&\A\\
\A&\A&\A&\A\\
\A&\A&\B&\A\\
\A&\A&\A&\B\\
};
\end{tikzpicture}

Roll column 2 down~:
\begin{tikzpicture}[every node/.style={anchor=base,text depth=.5ex,text height=2ex,text width=1em,outer sep=0pt,align=center,inner sep=0pt},baseline=(current bounding box.center)] \matrix [matrix of nodes,draw=white,nodes in empty cells] {
\B&\B&\A&\A\\
\A&\A&\A&\A\\
\A&\A&\A&\A\\
\A&\A&\B&\B\\
};
\end{tikzpicture}

Roll column 2 down~:
\begin{tikzpicture}[every node/.style={anchor=base,text depth=.5ex,text height=2ex,text width=1em,outer sep=0pt,align=center,inner sep=0pt},baseline=(current bounding box.center)] \matrix [matrix of nodes,draw=white,nodes in empty cells] {
\B&\B&\B&\A\\
\A&\A&\A&\A\\
\A&\A&\A&\A\\
\A&\A&\A&\B\\
};
\end{tikzpicture}

Roll column 3 down~:
\begin{tikzpicture}[every node/.style={anchor=base,text depth=.5ex,text height=2ex,text width=1em,outer sep=0pt,align=center,inner sep=0pt},baseline=(current bounding box.center)] \matrix [matrix of nodes,draw=white,nodes in empty cells] {
\B&\B&\B&\B\\
\A&\A&\A&\A\\
\A&\A&\A&\A\\
\A&\A&\A&\A\\
};
\end{tikzpicture}



Here's another example of a board:

\begin{tikzpicture}[every node/.style={anchor=base,text depth=.5ex,text height=2ex,text width=1em,outer sep=0pt,align=center,inner sep=0pt}] \matrix [matrix of nodes,draw=white,nodes in empty cells] {
\A&\A&\A&\A\\
\B&\B&\A&\A\\
\B&\A&\A&\A\\
\A&\B&\A&\A\\
\A&\A&\A&\A\\
};
\end{tikzpicture}

and how to solve it:

\begin{tikzpicture}[every node/.style={anchor=base,text depth=.5ex,text height=2ex,text width=1em,outer sep=0pt,align=center,inner sep=0pt}] \matrix [matrix of nodes,draw=white,nodes in empty cells] {
\A&\A&\A&\A\\
\B&\B&\A&\A\\
\B&\A&\A&\A\\
\A&\B&\A&\A\\
\A&\A&\A&\A\\
};
\end{tikzpicture}
\hspace{0.25ex}
\begin{tikzpicture}[every node/.style={anchor=base,text depth=.5ex,text height=2ex,text width=1em,outer sep=0pt,align=center,inner sep=0pt}] \matrix [matrix of nodes,draw=white,nodes in empty cells] {
\A&\A&\A&\A\\
\B&\A&\A&\B\\
\B&\A&\A&\A\\
\A&\B&\A&\A\\
\A&\A&\A&\A\\
};
\end{tikzpicture}
\hspace{0.25ex}
\begin{tikzpicture}[every node/.style={anchor=base,text depth=.5ex,text height=2ex,text width=1em,outer sep=0pt,align=center,inner sep=0pt}] \matrix [matrix of nodes,draw=white,nodes in empty cells] {
\B&\A&\A&\A\\
\B&\A&\A&\B\\
\A&\A&\A&\A\\
\A&\B&\A&\A\\
\A&\A&\A&\A\\
};
\end{tikzpicture}
\hspace{0.25ex}
\begin{tikzpicture}[every node/.style={anchor=base,text depth=.5ex,text height=2ex,text width=1em,outer sep=0pt,align=center,inner sep=0pt}] \matrix [matrix of nodes,draw=white,nodes in empty cells] {
\B&\A&\A&\A\\
\A&\A&\B&\B\\
\A&\A&\A&\A\\
\A&\B&\A&\A\\
\A&\A&\A&\A\\
};
\end{tikzpicture}
\hspace{0.25ex}
\begin{tikzpicture}[every node/.style={anchor=base,text depth=.5ex,text height=2ex,text width=1em,outer sep=0pt,align=center,inner sep=0pt}] \matrix [matrix of nodes,draw=white,nodes in empty cells] {
\B&\A&\A&\A\\
\A&\A&\B&\B\\
\A&\A&\A&\A\\
\A&\A&\A&\A\\
\A&\B&\A&\A\\
};
\end{tikzpicture}
\hspace{0.25ex}
\begin{tikzpicture}[every node/.style={anchor=base,text depth=.5ex,text height=2ex,text width=1em,outer sep=0pt,align=center,inner sep=0pt}] \matrix [matrix of nodes,draw=white,nodes in empty cells] {
\B&\B&\A&\A\\
\A&\A&\B&\B\\
\A&\A&\A&\A\\
\A&\A&\A&\A\\
\A&\A&\A&\A\\
};
\end{tikzpicture}
\hspace{0.25ex}
\begin{tikzpicture}[every node/.style={anchor=base,text depth=.5ex,text height=2ex,text width=1em,outer sep=0pt,align=center,inner sep=0pt}] \matrix [matrix of nodes,draw=white,nodes in empty cells] {
\B&\B&\B&\A\\
\A&\A&\A&\B\\
\A&\A&\A&\A\\
\A&\A&\A&\A\\
\A&\A&\A&\A\\
};
\end{tikzpicture}
\hspace{0.25ex}
\begin{tikzpicture}[every node/.style={anchor=base,text depth=.5ex,text height=2ex,text width=1em,outer sep=0pt,align=center,inner sep=0pt}] \matrix [matrix of nodes,draw=white,nodes in empty cells] {
\B&\B&\B&\B\\
\A&\A&\A&\A\\
\A&\A&\A&\A\\
\A&\A&\A&\A\\
\A&\A&\A&\A\\
};
\end{tikzpicture}


\begin{exercise}
Write a program that reads in a roller-board file (specified on the command line), and shows the `moves' to solve it. Such a file looks something like~:
\begin{terminaloutput}
5 4
0000
1100
1000
0100
0000
\end{terminaloutput}

\noindent The first line has two numbers; the height of
the board (number of rows) and then the width (number of columns).

\noindent
In the remainder of the file, the number of $1$s must be equal to the width of the board, and only the characters `0' and `1' are valid.You may assume that the maximum height and width of a board is $6$.

\noindent
The brute-force algorithm for searching over all moves for a
solution goes like this~:
\begin{enumerate}
\item You will use an \verb^alloc()^'d array (list) of boards.
\item Put the initial board into the front of this list, \verb^f=0^.
\item Consider the board at the {\bf front} of the list (index \verb$f$).
\item For this (parent) board, find the resulting (child) boards 
which can be created from all the possible $(rows + columns) \times 2$ rolls. For each of these child boards:
\begin{itemize}
\item If this board is unique (i.e.\ it has not been seen before in the list), add it to the end of the list.
\item If it has been seen before (a duplicate) ignore it.
\item If it is the `final' board, stop and print the solution.
\end{itemize}
\item Add one of $f$. If there are more boards in the list, go to step $3$.
\end{enumerate}

\noindent To help with printing out the correct moves, when a solution
has been found, each structure in the list will need to contain (amongst
other things) a board and a record of its parent board, i.e. the board
that it was created from. Since you're using an array, this could simply
be which element of the array was the parent.

\noindent The program reads the name of the board definition file from
the command line.  If it finds a successful solution, it prints out the
number of boards that would be printed in the solution and {\bf nothing
else}, or else exactly the phrase `No Solution?'' if none can be found
(as might be the case if you simply run out of memory)~:

\begin{terminaloutput}
$ ./rollerboard 4x4diag.rbd
4 moves
$ ./rollerboard 5x5lhs.rbd
8 moves
\end{terminaloutput}


If the `verbose' flag is used, your program will print out the solution in the correct order~:
\begin{terminaloutput}
$ ./rollerboard -v 4x4lr.rbd
0:
0000
1001
1001
0000

1:
1000
1001
0001
0000

2:
0100
1001
0001
0000

3:
1100
0001
0001
0000

4:
1101
0001
0000
0000

5:
1110
0001
0000
0000

6:
1111
0000
0000
0000 

$ ./rollerboard -v 3x3crn.rbd
0:
000
001
011

1:
000
100
011

2:
100
000
011

3:
110
000
001

4:
111
000
000

\end{terminaloutput}

\noindent
Your program~:
\begin{itemize}
\item {\bf Must} use the algorithm detailed above (which is similar to a queue and therefore a breadth-first search). Do not use the other algorithms possible (e.g. best-first, guided, recursive etc.); the quality of your coding is being assessed against others taking the same approach.
\item {\bf Must not} use dynamic arrays or linked lists. Since boards cannot be any larger than $6 \times 6$, you can create boards of this size, and only use a sub-part of them if the board required is smaller. The list of boards can be a fixed (large) size.
\item {\bf Should} check for invalid board definition files, and report in a graceful way if there is a problem, aborting with \verb^exit(EXIT_FAILURE)^ if so.
\item {\bf Should not} print anything else out to screen after successfully
completing the search, except that which is shown above. Automated checking
will be used during marking, and therefore the output must be precise.
\item {\bf Should} call the function \verb^test()^ to perform any assertion testing etc.
\end{itemize}


\subsection*{Extension}

Basic assignment = {\Large $90\%$}.
Extension = {\Large $10\%$}.

\noindent
If you'd like to try an extension, make sure to submit {\it extension.c}
and a brief description in a {\it extension.txt} file. This could
involve a faster search technique, better graphical display, user input
or something else of your choosing.

\end{exercise}

\nsection{Car Park}

In a 2-dimesional car park as seen from above, vehicles are allowed to park either vertically or horizontally on a grid~:
\renewcommand{\A}{|[fill=gray,text=red]|A}
\renewcommand{\B}{|[fill=gray,text=green]|B}
\renewcommand{\X}{|[fill=ocre,text=ocre]|X}
\renewcommand{\H}{|[fill=gray,text=gray]|.}

\begin{tikzpicture}[every node/.style={anchor=base,text depth=.5ex,text height=2ex,text width=1em,outer sep=0pt,align=center,inner sep=0pt}]
\matrix [matrix of nodes,draw=white,nodes in empty cells]
{
\X&\H&\X&\X&\X&\X\\
\H&\B&\B&\B&\H&\X\\
\X&\A&\H&\H&\H&\X\\
\X&\A&\H&\H&\H&\X\\
\X&\A&\H&\H&\H&\X\\
\X&\X&\X&\X&\X&\X\\
};
\end{tikzpicture}

Here there are two vehicles, both of length $3$. One is vertical (aligned
north/south) coloured red, and one is horizontal (aligned west/east)
coloured green. Empty cells of the gridded car park are shown in gray,
and immovable boundaries (bollards) are shown in orange. Vehicles can
only move forwards or backwards one square at a time (vertical vehicles
move up and down, horizontal vehicles move left and right). These moves
cannot be into, or across, another vehicle, nor into a bollard.

Our aim is to get all the vehicles safely out of the car park.
During a turn, one vehicle is allowed to move forwards or backwards one
square.  Once the front of a vehicle touches an empty square on the edge
of the car park, it is deemed to have exited the car park, and removed.

To `solve' the car park above, we would move vehicle B one move left (so that it exits)~:

\begin{tikzpicture}[every node/.style={anchor=base,text depth=.5ex,text height=2ex,text width=1em,outer sep=0pt,align=center,inner sep=0pt}]
\matrix [matrix of nodes,draw=white,nodes in empty cells]
{
\X&\H&\X&\X&\X&\X\\
\H&\H&\H&\H&\H&\X\\
\X&\A&\H&\H&\H&\X\\
\X&\A&\H&\H&\H&\X\\
\X&\A&\H&\H&\H&\X\\
\X&\X&\X&\X&\X&\X\\
};
\end{tikzpicture}

and then vehicle A up one~:

\begin{tikzpicture}[every node/.style={anchor=base,text depth=.5ex,text height=2ex,text width=1em,outer sep=0pt,align=center,inner sep=0pt}]
\matrix [matrix of nodes,draw=white,nodes in empty cells]
{
\X&\H&\X&\X&\X&\X\\
\H&\A&\H&\H&\H&\X\\
\X&\A&\H&\H&\H&\X\\
\X&\A&\H&\H&\H&\X\\
\X&\H&\H&\H&\H&\X\\
\X&\X&\X&\X&\X&\X\\
};
\end{tikzpicture}

and then up one more move so that it too exits~:

\begin{tikzpicture}[every node/.style={anchor=base,text depth=.5ex,text height=2ex,text width=1em,outer sep=0pt,align=center,inner sep=0pt}]
\matrix [matrix of nodes,draw=white,nodes in empty cells]
{
\X&\H&\X&\X&\X&\X\\
\H&\H&\H&\H&\H&\X\\
\X&\H&\H&\H&\H&\X\\
\X&\H&\H&\H&\H&\X\\
\X&\H&\H&\H&\H&\X\\
\X&\X&\X&\X&\X&\X\\
};
\end{tikzpicture}

All vehicles have now exited the car park taking a total of three turns.


\begin{exercise}
Write a program that reads in a car park file (specified on the command line), and shows the `turns' to solve it. The file for the car park above looks like~:
\begin{terminaloutput}
6x6
#.####
.BBB.#
#A...#
#A...#
#A...#
######
\end{terminaloutput}

\noindent The first line has two numbers; the height of
the car park (number of rows) and then the width (number of columns).

\noindent
In the remainder of the file, vehicles are shown as a capital letter,
gaps as a full-stop and bollards as a hash symbol. Each cars may only
lie in the grid vertically or horizontally, and must be of at least 
length $2$. Each vehicle must have a unique uppercase letter, the first of which must be an `A', the next one be `B' and so on.

\noindent
We wil use a brute-force algorithm for searching over all moves for a
solution~:
\begin{enumerate}
\item You will use an array (list) of structures, each one containing the data for one car park.
Note that you may choose to store the state of each car park, not as
a 2D array, but as something that is easier to manipulate, e.g. an array
of vehicles, including their position, orientation and whether they've exited or not.
Each approach has pros and cons.
\item Put the initial car park into the front of this list, \verb^f=0^.
\item Consider the car park at the {\bf front} of the list (index \verb$f$).
\item For this (parent) car park, find the resulting (child) car parks 
which can be created from all the possible vehicle moves. For each of these child car parks:
\begin{itemize}
\item If this car park is unique (i.e.\ it has not been seen before in the list), add it to the end of the list.
\item If it has been seen before (a duplicate) ignore it.
\item If it is the `final' car park, stop and print the solution.
\end{itemize}
\item Add one to $f$. If there are more car parks in the list, go to step $3$.
\end{enumerate}

\noindent To help with printing out the correct moves, when a solution
has been found, each structure in the list will need to contain (amongst
other things) a car park and a record of its parent car park, i.e. the car park
that it was created from. Since you're using an array, this could simply
be which index of the array was the parent.

\noindent The program reads the name of the car park definition file from
the command line.  If it finds a successful solution, it prints out the
number of car parks that would be printed in the solution and {\bf nothing
else}, or else exactly the phrase `No Solution?'' if none can be found
(as might be the case if it is impossible, or you simply run out of memory)~:

\begin{terminaloutput}
$ ./carpark ../Git/Data/CarPark/6x6_2c_3t.prk
3 moves
$ .car/park ../Git/Data/CarPark/11x9_10c_26t.prk
26 moves
\end{terminaloutput}

If the `show' flag is used, your program will print out the solution in the correct order~:
\begin{terminaloutput}
$ ./carpark -show ../Git/Data/CarPark/6x6_2c_3t.prk
#.####
.BBB.#
#A...#
#A...#
#A...#
######

#.####
.....#
#A...#
#A...#
#A...#
######

#.####
.A...#
#A...#
#A...#
#....#
######

#.####
.....#
#....#
#....#
#....#
######

3 moves
\end{terminaloutput}

\noindent
Your program~:
\begin{itemize}
\item {\bf Must} use the algorithm detailed above (which is similar to a queue and therefore a breadth-first search). Do not use the other algorithms possible (e.g. best-first, guided, recursive etc.); the quality of your coding is being assessed against others taking the same approach.
\item {\bf Must not} use dynamic arrays or linked lists. Since car parks cannot be any larger than $20 \times 20$, you can create car parks of this size, and only use a sub-part of them if the car park required is smaller. The list of car parks can be a fixed (large) size.
\item {\bf Should} check for invalid car park definition files, and report in a graceful way if there is a problem, aborting with \verb^exit(EXIT_FAILURE)^ if so.
\item {\bf Should not} print anything else out to screen after successfully
completing the search, except that which is shown above. Automated checking
will be used during marking, and therefore the output must be precise.
\item {\bf Should} call the function \verb^test()^ to perform any assertion testing etc.
\end{itemize}

\subsection*{Extension}

Basic assignment = {\Large $90\%$}.
Extension = {\Large $10\%$}.

\noindent
If you'd like to try an extension, make sure to submit {\it extension.c}
and a brief description in a {\it extension.txt} file, and an {\it extension.mak} Makefile, allowing me to build your code using \verb^make extension^.
\noindent The extension could
involve a faster search technique, better graphical display, user input
or something else of your choosing.
\end{exercise}

\input{8queens}
\renewcommand{\A}{|[fill=black,text=white]|A}
\renewcommand{\B}{|[fill=black,text=ocre]|B}
\renewcommand{\C}{|[fill=black,text=green]|C}

\nsection{Match Drop}

The puzzle {\it Match Drop} consists of a $2D$ rectangular grid of
tiles, all of which are labelled with an upper-case letter e.g. `$A$', `$B \ldots$ `$Z$'. Outside of this grid is another tile, known as the
`hawk' tile:

\begin{tikzpicture}[every node/.style={anchor=base,text depth=.5ex,text height=2ex,text width=1em,outer sep=0pt,align=center,inner sep=0pt}] \matrix [matrix of nodes,draw=white,nodes in empty cells] {
\A&  &  \\
\A&\B&\C\\
\A&\B&\C\\
\C&\B&\A\\
};
\end{tikzpicture}

\noindent The 'hawk' tile can be used to push down one column.
The hawk becomes the top tile in this column, and the bottom tile of
this column becomes the new hawk tile. The task is to roll one column
at a time, so that every column contains the same letters.

\noindent
In the above example, if the hawk tile is played to push down the first
column, then the new board now looks like:

\begin{tikzpicture}[every node/.style={anchor=base,text depth=.5ex,text height=2ex,text width=1em,outer sep=0pt,align=center,inner sep=0pt}] \matrix [matrix of nodes,draw=white,nodes in empty cells] {
\C&  &  \\
\A&\B&\C\\
\A&\B&\C\\
\A&\B&\A\\
};
\end{tikzpicture}

\noindent
Both the first and seond columns are now completed and are never
altered again. Using the hawk on column three produces:

\begin{tikzpicture}[every node/.style={anchor=base,text depth=.5ex,text height=2ex,text width=1em,outer sep=0pt,align=center,inner sep=0pt}] \matrix [matrix of nodes,draw=white,nodes in empty cells] {
\A&  &  \\
\A&\B&\C\\
\A&\B&\C\\
\A&\B&\C\\
};
\end{tikzpicture}

\noindent
and our search for a finshed (completed) board is over.


\begin{exercise}
Write the functions specified in \verb^md.h^ that allows a board file
to be read in, and computes the number of moves required to solve it.

\noindent
You may assume that the maximum height and width of a board is $6$.

\noindent
The brute-force algorithm for searching over all moves for a
solution goes like this~:
\begin{enumerate}
\item You will use an \verb^alloc()^'d array (list) of boards.
\item Put the initial board into the front of this list, \verb^f=0^.
\item Consider the board at the {\bf front} of the list (index \verb$f$).
\item For this (parent) board, find the resulting (child) boards 
which can be created from all the possible column pushes (already
completed columns are not altered). For each of these child boards:
\begin{itemize}
\item If this board is unique (i.e.\ it has not been seen before in the list), add it to the end of the list.
\item If it has been seen before (it's a duplicate) ignore it.
\item If it is the `final' board, stop and (possibly, print the solution).
\end{itemize}
\item Add one to $f$. If there are more boards in the list, go to step $3$.
\end{enumerate}








\noindent To help with printing out the correct moves, when a solution
has been found, each board in the list will need to contain (amongst
other things) a 2D grid of tiles, the hawk, and a record of its parent board, i.e. the board
that it was created from. Since you're using an array, this could simply
be the index of the array that was the parent.

\noindent
Your program~:
\begin{itemize}
\item {\bf Must} use the algorithm detailed above (which is similar to a queue and therefore a breadth-first search). Do not use the other algorithms possible (e.g. best-first, guided, recursive etc.); the quality of your coding is being assessed against others taking the same approach, and if you do something different it won't get any marks.
\item {\bf Must not} use dynamic arrays or linked lists. Since boards cannot be any larger than $6 \times 6$, you can create boards of this size, and only use a sub-part of them if the board required is smaller. The list of boards can be a fixed (large) size (maybe $200,000$?)
\item {\bf Should} be able to cope with invalid board definition files with a graceful exit.
\item {\bf Should not} print anything out to screen after successfully
completing the search, except when in verbose mode. Automated checking
will be used during marking, and therefore the output must be very precise.
For the \verb^driver.c^ file given, the verbose output is required for
\verb^2moves.brd^, for which the verbose flag has been set in the 
solve function. In this case, the output will look like:
\begin{codesnippet}
% ./md
ABC
ABC
ABC
CBA

ABC
ABC
ABC
ABA

ABC
ABC
ABC
ABC
\end{codesnippet}

\item {\bf Should} call the function \verb^test()^ to perform any assertion testing etc.
\end{itemize}

\end{exercise}



%%%%%%%%%%%%%%%%%%%%%%%%%%%% WEEK 9 %%%%%%%%%%%%%%%%%%%%%%%%%%%%%%%%%%%%%
\chapterimage{../Pictures/btree.png}
\chapter{ADTs \& Data Structures I}

\input{indexarr}
\input{boolarr}
\input{sets}
\input{polymorphism}
\nsection{Self-Organising Linked Lists}

A self-organising linked list (SOLL) improves search efficiency (over
an unsorted list) by rearranging the elements in the list each time they
are accessed: 

\wwwurl{https://en.wikipedia.org/wiki/Self-organizing_list}

If no self-organisation is done, a SOLL behaves similarly to a collection.
New elements are inserted at the end of the list, and searching is done
from the start to the end using pointer-chasing.  For efficiencies sake,
we keep pointers to both the start and the end of the list. Insertion,
therefore, has a $O(1)$ cost.

However, if self-organisation is done via the {\it move-to-front} (MFT)
policy, each time an element is accessed, it is moved to the start of the
list. This means that the next time we search for the same element, it
will be found more quickly, since it is near the start of the list. Once
the element is found, this re-ordering also has a complexity of $O(1)$.

The MFT policy can sometimes be too aggressive; uncommon elements that
are searched for will move to the front, potentially displacing elements that
are being searched for more frequently. An alternative is the
{\it transpose} policy. When an element is accessed, it is moved one
place in the list closer to the start. This once again can be done in
constant time, providing we keep a copy of a pointer to the previous
element during the search, or alternatively, use a doubly-linked list
e.g. each element has both a next and a previous pointer.


\begin{exercise}

Implement a SOLL using linked lists to implement these different policies
and store strings.

\begin{itemize}
\item {\bf $55\%$}
Write the files \verb^Linked/specific.h^ and \verb^Linked/linked.c^.
Using the \verb^soll.mak^ makefile, you can compile this against the
\verb^soll.h^ file and one of the test/driver files \verb^testsoll.c^,
\verb^build.c^ or \verb^uniq.c^.

The ADT should implement the standard functions: \verb^soll_init()^,
\verb^soll_add()^, \verb^soll_remove()^, \verb^soll_isin()^,
\verb^soll_tostring()^, \verb^soll_size()^ and \verb^soll_free()^.
In addition, write the function \verb^soll_freq()^ which reports
the frequency of access for a particular element.
Therefore, each element (structure) of the list has the
overhead of keeping track of the number of times it has been accessed.

\item {\bf $30\%$}
Show a testing strategy on the above by submitting \verb^testing.txt^
where you give details of
unit testing, white/black-box testing done on your code. Describe any
test-harnesses used. Convince me that every line of your C code
has been tested.

\item {\bf $15\%$}
Show an extension to the project in a direction of
your choice via \verb^extension.txt^.
It should demonstrate your {\bf understanding} of some aspect
of programming or S/W engineering. Make sure it's clear what has been
done, why, and how to compile it.
\end{itemize}

\end{exercise}

\input{sudoku}
\nsection{MultiValue Maps}

Many data types concern a single value (e.g. a hash table), so that
a string (say) acts as both the key (by which we search for the data)
and also as the object we need to store (the value). An example of this a spelling checker,
where one word is stored (and searched for) at a time.  However, sometimes
there is a need to store a value based on a particular key - for instance
an associative array in Python allows you to perform operations such as :
\begin{codesnippet}
population["Bristol"] = 536000
\end{codesnippet}
where a value (the number 536000) is stored using the key (the string "Bristol").
One decision you need to make when designing such a data type is whether
multiple values are allowed for the same key; in the above example this
would make no sense - Bristol can only have one population size. But if
you wanted to store people as the key, with their salary as the value,
you might need to use a MultiValue Map (MVM) since people can have more than
one job.

Here we write the abstract type for a MultiValueMap that stores key-value pairs,
where both the key and the value are strings.

\begin{exercise}
\label{ex:mvm}
The definition of an MVM ADT is given in \verb^mvm.h^, and a file to test it is given
in \verb^testmvm.c^.  Write \verb^mvm.c^, so that:
{\small
\begin{terminaloutput}
% make -f mvm_adt.mk
./testmvm
Basic MVM Tests ... Start
Basic MVM Tests ... Stop
\end{terminaloutput}
}
\noindent works correctly. Use a simple linked list for this, inserting
new items at the head of the list.
Make no changes to any of my files.
\end{exercise}

\input{rhymes}
\input{advent}


%%%%%%%%%%%%%%%%%%%%%%%%%%%% WEEK 10 %%%%%%%%%%%%%%%%%%%%%%%%%%%%%%%%%%%%
\chapterimage{../Pictures/hash.jpg}
\chapter{Trees \& Hashing}

\input{depthtree}
\input{twotrees}
\nsection{Huffman Encoding}
Huffman encoding is commonly used for data compression.
Based on the frequency of occurrence of characters, you build
a tree where rare characters appear at the bottom of the tree, and
commonly occurring characters are near the top of the tree.

For an example input text file, a Huffman tree might look something like:

{\small
\begin{verbatim}
010 :        00101 (   5 *  125)
` ' :          110 (   3 *  792)
`"' :    111001010 (   9 *   12)
`'' :     00100000 (   8 *   15)
`(' :  01100000100 (  11 *    2)
`)' :  01100001101 (  11 *    2)
`,' :      1001001 (   7 *   39)
`-' :      0010010 (   7 *   31)
`.' :      1001100 (   7 *   40)
`/' :  00100110000 (  11 *    1)
`0' :  11100110010 (  11 *    3)
`1' :     00100010 (   8 *   15)
`3' :  01100000101 (  11 *    2)
`4' :  01100001001 (  11 *    2)
`5' :  11100110011 (  11 *    3)
`6' :  01100001000 (  11 *    2)
`7' :  01100001100 (  11 *    2)
`8' :    001001101 (   9 *    8)
`9' :     10010000 (   8 *   18)
`:' :  01100001011 (  11 *    2)
`A' :     00100111 (   8 *   16)
`B' :    111001101 (   9 *   13)
`C' :     10011011 (   8 *   22)
`D' :    111001110 (   9 *   13)
`E' :     10011010 (   8 *   19)
`F' :    111001000 (   9 *   11)
`G' :   0110000000 (  10 *    4)
`H' :   1110011111 (  10 *    7)
`I' :   1110010011 (  10 *    6)
`J' :  11100111101 (  11 *    3)
`K' :   1110010111 (  10 *    6)
`L' :     00100011 (   8 *   15)
`M' :  11100111100 (  11 *    3)
`N' :  01100001010 (  11 *    2)
`O' :  01100000111 (  11 *    2)
`P' :   1110011000 (  10 *    6)
`R' :   0110000111 (  10 *    5)
`S' :     10010001 (   8 *   19)
`T' :   0010011001 (  10 *    4)
`U' :   1110010010 (  10 *    5)
`W' :   0110000001 (  10 *    4)
`a' :         1010 (   4 *  339)
`b' :      1111110 (   7 *   60)
`c' :       100101 (   6 *   77)
`d' :        01101 (   5 *  143)
`e' :          000 (   3 *  473)
`f' :       100111 (   6 *   84)
`g' :       111000 (   6 *   94)
`h' :        11110 (   5 *  223)
`i' :         0100 (   4 *  266)
`j' :  01100000110 (  11 *    2)
`k' :     00100001 (   8 *   15)
`l' :        10110 (   5 *  176)
`m' :       101111 (   6 *   92)
`n' :         0111 (   4 *  288)
`o' :         0101 (   4 *  269)
`p' :       101110 (   6 *   89)
`q' :  00100110001 (  11 *    2)
`r' :        11101 (   5 *  214)
`s' :         0011 (   4 *  260)
`t' :         1000 (   4 *  305)
`u' :       111110 (   6 *  108)
`v' :      0110001 (   7 *   37)
`w' :      1111111 (   7 *   60)
`x' :   1110010110 (  10 *    6)
`y' :       011001 (   6 *   72)
2916 bytes
\end{verbatim}
}

Each character is shown, along with its Huffman bit-pattern, the length
of the bit-pattern and the frequency of occurrence. At the bottom, the total
number of bytes required to compress the file is displayed.

\begin{exercise}
Write a program that reads in a file (argv[1]) and, based on the characters
it contains, computes the Huffman tree, displaying it as above. 
\end{exercise}

\input{bintreevis}
\input{lowestca}
\input{fastermvm}
\input{doublehash}
\input{sepchain}

%%%%%%%%%%%%%%%%%%%%%%%%%%%% WEEK 11 %%%%%%%%%%%%%%%%%%%%%%%%%%%%%%%%%%%%
\chapterimage{../Pictures/nursery-rhymes.jpg}
\chapter{ADTs II \& Data Structures }

\input{hashpoly}
\input{cuckoo}
\input{dicts}
\nsection{Cons, Car and Cdr}
\label{ex:carcdr}

When storing real-world data, lists containing other lists are ubiquitous e.g.~:
\verb^(0 (1 2) 3 4 5)^.

\noindent In fact many functional (or at least partly functional) languages are
based around the idea that all data are, conceptually, nested lists. Examples
of such languages include Haskell and, of interest here, Lisp~:
\wwwurl{https://en.wikipedia.org/wiki/Lisp\_(programming_language)}

One of the main operations of these languages is the ability to extract
the front (head) or remainder (tail) of the list {\bf very} efficiently.   
A traditional linked list is not good for this, since it would require work
to untangle the head (it has a pointer to the rest of the list) and
it's unclear how a list-within-a-list would be stored.

For Lisp, a better structure was developed, which is slightly more complex than
a traditional linked list, but which makes the extraction of the head (and remainder)
simple and fast. To encode the list \verb^(1 2)^, which is a list containing two atoms,
we use~:
\begin{terminaloutput}
[*|*]--->[*|*]--->NIL
 |        |
 v        v
 1        2
\end{terminaloutput}

\noindent Here the main struct (known as the {\it cons})
has two pointers.  The ones going
downwards known as the {\it car} pointers and the ones going horizontally
the {\it cdr}\footnote{ From wiki: Two assembly language macros $...$
became the primitive operations for decomposing lists: {\it car} (Contents
of the Address part of Register number) and {\it cdr} (Contents of the
Decrement part of Register number).  } pointers.
If we were to have a pointer $p$ to the first {\it cons} of this list, the head of the
list is pointed to by \verb^p->car^, and the remainder of the list simply by \verb^p->cdr^.

\newpage 

A more complex list, \verb^(0 (1 2) 3 4 5)^ would be stored as~:
\begin{terminaloutput}
[*|*]--->[*|*]------------------>[*|*]--->[*|*]--->[*|*]--->NIL
 |        |                       |        |        |
 v        v                       v        v        v
 0       [*|*]--->[*|*]--->NIL    3        4        5
          |        |
          v        v
          1        2
\end{terminaloutput}

\noindent The atomic elements (the integers) are stored as `leaf' nodes with both {\it car} and {\it cdr}
pointers set to NULL. The data structure used to store the atom and also {\it car} and {\it cdr} pointers
is known as a {\it cons} (short for constructor), due the to process by which we build lists.

So, here, we're interested in recreating this data structure, allowing the user to build lists (the
{\it cons} operation), extract the head and remainder of lists (the {\it car} and {\it cdr} operations)
and other associated functions such as copying a list or counting the number of elements in it.

\begin{exercise}
Use a dynamic/linkedlist style approach to implement a car/cdr ADT. Write the source files
\verb^Linked/specific.h^ and \verb^Linked/linked.c^, so that they compile against my
\verb^lisp.h^ and \verb^testlisp.c^ files and run successfully using my \verb^Makefile^.
The basic operations are~:
\begin{verbatim}
lisp* lisp_atom(const atomtype a);
lisp* lisp_cons(const lisp* l1,  const lisp* l2);
lisp* lisp_car(const lisp* l);
lisp* lisp_cdr(const lisp* l);
atomtype lisp_getval(const lisp* l);
bool lisp_isatomic(const lisp* l);
lisp* lisp_copy(const lisp* l);
int lisp_length(const lisp* l);
void lisp_tostring(const lisp* l, char* str);
void lisp_free(lisp** l);
\end{verbatim}

\noindent This is worth $90\%$ of the marks.
Additional functions you can implement, worth $10\%$, are~:
\begin{verbatim}
lisp* lisp_fromstring(const char* str);
lisp* lisp_list(const int n, ...);
void lisp_reduce(void (*func)(lisp* l, atomtype* n), lisp* l, atomtype* acc);
\end{verbatim}
and are {\bf Extensions} worth $10\%$ of the marks.

\noindent Even if you don't get these extensions (or other functions) to work correctly, make
sure the code still compiles by writing `dummy' functions as placeholders,
even if some of the assertions fail.
\end{exercise}

\input{bsa}

%%%%%%%%%%%%%%%%%%%%%%%%%%%% WEEK 12 %%%%%%%%%%%%%%%%%%%%%%%%%%%%%%%%%%%%
\chapterimage{../Pictures/ttl2.jpg}
\chapter{Parsing Data}

\newcommand{\bb}{white}
\newcommand{\ff}{black}

\newcommand{\sixel}[6]{%
\begin{tikzpicture}[scale=0.333, every node/.style={scale=0.333}]
\matrix[sixelstyle]
{
|[fill=#1]| \& |[fill=#2]| \\
|[fill=#3]| \& |[fill=#4]| \\
|[fill=#5]| \& |[fill=#6]| \\
};
\end{tikzpicture}%
}

\newcommand{\sepsix}[6]{%
\begin{tikzpicture}[scale=0.333, every node/.style={scale=0.333}]
\matrix[sepsixstyle]
{
|[fill=#1]| \& |[fill=#2]| \\
|[fill=#3]| \& |[fill=#4]| \\
|[fill=#5]| \& |[fill=#6]| \\
};
\end{tikzpicture}%
}


\input{tikzsets}

\nsection{Teletext}

In the early 1970s, Phillips Labs began work on transmitting digital
information across the television network. The aim was to
provide up-to-date news and weather information via a television set. 
This system was trialled first by the BBC in a system that eventually became
known as ``Ceefax'' and then on other independent British terrestrial stations as ``Oracle''.
A very similar system was implemented on the BBC microcomputer, known as {\it Mode 7}.
\begin{figure}[ht]
\centering{
\includegraphics{../Pictures/teletext100.pdf}
}
\caption{An example Ceefax page circa 1983. Taken from {\tt http://teletext.mb21.co.uk/gallery/ceefax/main1.shtml}}
\label{fig:tt100}
\end{figure}
An example of such a Ceefax screen is shown in Figure~\ref{fig:tt100}.

This project, inspired by such teletext systems, will allow a $40 \times 25$ ($1000$) character
file to be rendered to the screen, using similar control codes. However, some control
codes are not implemented, including those to do with flashing or hidden text, and transparent backgrounds. In particular, our definition of the {\it double height} control code differs from that of
the traditional one.

\subsection*{The Control Codes}

This section is based to a large extent to Richard Russell's description of
Mode 7 on the BBC Micro:
\verb^http://www.bbcbasic.co.uk/tccgen/manual/tcgen2.html^.

\begin{table}
\begin{tabular}{|c|c|c|c|c|c|c|c|c|}\hline
   &0x8            & 0x9               & 0xA&0xB&0xC& 0xD         &0xE& 0xF         \\
 0 &Unused/Reserved&Unused/Reserved    &    & 0 & @ & P           & - & p           \\
 1 &Red Alphanumeric       &Red Graphics       & !  & 1 & A & Q           & a & q           \\
 2 &Green Alphanumeric     &Green Graphics     & "  & 2 & B & R           & b & r           \\
 3 &Yellow Alphanumeric    &Yellow Graphics    & $\pounds$  & 3 & C & S           & c & s           \\
 4 &Blue   Alphanumeric    &Blue   Graphics    & \$ & 4 & D & T           & d & t           \\
 5 &Magenta Alphanumeric   &Magenta Graphics   & \% & 5 & E & U           & e & u           \\
 6 &Cyan Alphanumeric      &Cyan Graphics      & \& & 6 & F & V           & f & v           \\
 7 &White Alphanumeric     &White Graphics     &  ' & 7 & G & W           & g & w           \\
 8 &Unused/Reserved&Unused/Reserved    &  ( & 8 & H & X           & h & x           \\
 9 &Unused/Reserved&Contiguous Graphics&  ) & 9 & I & Y           & i & y           \\
 A &Unused/Reserved&Separated Graphics &  * & : & J & Z           & j & z           \\
 B &Unused/Reserved&Unused/Reserved    &  + & ; & K & $\leftarrow$& k &$\sfrac{1}{4}$\\
 C &Single Height  &Black Background   &  , & < & L &$\sfrac{1}{2}$& l & $||$           \\
 D &Double Height  &New Background     &  - & = & M &$\rightarrow$ & m &$\sfrac{3}{4}$\\
 E &Unused/Reserved&Hold Graphics      &  . & > & N & $\uparrow$  & n &$\div$    \\
 F &Unused/Reserved&Release Graphics   &  / & ? & O & \#           & o & \textblock           \\ \hline
\end{tabular}
\caption{The control codes and characters for alphanumeric mode. Note here (because we're using white paper) foreground is shown in black and background in white. On a teletext screen we use white on a black background.}
\label{tab:normgraph}
\end{table}

\begin{table}
\begin{tabular}{|c|c|c|c|c|c|c|c|c|}\hline
   &0x8            & 0x9               		& 0xA&0xB&0xC& 0xD         &0xE& 0xF         \\
 0 &Unused/Reserved&Unused/Reserved    		& \sixel{\bb}{\bb}{\bb}{\bb}{\bb}{\bb} & \sixel{\bb}{\bb}{\bb}{\bb}{\ff}{\bb} & @ & P           & \sixel{\bb}{\bb}{\bb}{\bb}{\bb}{\ff} & \sixel{\bb}{\bb}{\bb}{\bb}{\ff}{\ff}\\
 1 &Red Alphanumeric       &Red Graphics	& \sixel{\ff}{\bb}{\bb}{\bb}{\bb}{\bb} & \sixel{\ff}{\bb}{\bb}{\bb}{\ff}{\bb} & A & Q           & \sixel{\ff}{\bb}{\bb}{\bb}{\bb}{\ff} & \sixel{\ff}{\bb}{\bb}{\bb}{\ff}{\ff}\\
 2 &Green Alphanumeric     &Green Graphics	& \sixel{\bb}{\ff}{\bb}{\bb}{\bb}{\bb} & \sixel{\bb}{\ff}{\bb}{\bb}{\ff}{\bb} & B & R           & \sixel{\bb}{\ff}{\bb}{\bb}{\bb}{\ff} & \sixel{\bb}{\ff}{\bb}{\bb}{\ff}{\ff}\\
 3 &Yellow Alphanumeric    &Yellow Graphics	& \sixel{\ff}{\ff}{\bb}{\bb}{\bb}{\bb} & \sixel{\ff}{\ff}{\bb}{\bb}{\ff}{\bb} & C & S           & \sixel{\ff}{\ff}{\bb}{\bb}{\bb}{\ff} & \sixel{\ff}{\ff}{\bb}{\bb}{\ff}{\ff}\\
 4 &Blue   Alphanumeric    &Blue   Graphics	& \sixel{\bb}{\bb}{\ff}{\bb}{\bb}{\bb} & \sixel{\bb}{\bb}{\ff}{\bb}{\ff}{\bb} & D & T           & \sixel{\bb}{\bb}{\ff}{\bb}{\bb}{\ff} & \sixel{\bb}{\bb}{\ff}{\bb}{\ff}{\ff}\\
 5 &Magenta Alphanumeric   &Magenta Graphics	& \sixel{\ff}{\bb}{\ff}{\bb}{\bb}{\bb} & \sixel{\ff}{\bb}{\ff}{\bb}{\ff}{\bb} & E & U           & \sixel{\ff}{\bb}{\ff}{\bb}{\bb}{\ff} & \sixel{\ff}{\bb}{\ff}{\bb}{\ff}{\ff}\\
 6 &Cyan Alphanumeric      &Cyan Graphics	& \sixel{\bb}{\ff}{\ff}{\bb}{\bb}{\bb} & \sixel{\bb}{\ff}{\ff}{\bb}{\ff}{\bb} & F & V           & \sixel{\bb}{\ff}{\ff}{\bb}{\bb}{\ff} & \sixel{\bb}{\ff}{\ff}{\bb}{\ff}{\ff}\\
 7 &White Alphanumeric     &White Graphics	& \sixel{\ff}{\ff}{\ff}{\bb}{\bb}{\bb} & \sixel{\ff}{\ff}{\ff}{\bb}{\ff}{\bb} & G & W           & \sixel{\ff}{\ff}{\ff}{\bb}{\bb}{\ff} & \sixel{\ff}{\ff}{\ff}{\bb}{\ff}{\ff}\\
 8 &Unused/Reserved&Unused/Reserved		& \sixel{\bb}{\bb}{\bb}{\ff}{\bb}{\bb} & \sixel{\bb}{\bb}{\bb}{\ff}{\ff}{\bb} & H & X           & \sixel{\bb}{\bb}{\bb}{\ff}{\bb}{\ff} & \sixel{\bb}{\bb}{\bb}{\ff}{\ff}{\ff}\\
 9 &Unused/Reserved&Contiguous Graphics		& \sixel{\ff}{\bb}{\bb}{\ff}{\bb}{\bb} & \sixel{\ff}{\bb}{\bb}{\ff}{\ff}{\bb} & I & Y           & \sixel{\ff}{\bb}{\bb}{\ff}{\bb}{\ff} & \sixel{\ff}{\bb}{\bb}{\ff}{\ff}{\ff}\\
 A &Unused/Reserved&Separated Graphics		& \sixel{\bb}{\ff}{\bb}{\ff}{\bb}{\bb} & \sixel{\bb}{\ff}{\bb}{\ff}{\ff}{\bb} & J & Z           & \sixel{\bb}{\ff}{\bb}{\ff}{\bb}{\ff} & \sixel{\bb}{\ff}{\bb}{\ff}{\ff}{\ff}\\
 B &Unused/Reserved&Unused/Reserved		& \sixel{\ff}{\ff}{\bb}{\ff}{\bb}{\bb} & \sixel{\ff}{\ff}{\bb}{\ff}{\ff}{\bb} & K & $\leftarrow$& \sixel{\ff}{\ff}{\bb}{\ff}{\bb}{\ff} & \sixel{\ff}{\ff}{\bb}{\ff}{\ff}{\ff}\\
 C &Single Height  &Black Background		& \sixel{\bb}{\bb}{\ff}{\ff}{\bb}{\bb} & \sixel{\bb}{\bb}{\ff}{\ff}{\ff}{\bb} & L &$\sfrac{1}{2}$& \sixel{\bb}{\bb}{\ff}{\ff}{\bb}{\ff} & \sixel{\bb}{\bb}{\ff}{\ff}{\ff}{\ff}\\
 D &Double Height  &New Background		& \sixel{\ff}{\bb}{\ff}{\ff}{\bb}{\bb} & \sixel{\ff}{\bb}{\ff}{\ff}{\ff}{\bb} & M &$\rightarrow$ & \sixel{\ff}{\bb}{\ff}{\ff}{\bb}{\ff} & \sixel{\ff}{\bb}{\ff}{\ff}{\ff}{\ff}\\
 E &Unused/Reserved&Hold Graphics		& \sixel{\bb}{\ff}{\ff}{\ff}{\bb}{\bb} & \sixel{\bb}{\ff}{\ff}{\ff}{\ff}{\bb} & N & $\uparrow$  & \sixel{\bb}{\ff}{\ff}{\ff}{\bb}{\ff} & \sixel{\bb}{\ff}{\ff}{\ff}{\ff}{\ff}\\
 F &Unused/Reserved&Release Graphics		& \sixel{\ff}{\ff}{\ff}{\ff}{\bb}{\bb} & \sixel{\ff}{\ff}{\ff}{\ff}{\ff}{\bb} & O & \#           & \sixel{\ff}{\ff}{\ff}{\ff}{\bb}{\ff} & \sixel{\ff}{\ff}{\ff}{\ff}{\ff}{\ff}\\ \hline
\end{tabular}
\caption{The control codes and characters for contiguous graphics mode.}
\label{tab:contgraph}
\end{table}

\begin{table}
\begin{tabular}{|c|c|c|c|c|c|c|c|c|}\hline
   &0x8            & 0x9               		& 0xA&0xB&0xC& 0xD         &0xE& 0xF         \\
 0 &Unused/Reserved&Unused/Reserved    		& \sepsix{\bb}{\bb}{\bb}{\bb}{\bb}{\bb} & \sepsix{\bb}{\bb}{\bb}{\bb}{\ff}{\bb} & @ & P           & \sepsix{\bb}{\bb}{\bb}{\bb}{\bb}{\ff} & \sepsix{\bb}{\bb}{\bb}{\bb}{\ff}{\ff}\\
 1 &Red Alphanumeric       &Red Graphics	& \sepsix{\ff}{\bb}{\bb}{\bb}{\bb}{\bb} & \sepsix{\ff}{\bb}{\bb}{\bb}{\ff}{\bb} & A & Q           & \sepsix{\ff}{\bb}{\bb}{\bb}{\bb}{\ff} & \sepsix{\ff}{\bb}{\bb}{\bb}{\ff}{\ff}\\
 2 &Green Alphanumeric     &Green Graphics	& \sepsix{\bb}{\ff}{\bb}{\bb}{\bb}{\bb} & \sepsix{\bb}{\ff}{\bb}{\bb}{\ff}{\bb} & B & R           & \sepsix{\bb}{\ff}{\bb}{\bb}{\bb}{\ff} & \sepsix{\bb}{\ff}{\bb}{\bb}{\ff}{\ff}\\
 3 &Yellow Alphanumeric    &Yellow Graphics	& \sepsix{\ff}{\ff}{\bb}{\bb}{\bb}{\bb} & \sepsix{\ff}{\ff}{\bb}{\bb}{\ff}{\bb} & C & S           & \sepsix{\ff}{\ff}{\bb}{\bb}{\bb}{\ff} & \sepsix{\ff}{\ff}{\bb}{\bb}{\ff}{\ff}\\
 4 &Blue   Alphanumeric    &Blue   Graphics	& \sepsix{\bb}{\bb}{\ff}{\bb}{\bb}{\bb} & \sepsix{\bb}{\bb}{\ff}{\bb}{\ff}{\bb} & D & T           & \sepsix{\bb}{\bb}{\ff}{\bb}{\bb}{\ff} & \sepsix{\bb}{\bb}{\ff}{\bb}{\ff}{\ff}\\
 5 &Magenta Alphanumeric   &Magenta Graphics	& \sepsix{\ff}{\bb}{\ff}{\bb}{\bb}{\bb} & \sepsix{\ff}{\bb}{\ff}{\bb}{\ff}{\bb} & E & U           & \sepsix{\ff}{\bb}{\ff}{\bb}{\bb}{\ff} & \sepsix{\ff}{\bb}{\ff}{\bb}{\ff}{\ff}\\
 6 &Cyan Alphanumeric      &Cyan Graphics	& \sepsix{\bb}{\ff}{\ff}{\bb}{\bb}{\bb} & \sepsix{\bb}{\ff}{\ff}{\bb}{\ff}{\bb} & F & V           & \sepsix{\bb}{\ff}{\ff}{\bb}{\bb}{\ff} & \sepsix{\bb}{\ff}{\ff}{\bb}{\ff}{\ff}\\
 7 &White Alphanumeric     &White Graphics	& \sepsix{\ff}{\ff}{\ff}{\bb}{\bb}{\bb} & \sepsix{\ff}{\ff}{\ff}{\bb}{\ff}{\bb} & G & W           & \sepsix{\ff}{\ff}{\ff}{\bb}{\bb}{\ff} & \sepsix{\ff}{\ff}{\ff}{\bb}{\ff}{\ff}\\
 8 &Unused/Reserved&Unused/Reserved		& \sepsix{\bb}{\bb}{\bb}{\ff}{\bb}{\bb} & \sepsix{\bb}{\bb}{\bb}{\ff}{\ff}{\bb} & H & X           & \sepsix{\bb}{\bb}{\bb}{\ff}{\bb}{\ff} & \sepsix{\bb}{\bb}{\bb}{\ff}{\ff}{\ff}\\
 9 &Unused/Reserved&Contiguous Graphics		& \sepsix{\ff}{\bb}{\bb}{\ff}{\bb}{\bb} & \sepsix{\ff}{\bb}{\bb}{\ff}{\ff}{\bb} & I & Y           & \sepsix{\ff}{\bb}{\bb}{\ff}{\bb}{\ff} & \sepsix{\ff}{\bb}{\bb}{\ff}{\ff}{\ff}\\
 A &Unused/Reserved&Separated Graphics		& \sepsix{\bb}{\ff}{\bb}{\ff}{\bb}{\bb} & \sepsix{\bb}{\ff}{\bb}{\ff}{\ff}{\bb} & J & Z           & \sepsix{\bb}{\ff}{\bb}{\ff}{\bb}{\ff} & \sepsix{\bb}{\ff}{\bb}{\ff}{\ff}{\ff}\\
 B &Unused/Reserved&Unused/Reserved		& \sepsix{\ff}{\ff}{\bb}{\ff}{\bb}{\bb} & \sepsix{\ff}{\ff}{\bb}{\ff}{\ff}{\bb} & K & $\leftarrow$& \sepsix{\ff}{\ff}{\bb}{\ff}{\bb}{\ff} & \sepsix{\ff}{\ff}{\bb}{\ff}{\ff}{\ff}\\
 C &Single Height  &Black Background		& \sepsix{\bb}{\bb}{\ff}{\ff}{\bb}{\bb} & \sepsix{\bb}{\bb}{\ff}{\ff}{\ff}{\bb} & L &$\sfrac{1}{2}$& \sepsix{\bb}{\bb}{\ff}{\ff}{\bb}{\ff} & \sepsix{\bb}{\bb}{\ff}{\ff}{\ff}{\ff}\\
 D &Double Height  &New Background		& \sepsix{\ff}{\bb}{\ff}{\ff}{\bb}{\bb} & \sepsix{\ff}{\bb}{\ff}{\ff}{\ff}{\bb} & M &$\rightarrow$ & \sepsix{\ff}{\bb}{\ff}{\ff}{\bb}{\ff} & \sepsix{\ff}{\bb}{\ff}{\ff}{\ff}{\ff}\\
 E &Unused/Reserved&Hold Graphics		& \sepsix{\bb}{\ff}{\ff}{\ff}{\bb}{\bb} & \sepsix{\bb}{\ff}{\ff}{\ff}{\ff}{\bb} & N & $\uparrow$  & \sepsix{\bb}{\ff}{\ff}{\ff}{\bb}{\ff} & \sepsix{\bb}{\ff}{\ff}{\ff}{\ff}{\ff}\\
 F &Unused/Reserved&Release Graphics		& \sepsix{\ff}{\ff}{\ff}{\ff}{\bb}{\bb} & \sepsix{\ff}{\ff}{\ff}{\ff}{\ff}{\bb} & O & \#           & \sepsix{\ff}{\ff}{\ff}{\ff}{\bb}{\ff} & \sepsix{\ff}{\ff}{\ff}{\ff}{\ff}{\ff}\\ \hline
\end{tabular}
\caption{The control codes and characters for separated graphics mode.}
\label{tab:sepgraph}
\end{table}

\subsubsection*{Coloured Text}
By using the control codes $129 - 135$ ($0x81 - 0x87$ in hexadecimal) the rest of the line will
have text in the selected foreground colour.

To change the background colour, you issue a foreground colour code first, and then the "New Background" character. All the following line will now have the appropriate background colour.
You'll typically then need to choose a new foreground text colour.

\subsubsection*{Block Graphics}

Teletext has a very limited ability to output low-resolution block graphics. These
shapes take the place of other characters in the font and are enabled by issuing one
of the coloured graphics codes e.g. {\it red graphics}. At this point the characters
available for printing are as displayed in Table~\ref{tab:contgraph}. These new graphics
characters are made up of six smaller boxes, known as {\it sixels}. Each individual sixel has
a code, either, $1,2,4,8,16$ or $64$ as shown in Figure~\ref{fig:graphcodes}.
\begin{figure}[ht]
\begin{center}
\begin{tabular}{|c|c|}\hline
1 & 2 \\ \hline
4 & 8 \\ \hline
16 & {\bf 64} \\ \hline
\end{tabular}
\end{center}
\caption{Values for computing graphics codes, as added to the base code $160$ ($0xA0$ in hexadecimal).}
\label{fig:graphcodes}
\end{figure}
By adding these values together we can define which
of these sixels are `lit' or not. If we wish the three left-hand
ones to be lit we'd use the base code ($160$) plus $1, 4$ and $16 = 181$ ($0xB5$ in
hexadecimal).
Therefore issuing the coding {\it green graphics} and then code $181$ puts
a green vertical bar on the screen.

Notice in Table~\ref{tab:contgraph} that some other characters are still available,
particularly all capital letters. This allow simple printing of capitals, even
when in graphics mode, and is know as {\it blast-through Text}.

There is another set of block graphics, as shown in Table~\ref{tab:sepgraph}. For these,
each sixel is separated from others by thin vertical and horizontal lines. This mode is known
as {\it separated graphics} mode.

\subsubsection*{Held Graphics}

Generally all control codes are displayed as spaces, in the current
background colour. In the held graphics mode, control code $158$ ($0x9E$
in hexadecimal), control codes are instead displayed as a copy of the most
recently displayed graphics symbol. This permits a limited range of abrupt
display colour changes.  The held graphics character is displayed in the
same contiguous/separated mode as when it was first displayed. If there
has been a change in the text/graphics mode or the normal/double-height
mode since the last graphics character was displayed, the held graphics
character is cleared and control codes once again display as spaces.

To switch held graphics mode off, use the {\it release graphics} control code.

\subsubsection*{Double Height}

By using the {\it double height} control code, characters are displayed as twice their
normal size. Since they span two lines, the control codes and characters
have to be repeated on the next line too, for them to be correctly displayed.
The rule here, is that if a character is to be displayed as double height, the top half
of the character is displayed on the first line, and the bottom half on the next line.
The bottom half is only displayed as double height if the character vertically above it was
also displayed in {\it double height} mode. The character in question need not be the same one.

Note: here we deviate from other definitions of this control code.

\subsubsection*{Some General Guidelines}

\begin{itemize}
\item Characters are considered 7-bit (the 8th bit was typically used for parity
checking over the noisy television signal). Therefore any character less
than $128 (0x80)$ should have $128$ added to it. For, example if you
read in character  $32$ (space), it should be `converted' to character $160$.
\item Each newline on the Teletext page automatically begins with
White text, single height, contiguous graphics, black background, release graphics.
\item With the exception of {\it hold graphics} (see above), control characters are generally rendered in the same manner as a space would
be. If the background is currently red and text colour yellow, say, then the control code would show as an empty red background.
\end{itemize}


\begin{exercise}
Implement a teletext rendering system. The $1000$ character input
file should be read in using \verb^argv[1]^.

There are many ambiguities
as to how various sequences of codes should be rendered. To help with
this, several example files have been posted on the unit web page. 
If there is still doubt, make a best-guess and state your assumptions
in the code.

Submit the testing you have undertaken, including examples and a description
of your strategies. This should convince us that you have tested every line
of code, and that it works correctly. If there are still issues/bugs state
them clearly. Also, point out any bugs that you have successfully found using
these approaches. Submit a file named \verb^testing.txt^, along with any other
files you feel necessary in the appropriate directory.

No particular strategy is mandated. You may wish to explore a couple and briefly
discuss strengths and weaknesses. 

Undertake an extension of your choosing. Examples of these include:
\begin{itemize}
\item A system that allows you to quickly author teletext pages (perhaps
using a recursive-descent parser?)
\item Automatic image to teletext conversion.
\item Automatic (simple) html to teletext conversion (and/or vice-versa).
\end{itemize}
Remember, that the assessment is based on the quality of your coding, so choose
something to demonstrate an aspect of programming or software engineering
that you haven't had a chance to use in the main assignment. Submit a file named
\verb^extension.txt^ outlining, in brief, your contribution. 

\subsection*{Hints}

\begin{itemize}
\item Don't add graphics too early - the
code is easier to test and debug with textual output to begin with.
\item I advise you to use SDL for your graphics output. The library provided previously contained
two functions to deal with printing characters~: \verb^Neill_SDL_ReadFont()^ and
\verb^Neill_SDL_DrawString()^. The font file \verb^m7fixed.fnt^ provides the basic
characters required here, but not the sixels. By understanding how the font data
is rendered, the double height version of the characters should be relatively simple.
\item Don't try to do all aspects of this at once - begin with coloured characters only. Add more
advanced functionality later.
\item Plan how you are going to store your data early in the design process.
Does each character need its own data structure? Does each line? Can this be abstracted?
\end{itemize}

Please create a directory structure, so that I can easily find the
different subsections.  Your testing strategy will be explained in
\verb^testing.txt^, and your extension as \verb^extension.txt^. For
the source and extension sections, make sure there's a
\verb^Makefile^, so that I can easily build the code.

\begin{verbatim}
            ------Top Directory------
            |            |          |           
            |            |          |           
            |            |          |            
            |            |          |             
          source      testing    extension   
            |            |          |
           ...          ...   extension.txt
           ...          ...       ...
         Makefile     Makefile    ...  
                    testing.txt
\end{verbatim}

Bundle all of these up as one {\bf single} \verb^.zip^ submission -
not one for each subsection.
\end{exercise}

\nsection{Guido van Robot}

\begin{center}
\includegraphics[scale=0.75]{./gnuLinuxGvR.jpg}
\end{center}

From \wwwurl{http://gvr.sourceforge.net/}
{\small
\begin{quote}
Guido van Robot can face in one of four directions, north, east, south, and west. He turns only 90 degrees at a time, so he can't face northeast, for instance. In Guido's world, streets run east-west, and are numbered starting at 1. There are no zero or negative street numbers. Avenues run north-south, and are also numbered starting at 1, with no zero or negative avenue numbers. At the intersection of a street and avenue is a corner. Guido moves from one corner to the next in a single movement. Because he can only face in one of four directions, when he moves he changes his location by one avenue, or by one street, but not both!
\end{quote}
}

\subsection*{Simple .wld File}

\begin{verbatim}
Robot 5 4 N 1
Wall 3 2 N 6
Wall 2 3 E 4
Wall 3 6 N 6
Wall 8 3 E 2
Wall 8 6 E
\end{verbatim}
\begin{center}
\includegraphics[scale=0.5]{../Pictures/GvRsimple1.jpg}
\end{center}

\subsection*{\bf Simple .gvr File}
\begin{verbatim}
move
move
move
move
turnoff
\end{verbatim}
\begin{center}
\includegraphics[scale=0.5]{./GvRsimple2.jpg}
\end{center}


\subsection*{Do Loops}
\begin{verbatim}
do 2 :
   putbeeper
   move
turnoff
\end{verbatim}

\subsection*{Conditional Loop}
\begin{verbatim}
while front_is_clear :
   putbeeper
   move
turnoff
\end{verbatim}

\subsection*{Branching}
\begin{verbatim}
do 13 :
   if front_is_clear :
      putbeeper
      move
   else :
      turnleft
turnoff
\end{verbatim}

\subsection*{The Formal Grammar}
{\small
\begin{verbatim}
<PROGRAM>   ::= <BLOCK>
<BLOCK>     ::= "turnoff" |
                "move" <BLOCK> |
                "turnleft" <BLOCK> |
                "pickbeeper" <BLOCK> |
                "putbeeper" <BLOCK> |
                <DO> <BLOCK> |
                <WHILE> <BLOCK> |
                <IF> <BLOCK>
<DO>        ::= "do" <num> ":"
                   <BLOCK>
<WHILE>     ::= "while" <TEST> ":"
                   <BLOCK>
<IF>        ::= "if" <TEST> ":"
                   <BLOCK> |
              "if" <TEST> ":"
                   <BLOCK>
              "else" ":"
                   <BLOCK>
<TEST>      ::= <WALL> | <BEEP> | <COMPASS>
\end{verbatim}
}

{\small
\begin{verbatim}
<WALL>      ::= "front_is_clear" |
                "front_is_blocked" |
                "left_is_clear" |
                "left_is_blocked" |
                "right_is_clear" |
                "right_is_blocked"
<BEEP>      ::= "next_to_a_beeper" |
                "not_next_to_a_beeper" |
                "any_beepers_in_beeper_bag" |
                "no_beepers_in_beeper_bag"
<COMPASS>   ::= "facing_north" |
                "not_facing_north" |
                "facing_south" |
                "not_facing_south" |
                "facing_east" |
                "not_facing_east" |
                "facing_west" |
                "not_facing_west"
\end{verbatim}
}

This ignores some Guido instructions, e.g. \verb^elseif^
and \verb^define^. It also doesn't well explain how to spot
the end of a \verb^DO^ etc. which is marked by a reduction in
indentation.
The definition of \verb^.wld^ files is so simple a recursive
descent parser (and hence grammar) is not required.

\begin{exercise}
\begin{itemize}

\item (25\%) To implement a recursive descent parser - this says
whether or not the given \verb^.gvr^ and \verb^.wld^ follow the formal grammar or not.
The input files are specified via \verb^argv[1]^ (\verb^.gvr^) and \verb^argv[2]^ (\verb^.wld^) .

\item (25\%) To implement an interpreter, so that the instructions are
executed. Printing the world and robot to screen
using simple characters is fine, but many will wish to use SDL.

\item (25\%) To show a testing strategy on the above -
you should give details of
white-box and black-box testing done on your code. Describe any
test-harnesses used. Give examples of the output of many different
programs. Convince me that every line of your C code
has been tested.

\item (25\%) To show an extension to the project in a direction of
your choice. It should demonstrate your understanding of some aspect
of programming or S/W engineering. If you extend the formal grammar
make sure that you show the new, full grammar.

Submit the program(s) and a Makefile so that I can:

\item Compile the parser by typing `make parse'.
\item Compile the interpreter by typing `make interp'.
\item Compile the extension by typing `make extension'.
\item Submit a test strategy report called test.txt. This will include
sample outputs, any code written especially for testing etc.
\item Submit an extension report called `extend.txt'. This is quite
brief and explains the extension attempted.

\item You need to be able to load a world file and a \verb^.gvr^
and say if they are valid of not.
\item Don't try to write the entire program in one go. Try a cut
down version of the grammar first, e.g.:
{\small
\begin{verbatim}
<PROGRAM>   ::= <BLOCK>
<BLOCK>     ::= "turnoff" |
                "move" <BLOCK> |
                "turnleft" <BLOCK> |
                "pickbeeper" <BLOCK> |
                "putbeeper" <BLOCK>
<DO>        ::= "do" <num> ":"
                   <BLOCK>
\end{verbatim}
}
\item Some issues, such as what happens if you hit a wall
are not clear from the formal grammar. In this case, use your
common sense, or do what the real program does.
\end{itemize}
\end{exercise}


\nsection{NLab}

\begin{itemize}
\item The programming language MATLAB (originally available in the late
1970s, for free) is one of the most widely used scientific languages in
the world.
\item One of the most interesting things about MATLAB, is that every
single variable is stored as a $2D$ array - even a scaler integer
is simply a $1\times1$ array \footnote{Actually as the name implies,
they are all stored as matrices, but we will ignore the mathematical
interpretion here.}.
\item Here, we develop a very simple version of this concept - a language
that allows such arrays to be created or read from file, and functions performed
on each part of the array, one element at a time.
\end{itemize}

\subsection*{Examples}

\lstinputlisting[columns=fixed,basicstyle=\small\ttfamily\color{ocre},numbers=none,backgroundcolor=\color{darkgray}]{../Code/Week12/NLab/setprinta.nlb}
\lstinputlisting[columns=fixed,basicstyle=\small\ttfamily\color{ocre},numbers=none,backgroundcolor=\color{darkgray}]{../Code/Week12/NLab/setprinta.nlb}

sets the variable I to have the value $5$, and prints it to the screen:

\lstinputlisting[basicstyle=\scriptsize\ttfamily,frame=none,numbers=none]{setprinta.results}



You can create an array full of ones and add $2$ to each cell of the array:
\lstinputlisting[columns=fixed,basicstyle=\small\ttfamily\color{ocre},numbers=none,backgroundcolor=\color{darkgray}]{../Code/Week12/NLab/setprintb.nlb}

\lstinputlisting[basicstyle=\scriptsize\ttfamily,frame=none,numbers=none]{setprintb.results}


Loops are possible too, here a loop counts from $1$ to $10$ via the variable $I$ and computes factorials in the variable $F$. Both variables are scalars (a $1\times1$ array)~:
\lstinputlisting[columns=fixed,basicstyle=\small\ttfamily\color{ocre},numbers=none,backgroundcolor=\color{darkgray}]{../Code/Week12/NLab/loopa.nlb}

\lstinputlisting[basicstyle=\scriptsize\ttfamily,frame=none,numbers=none]{loopa.results}

Such loops (like in C) have counters stored in a variable. Changing this variable inside the loop can affect when the loop ends~:
\lstinputlisting[columns=fixed,basicstyle=\small\ttfamily\color{ocre},numbers=none,backgroundcolor=\color{darkgray}]{../Code/Week12/NLab/loopb.nlb}

\lstinputlisting[basicstyle=\scriptsize\ttfamily,frame=none,numbers=none]{loopb.results}

As grammar tells you, loops can be nested too~:
\lstinputlisting[columns=fixed,basicstyle=\small\ttfamily\color{ocre},numbers=none,backgroundcolor=\color{darkgray}]{../Code/Week12/NLab/nestedloop.nlb}

\lstinputlisting[basicstyle=\scriptsize\ttfamily,frame=none,numbers=none]{nestedloop.results}

\subsection*{The Formal Grammar}
\lstinputlisting[language=bash,basicstyle=\scriptsize\ttfamily,frame=none,numbers=none]{../Code/Week12/NLab/nlab.grammar}

\begin{exercise}
\begin{itemize}
\item {\bf $30\%$}
Implement a recursive descent parser - this will report
whether or not a given NLab program follows the formal grammar or not.
The input file is specified via \verb^argv[1]^ - there is {\bf no} output if
the input file is {\bf valid}. Elsewise, a non-zero \verb^exit^ is made.

\item {\bf $30\%$}
Extend the parser, so it becomes an interpreter. The instructions are
now `executed'. Do not write a new program for this, simply extend your
existing parser.

\item {\bf $20\%$}
Show a testing strategy on the above -
you should give details of
unit testing, white/black-box testing done on your code. Describe any
test-harnesses used. In addition, give examples of the output of many different
NLab programs. Convince me that every line of your C code
has been tested.

\item {\bf $20\%$}
Show an extension to the project in a direction of
your choice. It should demonstrate your {\bf understanding} of some aspect
of programming or S/W engineering. If you extend the formal grammar
make sure that you show the new, full grammar.
\end{itemize}

\subsection*{Hints}
\begin{itemize}
\item Don't try to write the entire program in one go. Try a cut
down version of the grammar first, e.g.:
\begin{verbatim}
<PROG> ::== "BEGIN" { <INSTRCLIST>
INSTRCLIST ::= "}" | <INSTR> <INSTRCLIST>
<INSTR> ::= <PRINT> | <SET>
<PRINT} ::= "PRINT" <VARNAME>
<SET> ::= <VARNAME> ":=" <POLISHLIST>
<POLISHLIST> ::= <POLISH> <POLISHLIST> | ";"
<POLISH> ::= <VARNAME> | <INTEGER>
\end{verbatim}
\item The language is simply a sequence of words (even the semi-colons),
so use \verb^fscanf()^.
\item Some issues, such as what happens if you use an undefined variable,
or if you use a variable before it is set, are not explained by the formal
grammar. Use your own common-sense, and explain what you have done.
\item Once your parser works, extend it to become an interpreter. DO NOT
aim to parse the program first and then interpret it separately. Interpreting
and parsing are inseparably bound together.
\item Start testing very early - this is a complex beast to test and trying to
do it near the end won't work.
\item In NLab, all variables are global i.e. they are not local to loops etc.
\end{itemize}

\subsection*{Submission}
Your testing strategy will be explained in \verb^testing.txt^, and your extension
as \verb^extension.txt^. For the parser, interpreter and extension sections, make
sure there's a \verb^Makefile^, so that I can easily build the code using \verb^make parse^,
\verb^make interp^ and \verb^make extension^. Submit a single \verb^nlab.zip^ file.

\end{exercise}


\input{cawk}

\begin{exercise}
Write a C program to implement the above formal grammar. Your program
should read in a cawk program (argv[1]) and expect the data
file to be read from standard input (or from argv[2] if specified).

The marks are split as follows:
\begin{itemize}
\item (25\%) To implement a recursive descent parser - this says
whether or not a given CAWK program follows the formal grammar or not.

\item (25\%) To implement an interpreter, so that the instructions are
executed.

\item (25\%) To show a testing strategy on the above -
you should give details of
white-box and black-box testing done on your code. Describe any
test-harnesses used. Give examples of the output of many different
cawk programs.

\item (25\%) To show an extension to the project in a direction of
your choice. It should demonstrate your understanding of some aspect
of programming or S/W engineering. If you extend the formal grammar
make sure that you show the new, full grammar.
\end{itemize}

Submit the program(s) and a Makefile so that I can:

\begin{itemize}
\item Compile the parser by typing `make parse'.
\item Compile the interpreter by typing `make interp'.
\item Compile the extension by typing `make extension'.
\end{itemize}

In addition:
\begin{itemize}
\item Submit a test strategy report called test.txt. This will include
sample outputs, any code written especially for testing etc.
\item Submit an extension report called `extend.txt'. This is quite
brief and explains the extension attempted.
\end{itemize}

\end{exercise}

\input{nal}
\begin{exercise}
\begin{itemize}

\strut

\item {\bf (40\%)}
Implement a parser. The \verb^.nal^ file should be read in using
\verb^argv[1]^.  If the file is parsed correctly, the only output should
be:
\begin{terminaloutput}
Parsed OK
\end{terminaloutput}

\item {\bf (30\%)}
Implement an interpreter, building on top of the parser in the
manner described in the lectures. Do not write a brand new program -
interpretation will be done alongside parsing.

\item {\bf (20\%)}
Submit the testing you have undertaken, including examples and a
description of your strategies. This should convince us that you have
tested every line of code, and that it works correctly. If there are
still issues/bugs state them clearly. Also, point out any bugs that
you have successfully found using these approaches. Submit a file named
\verb^testing.txt^, along with any other files you feel necessary. Due
to the recursive nature of this assignment testing is non-trivial -
simply submitting many \verb^.nal^ files that `work' is not sufficient.
No particular strategy is mandated. You may wish to explore a couple
and briefly discuss strengths and weaknesses.

\item {\bf (10\%)}
Undertake an extension of your choosing.  Remember, that the assessment is
based on the quality of your coding, so choose something to demonstrate
an aspect of programming or software engineering that you haven't
had a chance to use in the main assignment. Submit a file named
\verb^extension.txt^ outlining, in brief, your contribution.
\end{itemize}

\subsection*{Hints}
\begin{itemize}
\item Don't try to write the entire program in one go. Try a cut
down version of the grammar first. Build-up from the \verb^01s^
example given in lectures.
\item Some issues, such as what happens if you use an undefined variable,
or if you use a variable before it is set, are not explained by the formal
grammar. Use your own common-sense, and explain what you have done.
\item Once your parser works, extend it to become an interpreter. DO NOT
aim to parse the program first and then interpret it separately.
Interpreting and parsing are inseparably bound together.
\end{itemize}
 
\subsection*{Submission}

Your testing strategy will be explained in \verb^testing.txt^, and
your extension as \verb^extension.txt^. For the parser, interpreter
and extension sections , make sure there's one \verb^Makefile^, so that I
can easily build the code using \verb^make parse^, \verb^make interp^
and \verb^make extension^. I've given an example \verb^makefile^ in the
usual place, but this is an example only - yours may be different.
I wrote only one program \verb^nal.c^ and built the two
different version by setting a \verb^#define^ {\bf via the makefile with}
\verb^-DINTERP^. Inside the code itself \verb^#ifdef INTERP^ and \verb^#endif^ are used.
Also make sure that basic testing is available using \verb^make testparse^ and \verb^make testinterp^.

\noindent Place all the files required for your submission in a single \verb^.zip^ file called \verb^nal.zip^ - this file will not contain other zipped files.

\end{exercise}

\nsection{NUCLEI}

The programming language LISP, developed in 1958,
is one of the oldest languages still in common use.
The language is famous for: being fully parenthesised (that is,
every instruction is inside its own brackets), having a prefix notation (e.g.
functions are written (PLUS 1 2) and not (1 PLUS 2)) and its efficient
linked-list Car/Cdr structure for (de-)composing lists.

Here, we develop a very simple language inspired by these concepts called
NUCLEI (Neill's UnCommon Lisp Expression Interpreter) and a means to
parse or interpret the instructions.

The interpreter (but not parser) builds on Exercise~\ref{ex:carcdr} -
you'll need to have your own version of the \verb^linked.c^,
\verb^lisp.h^ and \verb^specific.h^ files.

\subsection*{Examples}

Parsing~:
\begin{verbatim}
(
  (SET A '1')
  (PRINT A)
)
\end{verbatim}


\noindent leads to the output~:
\begin{terminaloutput}
Parsed OK
\end{terminaloutput}
\noindent or with the interpreter~:
\begin{terminaloutput}
1
\end{terminaloutput}

The \verb^CONS^ instruction is used to construct lists~:
\begin{verbatim}
(
    (PRINT (CONS '1' (CONS '2' NIL)))
)
\end{verbatim}

\begin{terminaloutput}
Parsed OK
\end{terminaloutput}
\noindent and when interpreted~:
\begin{terminaloutput}
(1 2)
\end{terminaloutput}

The \verb^CAR^ instruction is used to deconstruct lists~:
\begin{verbatim}
(
    (SET A '(5 (1 2 3))')
    (PRINT (CAR A))
)
\end{verbatim}
\begin{terminaloutput}
Parsed OK
\end{terminaloutput}
\noindent and when interpreted~:
\begin{terminaloutput}
5
\end{terminaloutput}


Loops are possible too, here a loop counts down from $5$ to $1$, using the variable \verb^C^ as a counter and a Boolean test~:
\begin{verbatim}
(
   (SET C '5')
   (WHILE (LESS '0' C)(
      (PRINT C)
      (SET A (PLUS '-1' C))
      (SET C A))
   )
)
\end{verbatim}
\begin{terminaloutput}
Parsed OK
\end{terminaloutput}
\noindent and when interpreted~:
\begin{terminaloutput}
5
4
3
2
1
\end{terminaloutput}

\noindent The \verb^IF^ is similar; based on a Boolean, one of two possible sets of instructions are taken~:
\begin{verbatim}
(
   (IF (EQUAL '1' '1') ((PRINT "YES"))((GARBAGE)))
)
\end{verbatim}
Here the parser fails because it doesn't understand \verb^GARBAGE^~:
\begin{terminaloutput}
Was expecting a Function name ?
\end{terminaloutput}
However, the interpreter never gets to the $false$ instruction since the Boolean equates to $true$ and so~:
\begin{terminaloutput}
YES
\end{terminaloutput}


\subsection*{The Formal Grammar}
\lstinputlisting[language=bash,basicstyle=\scriptsize\ttfamily,frame=none,numbers=none]{../Code/Week12/NUCLEI/nuclei.grammar}

\begin{exercise}
\begin{itemize}
\item {\bf $30\%$}
Implement a recursive descent parser - this will report
whether or not a given NUCLEI program follows the formal grammar or not.
The input file is specified via \verb^argv[1]^ - and if the file is valid the output is~:
\begin{terminaloutput}
Parsed OK
\end{terminaloutput}
Otherwise, a suitable error message is given and a non-zero \verb^exit^ is made.

\item {\bf $30\%$}
Extend the parser, so it becomes an interpreter. The instructions are
now `executed'. Do not write a new program for this, simply extend your
existing parser. To help with this, I've provided a Makefile that does
some conditional compilation - it effectively does a~:
\begin{codesnippet}
#define INTERP 
\end{codesnippet}
depending upon whether you're compiling the parser or interpreter version of the code.
In the C file, you can do conditional compilation using the \verb^#ifdef^~:
\begin{codesnippet}
#ifdef INTERP
      return Listfunc(s);
#else
      Listfunc(s);
      return;
#endif
\end{codesnippet}

\item {\bf $20\%$}
Show a testing strategy on the above in \verb^testing.txt^ - you should
give details of unit testing, white/black-box testing done on your code,
or any test-harnesses used.  Convince me that every line of your C code
has been tested, but not just by showing it running on some NUCLEI files.

\item {\bf $20\%$}
Show an extension to the project in a direction of your choice. It should
demonstrate your {\bf understanding} of some additional aspect of programming or
S/W engineering. If you extend the formal grammar make sure that you
show the new, full grammar.

\end{itemize}


\subsection*{Hints}
\begin{itemize}
\item Don't try to write the entire program in one go. Try a cut
down version of the grammar first, maybe something similar to~:
\lstinputlisting[language=bash,basicstyle=\scriptsize\ttfamily,frame=none,numbers=none]{../Code/Week12/NUCLEI/cutdown.grammar}
\item Some issues, such as what happens if you use an undefined variable,
or if you use a variable before it is set, are not explained by the formal
grammar. Use your own common-sense, and explain what you have done.
\item Once your parser works, extend it to become an interpreter. DO NOT
aim to parse the program first and then interpret it separately. Interpreting
and parsing are inseparably bound together.
\item Start testing very early - this is a complex beast to test and trying to
do it near the end won't work.
\item In NUCLEI, all variables are global i.e. they are not local to loops etc.
\end{itemize}

\subsection*{Submission}
Your testing strategy will be explained in \verb^testing.txt^, and your extension
as \verb^extension.txt^. For the parser, interpreter and extension sections, make
sure there's a \verb^Makefile^, so that I can easily build the code using \verb^make parse^,
\verb^make interp^ and \verb^make extension^. Submit a single \verb^nuceli.zip^
file, which has all the files required without sub-directories.

\end{exercise}


\input{turtle}

\appendix
\begin{appendices}

\chapterimage{../Pictures/style.jpg}
\chapter{House Style}
\label{appendix:style}
\nsection{Correctness}

These style rules ensure your code is as-correct-as-can-be with the aid
of the compiler and other tools:
\begin{description}
\item[FLAGS] Having no warnings (or errors!) when compiling and executing with the flags:\\
For array bounds checking, \verb^NULL^ pointers being dereferenced etc:
\begin{terminaloutput}
-Wall -Wextra -Wfloat-equal -Wvla -pedantic -std=c99
-fsanitize=undefined -fsanitize=address -g3
\end{terminaloutput}
For memory leaks:
\begin{terminaloutput}
-Wall -Wextra -Wfloat-equal -Wvla -pedantic -std=c99
-g3
\end{terminaloutput}
then run:
\begin{terminaloutput}
valgrind --leak-check=full ./myexec
\end{terminaloutput}
For `final' production-ready code:
\begin{terminaloutput}
-Wall -Wextra -Wfloat-equal -Wvla -pedantic -std=c99
-O3
\end{terminaloutput}

You can use more flags than this, obviously, but these will
make sure a few of the essential warnings that commonly indicate
the presence of bugs and leaks are checked. These guidelines are meant to
be independent of the particular compiler used though. Sometimes it is helpful to use many compilers too, e.g. \verb^gcc^ and \verb^clang^.

If you have unused variables (for example) in your code, it doesn't matter whether your compiler happened to tell you about it or not - it's still wrong~!

\item[BRACE] Always brace all functions, \verb^for^s, \verb^while^s, \verb^if/else^ etc.
Somewhat controversial, this ensures that `extra' lines tagged onto loops are
dealt with correctly. For instance:
\begin{codesnippet}
while(i < 10)
   printf("%i\n", i);
   i++;
\end{codesnippet}
looks like it should print out \verb^i^ $10$ times, but instead runs infinitely.
The programmer probably meant:
\begin{codesnippet}
while(i < 10){
   printf("%i\n", i);
   i++;
}
\end{codesnippet}

\item[GOTO] You do not use any of the keywords \verb^continue^, \verb^goto^ or \verb^break^. The one exception is inside \verb^switch^, where \verb^break^ is allowed because it is essential~!
These keywords usually lead to tangled and complex `spaghetti' coding style.
I often recommend that you rewrite the offending code using functions, which {\bf can} have multiple \verb^return^s
in them.

\item[NAMES] Meaningful identifiers. Make sure that functions names and variables having meaningful, but succinct,  names.

\item[REPC] Repetitive code. If you've cut-and-paste large chunks of code, and made minor changes to it, you've done it wrong. Make it a function, and pass parameters that make the changes required.
\begin{codesnippet}
int inbounds1(int i){
   if(i >=0 && i < MAX){
      return 1;
   }
   else{ 
      return 0;
   }
}

int inbounds2(int i){
   if(i >=0 && i < LEN){
      return 1;
   }
   else{ 
      return 0;
   }
}
\end{codesnippet}
might make more sense as:
\begin{codesnippet}
int inbounds2(int i, int mx){
   if(i >=0 && i < mx){
      return 1;
   }
   else{ 
      return 0;
   }
}
\end{codesnippet}

\item[GLOB] No global variables. Global variables are declared `above' \verb^main()^, are in scope of all functions, and can be altered {\bf anywhere} in the code. This makes it rather unclear {\bf which} functions should be reading or writing them. You can make a case for saying that occasionally they could be useful (or better) than the alternatives, but for now, they are banned~!

\item[RETV] Any functions that returns a value, should have it used:
\begin{codesnippet}
scanf("%i", &i);
\end{codesnippet}
is incorrect. It returns a value that is ignored. Instead do:
\begin{codesnippet}
	if(scanf("%i", &i) != 1{
	   /* PANIC */
\end{codesnippet}

The only exceptions are \verb^printf^ and \verb^putchar^ which do return values but
which are typically ignored.

\item[MATCH] For every \verb^fopen^ there should be a matching \verb^fclose^.
For every \verb^malloc^ there should be a \verb^free^.
This helps avoid memory leaks, when your program or functions are later used
in a larger project.

\item[STDERR] When exiting your program in an error state, make sure that you \verb^fprintf^ the error on \verb^stderr^ and not \verb^stdout^. Use \verb^exit^, e.g.
\begin{codesnippet}
if(argc != 2){
   fprintf(stderr, "Usage : %s <filename>\n", argv[0]);
   exit(EXIT_FAILURE);
}
\end{codesnippet}


\end{description}

\nsection{Prettifying}

These rules are about making your code easier to read and having a consistent
style in a form that others are expecting to see.

\begin{description}

\item[LLEN] Line length. Many people use terminal and editors that are
of a fixed-width. Having excessively long lines may cause the viewer to scroll to
off the screen. Keep lines short, perhaps $< 60$ characters. However, in a similar way
to the {\bf FLEN} rule below, it's really about the complexity of the line that's the issue,
not its absolute length. A programmer would generally find:
{\small
\begin{codesnippet}
bool arrcleanse(cell oldarr[HEIGHT][WIDTH], cell newarr[HEIGHT][WIDTH], int h, int w)
\end{codesnippet}
}
a great deal easier to read than:
\begin{codesnippet}
if(a < b && j++ >= szpar(e ? true : false) || h==4){
\end{codesnippet}
despite it being twice as long.

\item[TABS] Don't use tabs to indent your code. Every editor views these differently, so you have no guarantee that I'm seeing the same layout as you do. Use spaces. This also prevents issues when cutting-and-pasting from one source to another.

\item[INDENT] Indentation: choose a style for indentation
and keep to it. I happen to use $3$ spaces, put opening braces
for functions on a new line, but at the end of \verb^if^,\verb^else^, \verb^for^, \verb^while^ etc, then close them on a new line, underneath the `i' of the \verb^if^:
\begin{codesnippet}
int smallest(int a, int b)
{
   if(a < b){
      return a;
   }
   else{
      return b;
   }
}
\end{codesnippet}
You can use any style you like, as long as it's consistent.

\item[MAIN] The code should have function prototypes/definitions first, then \verb^main()^, followed by the function implementation. This means the reader always know where to find \verb^main()^, since it will be near the top of the file.


\item[CAPS] Constants are \verb^#define^d, and use all CAPITALS. For instance:
\begin{codesnippet}
#define WEEKS 52
#define MAX(a,b) (a < b ? b:a)
\end{codesnippet}


\item[FLEN] Short functions. All functions are short. It's quite
difficult to put a maximum number of lines on this, but use $20$ as a
starting point. Exceptions include a function that simply prints a list
of instructions. There would be no benefit in splitting it into smaller
functions. Short functions are easier to plan, write and test.

I find it more useful to think about how hard the function is to
understand, rather than its length. Therefore, a $30$ line, simple
function is fine, but an extremely complex and dense $15$ line function
might need to be split up, or more self-documentation added.


\end{description}

\nsection{Readability}

Your code should be self-documenting. Comments will be written when there
is something complex to explain, and only read when something has gone
catastrophically wrong. In many cases clever use of coding will avoid the
need for them.  The compiler never sees them, so cannot check them. If you
change your code, but not your comments, they can be highly misleading.

As Kevlin Henney said~:
\begin{quote}
A common fallacy is to assume authors of incomprehensible code will somehow
be able to express themselves lucidly and clearly in comments. 
\end{quote}

\begin{description}
\item[MAGIC] No magic numbers. There should be no inexplicable numbers in your code, such
as:
\begin{codesnippet}
	if(i < 36){
\end{codesnippet}

It's probably unclear to the reader where the $36$ has come from, or what it means,
even if it is obvious to the programmer at the time of writing the code. Instead,
\verb^#define^ them with a meaningful name.
Array overruns are often cured by being consistent with \verb^#define^s.

\item[BRIEF] Comments are brief, and non-trivial. Worthless commenting often
looks something like:
\begin{codesnippet}
// Set the variable i to zero
int i = 0;
\end{codesnippet}
The programmer extracts no additional information from it. However, for more
difficult edge cases, a comment might be useful.
\begin{codesnippet}
// Have we reached the end of the list ?
if(t1->h == NULL){ 
\end{codesnippet}

\noindent To prevent lines from becoming too long, it is good practice to put comments above
the line it refers to, not at the end of the same line.

\item[TYPE] You should use \verb^typedef^s, \verb^enum^s and \verb^struct^s to
increase readability. 

\item[INFIN] No loops should be infinite. I'll never ask you to write a program that is meant to run forever. Therefore statements such as
\begin{codesnippet}
while(1){
\end{codesnippet}
or
\begin{codesnippet}
for(;;;){
\end{codesnippet}
are to be avoided.

\item[2DINDEX] 2D Arrays in C are indexed \verb^[row][col]^.  Sometimes
it may still work correctly, especially if you've consistently confused
the two.  Therefore, if you write code that indexes it \verb^[col][row]^,
or \verb^[x][y]^ it will confuse anyone else trying to understand (or
reuse) your code. If you were to sketch a graph using \verb^(i,j)^ you'd
almost certainly make $i$ the horizontal axis, and $j$ the vertical.
Therefore, for any two variables it makes more sense to write \verb^[b][a]^
or \verb^[j][i]^.

\end{description}


%\chapterimage{../Pictures/peer2peer.jpg}
%\chapter{Peer Assessment}
%\input{peer}

%\include{labeexam}

\end{appendices}

\end{document}
