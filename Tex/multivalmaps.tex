\nsection{MultiValue Maps}

Many data types concern a single value (e.g. a hash table), so that
a string (say) acts as both the key (by which we search for the data)
and also as the object we need to store (the value). An example of this a spelling checker,
where one word is stored (and searched for) at a time.  However, sometimes
there is a need to store a value based on a particular key - for instance
an associative array in Python allows you to perform operations such as :
\begin{codesnippet}
population["Bristol"] = 536000
\end{codesnippet}
where a value (the number 536000) is stored using the key (the string "Bristol").
One decision you need to make when designing such a data type is whether
multiple values are allowed for the same key; in the above example this
would make no sense - Bristol can only have one population size. But if
you wanted to store people as the key, with their salary as the value,
you might need to use a MultiValue Map (MVM) since people can have more than
one job.

Here we write the abstract type for a MultiValueMap that stores key-value pairs,
where both the key and the value are strings.

\begin{exercise}
\label{ex:mvm}
The definition of an MVM ADT is given in \verb^mvm.h^, and a file to test it is given
in \verb^testmvm.c^.  Write \verb^mvm.c^, so that:
{\small
\begin{terminaloutput}
% make -f mvm_adt.mk
./testmvm
Basic MVM Tests ... Start
Basic MVM Tests ... Stop
\end{terminaloutput}
}
\noindent works correctly. Use a simple linked list for this, inserting
new items at the head of the list.
Make no changes to any of my files.
\end{exercise}
