\nsection{NUCLEI}

The programming language LISP, developed in 1958,
is one of the oldest languages still in common use.
The language is famous for: being fully parenthesised (that is,
every instruction is inside its own brackets), having a prefix notation (e.g.
functions are written (PLUS 1 2) and not (1 PLUS 2)) and its efficient
linked-list Car/Cdr structure for (de-)composing lists.

Here, we develop a very simple language inspired by these concepts called
NUCLEI (Neill's UnCommon Lisp Expression Interpreter) and a means to
parse or interpret the instructions.

The interpreter (but not parser) builds on Exercise~\ref{ex:carcdr} -
you'll need to have your own version of the \verb^linked.c^,
\verb^lisp.h^ and \verb^specific.h^ files.

\subsection*{Examples}

Parsing~:
\begin{verbatim}
(
  (SET A '1')
  (PRINT A)
)
\end{verbatim}


\noindent leads to the output~:
\begin{terminaloutput}
Parsed OK
\end{terminaloutput}
\noindent or with the interpreter~:
\begin{terminaloutput}
1
\end{terminaloutput}

The \verb^CONS^ instruction is used to construct lists~:
\begin{verbatim}
(
    (PRINT (CONS '1' (CONS '2' NIL)))
)
\end{verbatim}

\begin{terminaloutput}
Parsed OK
\end{terminaloutput}
\noindent and when interpreted~:
\begin{terminaloutput}
(1 2)
\end{terminaloutput}

The \verb^CAR^ instruction is used to deconstruct lists~:
\begin{verbatim}
(
    (SET A '(5 (1 2 3))')
    (PRINT (CAR A))
)
\end{verbatim}
\begin{terminaloutput}
Parsed OK
\end{terminaloutput}
\noindent and when interpreted~:
\begin{terminaloutput}
5
\end{terminaloutput}


Loops are possible too, here a loop counts down from $5$ to $1$, using the variable \verb^C^ as a counter and a Boolean test~:
\begin{verbatim}
(
   (SET C '5')
   (WHILE (LESS '0' C)(
      (PRINT C)
      (SET A (PLUS '-1' C))
      (SET C A))
   )
)
\end{verbatim}
\begin{terminaloutput}
Parsed OK
\end{terminaloutput}
\noindent and when interpreted~:
\begin{terminaloutput}
5
4
3
2
1
\end{terminaloutput}

\noindent The \verb^IF^ is similar; based on a Boolean, one of two possible sets of instructions are taken~:
\begin{verbatim}
(
   (IF (EQUAL '1' '1') ((PRINT "YES"))((GARBAGE)))
)
\end{verbatim}
Here the parser fails because it doesn't understand \verb^GARBAGE^~:
\begin{terminaloutput}
Was expecting a Function name ?
\end{terminaloutput}
However, the interpreter never gets to the $false$ instruction since the Boolean equates to $true$ and so~:
\begin{terminaloutput}
YES
\end{terminaloutput}


\subsection*{The Formal Grammar}
\lstinputlisting[language=bash,basicstyle=\scriptsize\ttfamily,frame=none,numbers=none]{../Code/Week12/NUCLEI/nuclei.grammar}

\begin{exercise}
\begin{itemize}
\item {\bf $30\%$}
Implement a recursive descent parser - this will report
whether or not a given NUCLEI program follows the formal grammar or not.
The input file is specified via \verb^argv[1]^ - and if the file is valid the output is~:
\begin{terminaloutput}
Parsed OK
\end{terminaloutput}
Otherwise, a suitable error message is given and a non-zero \verb^exit^ is made.

\item {\bf $30\%$}
Extend the parser, so it becomes an interpreter. The instructions are
now `executed'. Do not write a new program for this, simply extend your
existing parser. To help with this, I've provided a Makefile that does
some conditional compilation - it effectively does a~:
\begin{codesnippet}
#define INTERP 
\end{codesnippet}
depending upon whether you're compiling the parser or interpreter version of the code.
In the C file, you can do conditional compilation using the \verb^#ifdef^~:
\begin{codesnippet}
#ifdef INTERP
      return Listfunc(s);
#else
      Listfunc(s);
      return;
#endif
\end{codesnippet}

\item {\bf $20\%$}
Show a testing strategy on the above in \verb^testing.txt^ - you should
give details of unit testing, white/black-box testing done on your code,
or any test-harnesses used.  Convince me that every line of your C code
has been tested, but not just by showing it running on some NUCLEI files.

\item {\bf $20\%$}
Show an extension to the project in a direction of your choice. It should
demonstrate your {\bf understanding} of some additional aspect of programming or
S/W engineering. If you extend the formal grammar make sure that you
show the new, full grammar.

\end{itemize}


\subsection*{Hints}
\begin{itemize}
\item Don't try to write the entire program in one go. Try a cut
down version of the grammar first, maybe something similar to~:
\lstinputlisting[language=bash,basicstyle=\scriptsize\ttfamily,frame=none,numbers=none]{../Code/Week12/NUCLEI/cutdown.grammar}
\item Some issues, such as what happens if you use an undefined variable,
or if you use a variable before it is set, are not explained by the formal
grammar. Use your own common-sense, and explain what you have done.
\item Once your parser works, extend it to become an interpreter. DO NOT
aim to parse the program first and then interpret it separately. Interpreting
and parsing are inseparably bound together.
\item Start testing very early - this is a complex beast to test and trying to
do it near the end won't work.
\item In NUCLEI, all variables are global i.e. they are not local to loops etc.
\end{itemize}

\subsection*{Submission}
Your testing strategy will be explained in \verb^testing.txt^, and your extension
as \verb^extension.txt^. For the parser, interpreter and extension sections, make
sure there's a \verb^Makefile^, so that I can easily build the code using \verb^make parse^,
\verb^make interp^ and \verb^make extension^. Submit a single \verb^nuceli.zip^
file, which has all the files required without sub-directories.

\end{exercise}
