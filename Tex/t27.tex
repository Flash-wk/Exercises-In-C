\nsection{27-Way Trees}
\label{ex:t27}

\noindent The Dictionary Abstract Data Type (ADT) allows words to be
stored in such that they may be efficiently checked later, to ensure that words
have been spelled correctly etc. There are many ways of implementing the
Dictionary ADT (binary trees, hash tables and so on), but here we look
at a tree structure which has a dynamic collection of nodes.

\noindent The 27-Way Tree has $27$ downward links, each
corresponding to one of the letters $a \dots z$ and one to the
apostrophe character \textquotesingle .  This has many similarities with {\it tries}:
\wwwurl{https://en.wikipedia.org/wiki/Trie}

\noindent Any word can be stored in the tree. You start at the top node,
and, given the first character follow the relevant downward pointer.
So, if the first character is `a' you follow the first pointer down from
the first node to another. If the next letter is `s' then you follow the
$19^{th}$ downwards pointer from that node to the following one. Each
pointer corresponds to a letter (or the apostrophe), and each level of the tree corresponds
to the length of a word.

\begin{forest}
%for tree={s sep=10mm, inner sep=0, l=0}
for tree={s sep=30mm}
[$\strut$,draw
   [$\strut$,draw,edge
   label={node[midway,left,text=ocre,font=\scriptsize]{c}}
      [$\strut$,draw,edge
      label={node[midway,left,text=ocre,font=\scriptsize]{a}}
         [$\strut$,draw,fill=black,edge
         label={node[midway,left,text=ocre,font=\scriptsize]{r}}
            [$\strut$,draw,fill=black,edge
            label={node[midway,left,text=ocre,font=\scriptsize]{t}} ]
         ]
      ]
   ] [$\strut$,draw,edge
   label={node[midway,right,text=ocre,font=\scriptsize]{p}}
      [$\strut$,draw,edge
      label={node[midway,right,text=ocre,font=\scriptsize]{a}}
         [$\strut$,draw,edge
         label={node[midway,right,text=ocre,font=\scriptsize]{r}}
            [$\strut$,draw,fill=black,edge
            label={node[midway,right,text=ocre,font=\scriptsize]{t}} ]
         ]
      ]
   ]
]
\end{forest}

\noindent In the above diagram, the words `car', `cart' and `part' have
been stored in such a 27-Way Tree. Since `car' is a substring of `cart'
we have to find a way of making it clear where valid word ending are. In
the diagram, black-filled nodes show such {\it terminal} nodes. The
string `ca' is not a valid word since the block immediately afterwards
has not been marked as such. The word `par' is not valid either, since
it has never been added to the dictionary (but could, in principle,
be added later).


\begin{exercise}
Write an ADT to implement the \verb^t27.h^ interface given.  Write the
source file \verb^t27.c^ such that the driver file \verb^driver.c^
works correctly.  Ensure that this all works with my \verb^Makefile^
which, as with \verb^t27.h^ {\bf must be} used {\bf unaltered}.
The standard operations are~:
\begin{verbatim}
dict* dict_init(void);
bool dict_addword(dict* p, const char* str);
dict* dict_spell(const dict* p, const char* str);
int dict_nodecount(const dict* p);
int dict_wordcount(const dict* p);
int dict_mostcommon(const dict* p);
void dict_free(dict** p);
\end{verbatim}

\noindent The function \verb^dict_addword()^ allows a string to be
inserted into the tree, and the node after marked as a terminal node.

\noindent \verb^dict_spell()^ returns a pointer to the terminal
node corresponding to the string (if it has been inserted) and \verb^NULL^
otherwise.

\noindent \verb^dict_nodecount()^ returns the total number of nodes
which are part of the tree, and \verb^dict_wordcount()^ returns the total
number of words which have been inserted into the tree {\bf including}
duplicates.  A frequency counter is used in each node to keep a track
of words which have been added multiple times. In this case this counter
is incremented, but no new nodes are created.

\noindent \verb^dict_mostcommon()^ returns the number of times the most
common word has been inserted.

\noindent The above functions are the standard assignment and are worth
$60\%$.  There are two additional advanced functions for you to implement, each
worth up to $10\%$ of the assignemnt:
\begin{verbatim}
unsigned dict_cmp(dict* p1, dict* p2);
\end{verbatim}
which counts the least number of nodes you must traverse to move from one node
to another in the tree. In the above figure, there are $7$ steps from `car'
to `part'. The other is:
\begin{verbatim}
void dict_autocomplete(const dict* p, const char* wd, char* ret);
\end{verbatim}
which returns the substring corresponding to the additional letters required to
get from the \verb^wd^ to the most frequent word below it. In the figure, to
autocomplete `car' would be the word `cart', so the additional letters required
are simply `t', which is stored in \verb^ret^.

\begin{itemize}

\item I've provided the structure that {\bf must} be used for this assignment.
Do not attempt to change it. All functionality required is possible with
this structure.

\item For questions such as "but what about hyphenated words" etc., look
in the \verb^driver.c^ file. If I haven't tested for it in there, make a
sensible decision as to what should happen. Provided that this file still
operates correctly you can decide some of the edge-cases for yourself.

\item Even if you don't get all the functions to work correctly, make
sure the code still compiles by writing `dummy' functions as placeholders,
even if some of the assertions fail. I'll test the basic functionality
separately from the two advanced ones.

\item This assignment is brand new. If minor typos
etc. are unconvered as you complete the assignment I
may update this documentation, or the online files
provided. Please check regularly to make sure you
are using the most recent version.

\end{itemize}

\vspace*{1cm}
\noindent {\bf This is worth $80\%$ of the marks ($60\%+10\%+10\%$)}
\vspace*{1ex}

\noindent
The manner in which you've been asked to implement the functionality above is {\bf very}
specific. However, exactly the same functionality (adding/spell-checking etc.)
could be implemented in very different ways (linked lists, BSTs, hashing etc.).
As an extension, write a `rival' version to implement
the standard functionality of these functions without using a 27-Way Tree.
This should still ensure that the standard functionality in \verb^driver.c^
can be compiled against it. This means you
can ignore \verb^dict_autocomplete()^ and \verb^dict_cmp()^ since these
analyse something very
specific about the `shape' of the 27-Way Tree data structure.  Use your own
Makefile to achieve this, and put them in the sub-directory 
\verb^Extension^.

If you do this, also submit \verb^extension.txt^ which details what you
have done, the motivation for it, and how well it works in practice (or
not), and comparing it with the original tree.
This discussion (a few hundred words at most) is as (if not more)
important than the code itself. You may wish to discuss
which of the two methods is better, in terms of actual speed, complexity, memory usage, testability etc.

\noindent {\bf This worth $20\%$ of the marks.}

\end{exercise}
