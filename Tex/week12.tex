%%%%%%%%%%%%%%%%%%%%%%%%%%%% WEEK 12 %%%%%%%%%%%%%%%%%%%%%%%%%%%%%%%%%%%%
\chapterimage{../Pictures/ttl2.jpg}
\chapter{Parsing Data}

\newcommand{\bb}{white}
\newcommand{\ff}{black}

\newcommand{\sixel}[6]{%
\begin{tikzpicture}[scale=0.333, every node/.style={scale=0.333}]
\matrix[sixelstyle]
{
|[fill=#1]| \& |[fill=#2]| \\
|[fill=#3]| \& |[fill=#4]| \\
|[fill=#5]| \& |[fill=#6]| \\
};
\end{tikzpicture}%
}

\newcommand{\sepsix}[6]{%
\begin{tikzpicture}[scale=0.333, every node/.style={scale=0.333}]
\matrix[sepsixstyle]
{
|[fill=#1]| \& |[fill=#2]| \\
|[fill=#3]| \& |[fill=#4]| \\
|[fill=#5]| \& |[fill=#6]| \\
};
\end{tikzpicture}%
}


\input{tikzsets}

\nsection{Teletext}

In the early 1970s, Phillips Labs began work on transmitting digital
information across the television network. The aim was to
provide up-to-date news and weather information via a television set. 
This system was trialled first by the BBC in a system that eventually became
known as ``Ceefax'' and then on other independent British terrestrial stations as ``Oracle''.
A very similar system was implemented on the BBC microcomputer, known as {\it Mode 7}.
\begin{figure}[ht]
\centering{
\includegraphics{../Pictures/teletext100.pdf}
}
\caption{An example Ceefax page circa 1983. Taken from {\tt http://teletext.mb21.co.uk/gallery/ceefax/main1.shtml}}
\label{fig:tt100}
\end{figure}
An example of such a Ceefax screen is shown in Figure~\ref{fig:tt100}.

This project, inspired by such teletext systems, will allow a $40 \times 25$ ($1000$) character
file to be rendered to the screen, using similar control codes. However, some control
codes are not implemented, including those to do with flashing or hidden text, and transparent backgrounds. In particular, our definition of the {\it double height} control code differs from that of
the traditional one.

\subsection*{The Control Codes}

This section is based to a large extent to Richard Russell's description of
Mode 7 on the BBC Micro:
\verb^http://www.bbcbasic.co.uk/tccgen/manual/tcgen2.html^.

\begin{table}
\begin{tabular}{|c|c|c|c|c|c|c|c|c|}\hline
   &0x8            & 0x9               & 0xA&0xB&0xC& 0xD         &0xE& 0xF         \\
 0 &Unused/Reserved&Unused/Reserved    &    & 0 & @ & P           & - & p           \\
 1 &Red Alphanumeric       &Red Graphics       & !  & 1 & A & Q           & a & q           \\
 2 &Green Alphanumeric     &Green Graphics     & "  & 2 & B & R           & b & r           \\
 3 &Yellow Alphanumeric    &Yellow Graphics    & $\pounds$  & 3 & C & S           & c & s           \\
 4 &Blue   Alphanumeric    &Blue   Graphics    & \$ & 4 & D & T           & d & t           \\
 5 &Magenta Alphanumeric   &Magenta Graphics   & \% & 5 & E & U           & e & u           \\
 6 &Cyan Alphanumeric      &Cyan Graphics      & \& & 6 & F & V           & f & v           \\
 7 &White Alphanumeric     &White Graphics     &  ' & 7 & G & W           & g & w           \\
 8 &Unused/Reserved&Unused/Reserved    &  ( & 8 & H & X           & h & x           \\
 9 &Unused/Reserved&Contiguous Graphics&  ) & 9 & I & Y           & i & y           \\
 A &Unused/Reserved&Separated Graphics &  * & : & J & Z           & j & z           \\
 B &Unused/Reserved&Unused/Reserved    &  + & ; & K & $\leftarrow$& k &$\sfrac{1}{4}$\\
 C &Single Height  &Black Background   &  , & < & L &$\sfrac{1}{2}$& l & $||$           \\
 D &Double Height  &New Background     &  - & = & M &$\rightarrow$ & m &$\sfrac{3}{4}$\\
 E &Unused/Reserved&Hold Graphics      &  . & > & N & $\uparrow$  & n &$\div$    \\
 F &Unused/Reserved&Release Graphics   &  / & ? & O & \#           & o & \textblock           \\ \hline
\end{tabular}
\caption{The control codes and characters for alphanumeric mode. Note here (because we're using white paper) foreground is shown in black and background in white. On a teletext screen we use white on a black background.}
\label{tab:normgraph}
\end{table}

\begin{table}
\begin{tabular}{|c|c|c|c|c|c|c|c|c|}\hline
   &0x8            & 0x9               		& 0xA&0xB&0xC& 0xD         &0xE& 0xF         \\
 0 &Unused/Reserved&Unused/Reserved    		& \sixel{\bb}{\bb}{\bb}{\bb}{\bb}{\bb} & \sixel{\bb}{\bb}{\bb}{\bb}{\ff}{\bb} & @ & P           & \sixel{\bb}{\bb}{\bb}{\bb}{\bb}{\ff} & \sixel{\bb}{\bb}{\bb}{\bb}{\ff}{\ff}\\
 1 &Red Alphanumeric       &Red Graphics	& \sixel{\ff}{\bb}{\bb}{\bb}{\bb}{\bb} & \sixel{\ff}{\bb}{\bb}{\bb}{\ff}{\bb} & A & Q           & \sixel{\ff}{\bb}{\bb}{\bb}{\bb}{\ff} & \sixel{\ff}{\bb}{\bb}{\bb}{\ff}{\ff}\\
 2 &Green Alphanumeric     &Green Graphics	& \sixel{\bb}{\ff}{\bb}{\bb}{\bb}{\bb} & \sixel{\bb}{\ff}{\bb}{\bb}{\ff}{\bb} & B & R           & \sixel{\bb}{\ff}{\bb}{\bb}{\bb}{\ff} & \sixel{\bb}{\ff}{\bb}{\bb}{\ff}{\ff}\\
 3 &Yellow Alphanumeric    &Yellow Graphics	& \sixel{\ff}{\ff}{\bb}{\bb}{\bb}{\bb} & \sixel{\ff}{\ff}{\bb}{\bb}{\ff}{\bb} & C & S           & \sixel{\ff}{\ff}{\bb}{\bb}{\bb}{\ff} & \sixel{\ff}{\ff}{\bb}{\bb}{\ff}{\ff}\\
 4 &Blue   Alphanumeric    &Blue   Graphics	& \sixel{\bb}{\bb}{\ff}{\bb}{\bb}{\bb} & \sixel{\bb}{\bb}{\ff}{\bb}{\ff}{\bb} & D & T           & \sixel{\bb}{\bb}{\ff}{\bb}{\bb}{\ff} & \sixel{\bb}{\bb}{\ff}{\bb}{\ff}{\ff}\\
 5 &Magenta Alphanumeric   &Magenta Graphics	& \sixel{\ff}{\bb}{\ff}{\bb}{\bb}{\bb} & \sixel{\ff}{\bb}{\ff}{\bb}{\ff}{\bb} & E & U           & \sixel{\ff}{\bb}{\ff}{\bb}{\bb}{\ff} & \sixel{\ff}{\bb}{\ff}{\bb}{\ff}{\ff}\\
 6 &Cyan Alphanumeric      &Cyan Graphics	& \sixel{\bb}{\ff}{\ff}{\bb}{\bb}{\bb} & \sixel{\bb}{\ff}{\ff}{\bb}{\ff}{\bb} & F & V           & \sixel{\bb}{\ff}{\ff}{\bb}{\bb}{\ff} & \sixel{\bb}{\ff}{\ff}{\bb}{\ff}{\ff}\\
 7 &White Alphanumeric     &White Graphics	& \sixel{\ff}{\ff}{\ff}{\bb}{\bb}{\bb} & \sixel{\ff}{\ff}{\ff}{\bb}{\ff}{\bb} & G & W           & \sixel{\ff}{\ff}{\ff}{\bb}{\bb}{\ff} & \sixel{\ff}{\ff}{\ff}{\bb}{\ff}{\ff}\\
 8 &Unused/Reserved&Unused/Reserved		& \sixel{\bb}{\bb}{\bb}{\ff}{\bb}{\bb} & \sixel{\bb}{\bb}{\bb}{\ff}{\ff}{\bb} & H & X           & \sixel{\bb}{\bb}{\bb}{\ff}{\bb}{\ff} & \sixel{\bb}{\bb}{\bb}{\ff}{\ff}{\ff}\\
 9 &Unused/Reserved&Contiguous Graphics		& \sixel{\ff}{\bb}{\bb}{\ff}{\bb}{\bb} & \sixel{\ff}{\bb}{\bb}{\ff}{\ff}{\bb} & I & Y           & \sixel{\ff}{\bb}{\bb}{\ff}{\bb}{\ff} & \sixel{\ff}{\bb}{\bb}{\ff}{\ff}{\ff}\\
 A &Unused/Reserved&Separated Graphics		& \sixel{\bb}{\ff}{\bb}{\ff}{\bb}{\bb} & \sixel{\bb}{\ff}{\bb}{\ff}{\ff}{\bb} & J & Z           & \sixel{\bb}{\ff}{\bb}{\ff}{\bb}{\ff} & \sixel{\bb}{\ff}{\bb}{\ff}{\ff}{\ff}\\
 B &Unused/Reserved&Unused/Reserved		& \sixel{\ff}{\ff}{\bb}{\ff}{\bb}{\bb} & \sixel{\ff}{\ff}{\bb}{\ff}{\ff}{\bb} & K & $\leftarrow$& \sixel{\ff}{\ff}{\bb}{\ff}{\bb}{\ff} & \sixel{\ff}{\ff}{\bb}{\ff}{\ff}{\ff}\\
 C &Single Height  &Black Background		& \sixel{\bb}{\bb}{\ff}{\ff}{\bb}{\bb} & \sixel{\bb}{\bb}{\ff}{\ff}{\ff}{\bb} & L &$\sfrac{1}{2}$& \sixel{\bb}{\bb}{\ff}{\ff}{\bb}{\ff} & \sixel{\bb}{\bb}{\ff}{\ff}{\ff}{\ff}\\
 D &Double Height  &New Background		& \sixel{\ff}{\bb}{\ff}{\ff}{\bb}{\bb} & \sixel{\ff}{\bb}{\ff}{\ff}{\ff}{\bb} & M &$\rightarrow$ & \sixel{\ff}{\bb}{\ff}{\ff}{\bb}{\ff} & \sixel{\ff}{\bb}{\ff}{\ff}{\ff}{\ff}\\
 E &Unused/Reserved&Hold Graphics		& \sixel{\bb}{\ff}{\ff}{\ff}{\bb}{\bb} & \sixel{\bb}{\ff}{\ff}{\ff}{\ff}{\bb} & N & $\uparrow$  & \sixel{\bb}{\ff}{\ff}{\ff}{\bb}{\ff} & \sixel{\bb}{\ff}{\ff}{\ff}{\ff}{\ff}\\
 F &Unused/Reserved&Release Graphics		& \sixel{\ff}{\ff}{\ff}{\ff}{\bb}{\bb} & \sixel{\ff}{\ff}{\ff}{\ff}{\ff}{\bb} & O & \#           & \sixel{\ff}{\ff}{\ff}{\ff}{\bb}{\ff} & \sixel{\ff}{\ff}{\ff}{\ff}{\ff}{\ff}\\ \hline
\end{tabular}
\caption{The control codes and characters for contiguous graphics mode.}
\label{tab:contgraph}
\end{table}

\begin{table}
\begin{tabular}{|c|c|c|c|c|c|c|c|c|}\hline
   &0x8            & 0x9               		& 0xA&0xB&0xC& 0xD         &0xE& 0xF         \\
 0 &Unused/Reserved&Unused/Reserved    		& \sepsix{\bb}{\bb}{\bb}{\bb}{\bb}{\bb} & \sepsix{\bb}{\bb}{\bb}{\bb}{\ff}{\bb} & @ & P           & \sepsix{\bb}{\bb}{\bb}{\bb}{\bb}{\ff} & \sepsix{\bb}{\bb}{\bb}{\bb}{\ff}{\ff}\\
 1 &Red Alphanumeric       &Red Graphics	& \sepsix{\ff}{\bb}{\bb}{\bb}{\bb}{\bb} & \sepsix{\ff}{\bb}{\bb}{\bb}{\ff}{\bb} & A & Q           & \sepsix{\ff}{\bb}{\bb}{\bb}{\bb}{\ff} & \sepsix{\ff}{\bb}{\bb}{\bb}{\ff}{\ff}\\
 2 &Green Alphanumeric     &Green Graphics	& \sepsix{\bb}{\ff}{\bb}{\bb}{\bb}{\bb} & \sepsix{\bb}{\ff}{\bb}{\bb}{\ff}{\bb} & B & R           & \sepsix{\bb}{\ff}{\bb}{\bb}{\bb}{\ff} & \sepsix{\bb}{\ff}{\bb}{\bb}{\ff}{\ff}\\
 3 &Yellow Alphanumeric    &Yellow Graphics	& \sepsix{\ff}{\ff}{\bb}{\bb}{\bb}{\bb} & \sepsix{\ff}{\ff}{\bb}{\bb}{\ff}{\bb} & C & S           & \sepsix{\ff}{\ff}{\bb}{\bb}{\bb}{\ff} & \sepsix{\ff}{\ff}{\bb}{\bb}{\ff}{\ff}\\
 4 &Blue   Alphanumeric    &Blue   Graphics	& \sepsix{\bb}{\bb}{\ff}{\bb}{\bb}{\bb} & \sepsix{\bb}{\bb}{\ff}{\bb}{\ff}{\bb} & D & T           & \sepsix{\bb}{\bb}{\ff}{\bb}{\bb}{\ff} & \sepsix{\bb}{\bb}{\ff}{\bb}{\ff}{\ff}\\
 5 &Magenta Alphanumeric   &Magenta Graphics	& \sepsix{\ff}{\bb}{\ff}{\bb}{\bb}{\bb} & \sepsix{\ff}{\bb}{\ff}{\bb}{\ff}{\bb} & E & U           & \sepsix{\ff}{\bb}{\ff}{\bb}{\bb}{\ff} & \sepsix{\ff}{\bb}{\ff}{\bb}{\ff}{\ff}\\
 6 &Cyan Alphanumeric      &Cyan Graphics	& \sepsix{\bb}{\ff}{\ff}{\bb}{\bb}{\bb} & \sepsix{\bb}{\ff}{\ff}{\bb}{\ff}{\bb} & F & V           & \sepsix{\bb}{\ff}{\ff}{\bb}{\bb}{\ff} & \sepsix{\bb}{\ff}{\ff}{\bb}{\ff}{\ff}\\
 7 &White Alphanumeric     &White Graphics	& \sepsix{\ff}{\ff}{\ff}{\bb}{\bb}{\bb} & \sepsix{\ff}{\ff}{\ff}{\bb}{\ff}{\bb} & G & W           & \sepsix{\ff}{\ff}{\ff}{\bb}{\bb}{\ff} & \sepsix{\ff}{\ff}{\ff}{\bb}{\ff}{\ff}\\
 8 &Unused/Reserved&Unused/Reserved		& \sepsix{\bb}{\bb}{\bb}{\ff}{\bb}{\bb} & \sepsix{\bb}{\bb}{\bb}{\ff}{\ff}{\bb} & H & X           & \sepsix{\bb}{\bb}{\bb}{\ff}{\bb}{\ff} & \sepsix{\bb}{\bb}{\bb}{\ff}{\ff}{\ff}\\
 9 &Unused/Reserved&Contiguous Graphics		& \sepsix{\ff}{\bb}{\bb}{\ff}{\bb}{\bb} & \sepsix{\ff}{\bb}{\bb}{\ff}{\ff}{\bb} & I & Y           & \sepsix{\ff}{\bb}{\bb}{\ff}{\bb}{\ff} & \sepsix{\ff}{\bb}{\bb}{\ff}{\ff}{\ff}\\
 A &Unused/Reserved&Separated Graphics		& \sepsix{\bb}{\ff}{\bb}{\ff}{\bb}{\bb} & \sepsix{\bb}{\ff}{\bb}{\ff}{\ff}{\bb} & J & Z           & \sepsix{\bb}{\ff}{\bb}{\ff}{\bb}{\ff} & \sepsix{\bb}{\ff}{\bb}{\ff}{\ff}{\ff}\\
 B &Unused/Reserved&Unused/Reserved		& \sepsix{\ff}{\ff}{\bb}{\ff}{\bb}{\bb} & \sepsix{\ff}{\ff}{\bb}{\ff}{\ff}{\bb} & K & $\leftarrow$& \sepsix{\ff}{\ff}{\bb}{\ff}{\bb}{\ff} & \sepsix{\ff}{\ff}{\bb}{\ff}{\ff}{\ff}\\
 C &Single Height  &Black Background		& \sepsix{\bb}{\bb}{\ff}{\ff}{\bb}{\bb} & \sepsix{\bb}{\bb}{\ff}{\ff}{\ff}{\bb} & L &$\sfrac{1}{2}$& \sepsix{\bb}{\bb}{\ff}{\ff}{\bb}{\ff} & \sepsix{\bb}{\bb}{\ff}{\ff}{\ff}{\ff}\\
 D &Double Height  &New Background		& \sepsix{\ff}{\bb}{\ff}{\ff}{\bb}{\bb} & \sepsix{\ff}{\bb}{\ff}{\ff}{\ff}{\bb} & M &$\rightarrow$ & \sepsix{\ff}{\bb}{\ff}{\ff}{\bb}{\ff} & \sepsix{\ff}{\bb}{\ff}{\ff}{\ff}{\ff}\\
 E &Unused/Reserved&Hold Graphics		& \sepsix{\bb}{\ff}{\ff}{\ff}{\bb}{\bb} & \sepsix{\bb}{\ff}{\ff}{\ff}{\ff}{\bb} & N & $\uparrow$  & \sepsix{\bb}{\ff}{\ff}{\ff}{\bb}{\ff} & \sepsix{\bb}{\ff}{\ff}{\ff}{\ff}{\ff}\\
 F &Unused/Reserved&Release Graphics		& \sepsix{\ff}{\ff}{\ff}{\ff}{\bb}{\bb} & \sepsix{\ff}{\ff}{\ff}{\ff}{\ff}{\bb} & O & \#           & \sepsix{\ff}{\ff}{\ff}{\ff}{\bb}{\ff} & \sepsix{\ff}{\ff}{\ff}{\ff}{\ff}{\ff}\\ \hline
\end{tabular}
\caption{The control codes and characters for separated graphics mode.}
\label{tab:sepgraph}
\end{table}

\subsubsection*{Coloured Text}
By using the control codes $129 - 135$ ($0x81 - 0x87$ in hexadecimal) the rest of the line will
have text in the selected foreground colour.

To change the background colour, you issue a foreground colour code first, and then the "New Background" character. All the following line will now have the appropriate background colour.
You'll typically then need to choose a new foreground text colour.

\subsubsection*{Block Graphics}

Teletext has a very limited ability to output low-resolution block graphics. These
shapes take the place of other characters in the font and are enabled by issuing one
of the coloured graphics codes e.g. {\it red graphics}. At this point the characters
available for printing are as displayed in Table~\ref{tab:contgraph}. These new graphics
characters are made up of six smaller boxes, known as {\it sixels}. Each individual sixel has
a code, either, $1,2,4,8,16$ or $64$ as shown in Figure~\ref{fig:graphcodes}.
\begin{figure}[ht]
\begin{center}
\begin{tabular}{|c|c|}\hline
1 & 2 \\ \hline
4 & 8 \\ \hline
16 & {\bf 64} \\ \hline
\end{tabular}
\end{center}
\caption{Values for computing graphics codes, as added to the base code $160$ ($0xA0$ in hexadecimal).}
\label{fig:graphcodes}
\end{figure}
By adding these values together we can define which
of these sixels are `lit' or not. If we wish the three left-hand
ones to be lit we'd use the base code ($160$) plus $1, 4$ and $16 = 181$ ($0xB5$ in
hexadecimal).
Therefore issuing the coding {\it green graphics} and then code $181$ puts
a green vertical bar on the screen.

Notice in Table~\ref{tab:contgraph} that some other characters are still available,
particularly all capital letters. This allow simple printing of capitals, even
when in graphics mode, and is know as {\it blast-through Text}.

There is another set of block graphics, as shown in Table~\ref{tab:sepgraph}. For these,
each sixel is separated from others by thin vertical and horizontal lines. This mode is known
as {\it separated graphics} mode.

\subsubsection*{Held Graphics}

Generally all control codes are displayed as spaces, in the current
background colour. In the held graphics mode, control code $158$ ($0x9E$
in hexadecimal), control codes are instead displayed as a copy of the most
recently displayed graphics symbol. This permits a limited range of abrupt
display colour changes.  The held graphics character is displayed in the
same contiguous/separated mode as when it was first displayed. If there
has been a change in the text/graphics mode or the normal/double-height
mode since the last graphics character was displayed, the held graphics
character is cleared and control codes once again display as spaces.

To switch held graphics mode off, use the {\it release graphics} control code.

\subsubsection*{Double Height}

By using the {\it double height} control code, characters are displayed as twice their
normal size. Since they span two lines, the control codes and characters
have to be repeated on the next line too, for them to be correctly displayed.
The rule here, is that if a character is to be displayed as double height, the top half
of the character is displayed on the first line, and the bottom half on the next line.
The bottom half is only displayed as double height if the character vertically above it was
also displayed in {\it double height} mode. The character in question need not be the same one.

Note: here we deviate from other definitions of this control code.

\subsubsection*{Some General Guidelines}

\begin{itemize}
\item Characters are considered 7-bit (the 8th bit was typically used for parity
checking over the noisy television signal). Therefore any character less
than $128 (0x80)$ should have $128$ added to it. For, example if you
read in character  $32$ (space), it should be `converted' to character $160$.
\item Each newline on the Teletext page automatically begins with
White text, single height, contiguous graphics, black background, release graphics.
\item With the exception of {\it hold graphics} (see above), control characters are generally rendered in the same manner as a space would
be. If the background is currently red and text colour yellow, say, then the control code would show as an empty red background.
\end{itemize}


\begin{exercise}
Implement a teletext rendering system. The $1000$ character input
file should be read in using \verb^argv[1]^.

There are many ambiguities
as to how various sequences of codes should be rendered. To help with
this, several example files have been posted on the unit web page. 
If there is still doubt, make a best-guess and state your assumptions
in the code.

Submit the testing you have undertaken, including examples and a description
of your strategies. This should convince us that you have tested every line
of code, and that it works correctly. If there are still issues/bugs state
them clearly. Also, point out any bugs that you have successfully found using
these approaches. Submit a file named \verb^testing.txt^, along with any other
files you feel necessary in the appropriate directory.

No particular strategy is mandated. You may wish to explore a couple and briefly
discuss strengths and weaknesses. 

Undertake an extension of your choosing. Examples of these include:
\begin{itemize}
\item A system that allows you to quickly author teletext pages (perhaps
using a recursive-descent parser?)
\item Automatic image to teletext conversion.
\item Automatic (simple) html to teletext conversion (and/or vice-versa).
\end{itemize}
Remember, that the assessment is based on the quality of your coding, so choose
something to demonstrate an aspect of programming or software engineering
that you haven't had a chance to use in the main assignment. Submit a file named
\verb^extension.txt^ outlining, in brief, your contribution. 

\subsection*{Hints}

\begin{itemize}
\item Don't add graphics too early - the
code is easier to test and debug with textual output to begin with.
\item I advise you to use SDL for your graphics output. The library provided previously contained
two functions to deal with printing characters~: \verb^Neill_SDL_ReadFont()^ and
\verb^Neill_SDL_DrawString()^. The font file \verb^m7fixed.fnt^ provides the basic
characters required here, but not the sixels. By understanding how the font data
is rendered, the double height version of the characters should be relatively simple.
\item Don't try to do all aspects of this at once - begin with coloured characters only. Add more
advanced functionality later.
\item Plan how you are going to store your data early in the design process.
Does each character need its own data structure? Does each line? Can this be abstracted?
\end{itemize}

Please create a directory structure, so that I can easily find the
different subsections.  Your testing strategy will be explained in
\verb^testing.txt^, and your extension as \verb^extension.txt^. For
the source and extension sections, make sure there's a
\verb^Makefile^, so that I can easily build the code.

\begin{verbatim}
            ------Top Directory------
            |            |          |           
            |            |          |           
            |            |          |            
            |            |          |             
          source      testing    extension   
            |            |          |
           ...          ...   extension.txt
           ...          ...       ...
         Makefile     Makefile    ...  
                    testing.txt
\end{verbatim}

Bundle all of these up as one {\bf single} \verb^.zip^ submission -
not one for each subsection.
\end{exercise}

\nsection{Guido van Robot}

\begin{center}
\includegraphics[scale=0.75]{./gnuLinuxGvR.jpg}
\end{center}

From \wwwurl{http://gvr.sourceforge.net/}
{\small
\begin{quote}
Guido van Robot can face in one of four directions, north, east, south, and west. He turns only 90 degrees at a time, so he can't face northeast, for instance. In Guido's world, streets run east-west, and are numbered starting at 1. There are no zero or negative street numbers. Avenues run north-south, and are also numbered starting at 1, with no zero or negative avenue numbers. At the intersection of a street and avenue is a corner. Guido moves from one corner to the next in a single movement. Because he can only face in one of four directions, when he moves he changes his location by one avenue, or by one street, but not both!
\end{quote}
}

\subsection*{Simple .wld File}

\begin{verbatim}
Robot 5 4 N 1
Wall 3 2 N 6
Wall 2 3 E 4
Wall 3 6 N 6
Wall 8 3 E 2
Wall 8 6 E
\end{verbatim}
\begin{center}
\includegraphics[scale=0.5]{../Pictures/GvRsimple1.jpg}
\end{center}

\subsection*{\bf Simple .gvr File}
\begin{verbatim}
move
move
move
move
turnoff
\end{verbatim}
\begin{center}
\includegraphics[scale=0.5]{./GvRsimple2.jpg}
\end{center}


\subsection*{Do Loops}
\begin{verbatim}
do 2 :
   putbeeper
   move
turnoff
\end{verbatim}

\subsection*{Conditional Loop}
\begin{verbatim}
while front_is_clear :
   putbeeper
   move
turnoff
\end{verbatim}

\subsection*{Branching}
\begin{verbatim}
do 13 :
   if front_is_clear :
      putbeeper
      move
   else :
      turnleft
turnoff
\end{verbatim}

\subsection*{The Formal Grammar}
{\small
\begin{verbatim}
<PROGRAM>   ::= <BLOCK>
<BLOCK>     ::= "turnoff" |
                "move" <BLOCK> |
                "turnleft" <BLOCK> |
                "pickbeeper" <BLOCK> |
                "putbeeper" <BLOCK> |
                <DO> <BLOCK> |
                <WHILE> <BLOCK> |
                <IF> <BLOCK>
<DO>        ::= "do" <num> ":"
                   <BLOCK>
<WHILE>     ::= "while" <TEST> ":"
                   <BLOCK>
<IF>        ::= "if" <TEST> ":"
                   <BLOCK> |
              "if" <TEST> ":"
                   <BLOCK>
              "else" ":"
                   <BLOCK>
<TEST>      ::= <WALL> | <BEEP> | <COMPASS>
\end{verbatim}
}

{\small
\begin{verbatim}
<WALL>      ::= "front_is_clear" |
                "front_is_blocked" |
                "left_is_clear" |
                "left_is_blocked" |
                "right_is_clear" |
                "right_is_blocked"
<BEEP>      ::= "next_to_a_beeper" |
                "not_next_to_a_beeper" |
                "any_beepers_in_beeper_bag" |
                "no_beepers_in_beeper_bag"
<COMPASS>   ::= "facing_north" |
                "not_facing_north" |
                "facing_south" |
                "not_facing_south" |
                "facing_east" |
                "not_facing_east" |
                "facing_west" |
                "not_facing_west"
\end{verbatim}
}

This ignores some Guido instructions, e.g. \verb^elseif^
and \verb^define^. It also doesn't well explain how to spot
the end of a \verb^DO^ etc. which is marked by a reduction in
indentation.
The definition of \verb^.wld^ files is so simple a recursive
descent parser (and hence grammar) is not required.

\begin{exercise}
\begin{itemize}

\item (25\%) To implement a recursive descent parser - this says
whether or not the given \verb^.gvr^ and \verb^.wld^ follow the formal grammar or not.
The input files are specified via \verb^argv[1]^ (\verb^.gvr^) and \verb^argv[2]^ (\verb^.wld^) .

\item (25\%) To implement an interpreter, so that the instructions are
executed. Printing the world and robot to screen
using simple characters is fine, but many will wish to use SDL.

\item (25\%) To show a testing strategy on the above -
you should give details of
white-box and black-box testing done on your code. Describe any
test-harnesses used. Give examples of the output of many different
programs. Convince me that every line of your C code
has been tested.

\item (25\%) To show an extension to the project in a direction of
your choice. It should demonstrate your understanding of some aspect
of programming or S/W engineering. If you extend the formal grammar
make sure that you show the new, full grammar.

Submit the program(s) and a Makefile so that I can:

\item Compile the parser by typing `make parse'.
\item Compile the interpreter by typing `make interp'.
\item Compile the extension by typing `make extension'.
\item Submit a test strategy report called test.txt. This will include
sample outputs, any code written especially for testing etc.
\item Submit an extension report called `extend.txt'. This is quite
brief and explains the extension attempted.

\item You need to be able to load a world file and a \verb^.gvr^
and say if they are valid of not.
\item Don't try to write the entire program in one go. Try a cut
down version of the grammar first, e.g.:
{\small
\begin{verbatim}
<PROGRAM>   ::= <BLOCK>
<BLOCK>     ::= "turnoff" |
                "move" <BLOCK> |
                "turnleft" <BLOCK> |
                "pickbeeper" <BLOCK> |
                "putbeeper" <BLOCK>
<DO>        ::= "do" <num> ":"
                   <BLOCK>
\end{verbatim}
}
\item Some issues, such as what happens if you hit a wall
are not clear from the formal grammar. In this case, use your
common sense, or do what the real program does.
\end{itemize}
\end{exercise}


\nsection{NLab}

\begin{itemize}
\item The programming language MATLAB (originally available in the late
1970s, for free) is one of the most widely used scientific languages in
the world.
\item One of the most interesting things about MATLAB, is that every
single variable is stored as a $2D$ array - even a scaler integer
is simply a $1\times1$ array \footnote{Actually as the name implies,
they are all stored as matrices, but we will ignore the mathematical
interpretion here.}.
\item Here, we develop a very simple version of this concept - a language
that allows such arrays to be created or read from file, and functions performed
on each part of the array, one element at a time.
\end{itemize}

\subsection*{Examples}

\lstinputlisting[columns=fixed,basicstyle=\small\ttfamily\color{ocre},numbers=none,backgroundcolor=\color{darkgray}]{../Code/Week12/NLab/setprinta.nlb}
\lstinputlisting[columns=fixed,basicstyle=\small\ttfamily\color{ocre},numbers=none,backgroundcolor=\color{darkgray}]{../Code/Week12/NLab/setprinta.nlb}

sets the variable I to have the value $5$, and prints it to the screen:

\lstinputlisting[basicstyle=\scriptsize\ttfamily,frame=none,numbers=none]{setprinta.results}



You can create an array full of ones and add $2$ to each cell of the array:
\lstinputlisting[columns=fixed,basicstyle=\small\ttfamily\color{ocre},numbers=none,backgroundcolor=\color{darkgray}]{../Code/Week12/NLab/setprintb.nlb}

\lstinputlisting[basicstyle=\scriptsize\ttfamily,frame=none,numbers=none]{setprintb.results}


Loops are possible too, here a loop counts from $1$ to $10$ via the variable $I$ and computes factorials in the variable $F$. Both variables are scalars (a $1\times1$ array)~:
\lstinputlisting[columns=fixed,basicstyle=\small\ttfamily\color{ocre},numbers=none,backgroundcolor=\color{darkgray}]{../Code/Week12/NLab/loopa.nlb}

\lstinputlisting[basicstyle=\scriptsize\ttfamily,frame=none,numbers=none]{loopa.results}

Such loops (like in C) have counters stored in a variable. Changing this variable inside the loop can affect when the loop ends~:
\lstinputlisting[columns=fixed,basicstyle=\small\ttfamily\color{ocre},numbers=none,backgroundcolor=\color{darkgray}]{../Code/Week12/NLab/loopb.nlb}

\lstinputlisting[basicstyle=\scriptsize\ttfamily,frame=none,numbers=none]{loopb.results}

As grammar tells you, loops can be nested too~:
\lstinputlisting[columns=fixed,basicstyle=\small\ttfamily\color{ocre},numbers=none,backgroundcolor=\color{darkgray}]{../Code/Week12/NLab/nestedloop.nlb}

\lstinputlisting[basicstyle=\scriptsize\ttfamily,frame=none,numbers=none]{nestedloop.results}

\subsection*{The Formal Grammar}
\lstinputlisting[language=bash,basicstyle=\scriptsize\ttfamily,frame=none,numbers=none]{../Code/Week12/NLab/nlab.grammar}

\begin{exercise}
\begin{itemize}
\item {\bf $30\%$}
Implement a recursive descent parser - this will report
whether or not a given NLab program follows the formal grammar or not.
The input file is specified via \verb^argv[1]^ - there is {\bf no} output if
the input file is {\bf valid}. Elsewise, a non-zero \verb^exit^ is made.

\item {\bf $30\%$}
Extend the parser, so it becomes an interpreter. The instructions are
now `executed'. Do not write a new program for this, simply extend your
existing parser.

\item {\bf $20\%$}
Show a testing strategy on the above -
you should give details of
unit testing, white/black-box testing done on your code. Describe any
test-harnesses used. In addition, give examples of the output of many different
NLab programs. Convince me that every line of your C code
has been tested.

\item {\bf $20\%$}
Show an extension to the project in a direction of
your choice. It should demonstrate your {\bf understanding} of some aspect
of programming or S/W engineering. If you extend the formal grammar
make sure that you show the new, full grammar.
\end{itemize}

\subsection*{Hints}
\begin{itemize}
\item Don't try to write the entire program in one go. Try a cut
down version of the grammar first, e.g.:
\begin{verbatim}
<PROG> ::== "BEGIN" { <INSTRCLIST>
INSTRCLIST ::= "}" | <INSTR> <INSTRCLIST>
<INSTR> ::= <PRINT> | <SET>
<PRINT} ::= "PRINT" <VARNAME>
<SET> ::= <VARNAME> ":=" <POLISHLIST>
<POLISHLIST> ::= <POLISH> <POLISHLIST> | ";"
<POLISH> ::= <VARNAME> | <INTEGER>
\end{verbatim}
\item The language is simply a sequence of words (even the semi-colons),
so use \verb^fscanf()^.
\item Some issues, such as what happens if you use an undefined variable,
or if you use a variable before it is set, are not explained by the formal
grammar. Use your own common-sense, and explain what you have done.
\item Once your parser works, extend it to become an interpreter. DO NOT
aim to parse the program first and then interpret it separately. Interpreting
and parsing are inseparably bound together.
\item Start testing very early - this is a complex beast to test and trying to
do it near the end won't work.
\item In NLab, all variables are global i.e. they are not local to loops etc.
\end{itemize}

\subsection*{Submission}
Your testing strategy will be explained in \verb^testing.txt^, and your extension
as \verb^extension.txt^. For the parser, interpreter and extension sections, make
sure there's a \verb^Makefile^, so that I can easily build the code using \verb^make parse^,
\verb^make interp^ and \verb^make extension^. Submit a single \verb^nlab.zip^ file.

\end{exercise}


\input{cawk}

\begin{exercise}
Write a C program to implement the above formal grammar. Your program
should read in a cawk program (argv[1]) and expect the data
file to be read from standard input (or from argv[2] if specified).

The marks are split as follows:
\begin{itemize}
\item (25\%) To implement a recursive descent parser - this says
whether or not a given CAWK program follows the formal grammar or not.

\item (25\%) To implement an interpreter, so that the instructions are
executed.

\item (25\%) To show a testing strategy on the above -
you should give details of
white-box and black-box testing done on your code. Describe any
test-harnesses used. Give examples of the output of many different
cawk programs.

\item (25\%) To show an extension to the project in a direction of
your choice. It should demonstrate your understanding of some aspect
of programming or S/W engineering. If you extend the formal grammar
make sure that you show the new, full grammar.
\end{itemize}

Submit the program(s) and a Makefile so that I can:

\begin{itemize}
\item Compile the parser by typing `make parse'.
\item Compile the interpreter by typing `make interp'.
\item Compile the extension by typing `make extension'.
\end{itemize}

In addition:
\begin{itemize}
\item Submit a test strategy report called test.txt. This will include
sample outputs, any code written especially for testing etc.
\item Submit an extension report called `extend.txt'. This is quite
brief and explains the extension attempted.
\end{itemize}

\end{exercise}

\input{nal}
\begin{exercise}
\begin{itemize}

\strut

\item {\bf (40\%)}
Implement a parser. The \verb^.nal^ file should be read in using
\verb^argv[1]^.  If the file is parsed correctly, the only output should
be:
\begin{terminaloutput}
Parsed OK
\end{terminaloutput}

\item {\bf (30\%)}
Implement an interpreter, building on top of the parser in the
manner described in the lectures. Do not write a brand new program -
interpretation will be done alongside parsing.

\item {\bf (20\%)}
Submit the testing you have undertaken, including examples and a
description of your strategies. This should convince us that you have
tested every line of code, and that it works correctly. If there are
still issues/bugs state them clearly. Also, point out any bugs that
you have successfully found using these approaches. Submit a file named
\verb^testing.txt^, along with any other files you feel necessary. Due
to the recursive nature of this assignment testing is non-trivial -
simply submitting many \verb^.nal^ files that `work' is not sufficient.
No particular strategy is mandated. You may wish to explore a couple
and briefly discuss strengths and weaknesses.

\item {\bf (10\%)}
Undertake an extension of your choosing.  Remember, that the assessment is
based on the quality of your coding, so choose something to demonstrate
an aspect of programming or software engineering that you haven't
had a chance to use in the main assignment. Submit a file named
\verb^extension.txt^ outlining, in brief, your contribution.
\end{itemize}

\subsection*{Hints}
\begin{itemize}
\item Don't try to write the entire program in one go. Try a cut
down version of the grammar first. Build-up from the \verb^01s^
example given in lectures.
\item Some issues, such as what happens if you use an undefined variable,
or if you use a variable before it is set, are not explained by the formal
grammar. Use your own common-sense, and explain what you have done.
\item Once your parser works, extend it to become an interpreter. DO NOT
aim to parse the program first and then interpret it separately.
Interpreting and parsing are inseparably bound together.
\end{itemize}
 
\subsection*{Submission}

Your testing strategy will be explained in \verb^testing.txt^, and
your extension as \verb^extension.txt^. For the parser, interpreter
and extension sections , make sure there's one \verb^Makefile^, so that I
can easily build the code using \verb^make parse^, \verb^make interp^
and \verb^make extension^. I've given an example \verb^makefile^ in the
usual place, but this is an example only - yours may be different.
I wrote only one program \verb^nal.c^ and built the two
different version by setting a \verb^#define^ {\bf via the makefile with}
\verb^-DINTERP^. Inside the code itself \verb^#ifdef INTERP^ and \verb^#endif^ are used.
Also make sure that basic testing is available using \verb^make testparse^ and \verb^make testinterp^.

\noindent Place all the files required for your submission in a single \verb^.zip^ file called \verb^nal.zip^ - this file will not contain other zipped files.

\end{exercise}

\nsection{NUCLEI}

The programming language LISP, developed in 1958,
is one of the oldest languages still in common use.
The language is famous for: being fully parenthesised (that is,
every instruction is inside its own brackets), having a prefix notation (e.g.
functions are written (PLUS 1 2) and not (1 PLUS 2)) and its efficient
linked-list Car/Cdr structure for (de-)composing lists.

Here, we develop a very simple language inspired by these concepts called
NUCLEI (Neill's UnCommon Lisp Expression Interpreter) and a means to
parse or interpret the instructions.

The interpreter (but not parser) builds on Exercise~\ref{ex:carcdr} -
you'll need to have your own version of the \verb^linked.c^,
\verb^lisp.h^ and \verb^specific.h^ files.

\subsection*{Examples}

Parsing~:
\begin{verbatim}
(
  (SET A '1')
  (PRINT A)
)
\end{verbatim}


\noindent leads to the output~:
\begin{terminaloutput}
Parsed OK
\end{terminaloutput}
\noindent or with the interpreter~:
\begin{terminaloutput}
1
\end{terminaloutput}

The \verb^CONS^ instruction is used to construct lists~:
\begin{verbatim}
(
    (PRINT (CONS '1' (CONS '2' NIL)))
)
\end{verbatim}

\begin{terminaloutput}
Parsed OK
\end{terminaloutput}
\noindent and when interpreted~:
\begin{terminaloutput}
(1 2)
\end{terminaloutput}

The \verb^CAR^ instruction is used to deconstruct lists~:
\begin{verbatim}
(
    (SET A '(5 (1 2 3))')
    (PRINT (CAR A))
)
\end{verbatim}
\begin{terminaloutput}
Parsed OK
\end{terminaloutput}
\noindent and when interpreted~:
\begin{terminaloutput}
5
\end{terminaloutput}


Loops are possible too, here a loop counts down from $5$ to $1$, using the variable \verb^C^ as a counter and a Boolean test~:
\begin{verbatim}
(
   (SET C '5')
   (WHILE (LESS '0' C)(
      (PRINT C)
      (SET A (PLUS '-1' C))
      (SET C A))
   )
)
\end{verbatim}
\begin{terminaloutput}
Parsed OK
\end{terminaloutput}
\noindent and when interpreted~:
\begin{terminaloutput}
5
4
3
2
1
\end{terminaloutput}

\noindent The \verb^IF^ is similar; based on a Boolean, one of two possible sets of instructions are taken~:
\begin{verbatim}
(
   (IF (EQUAL '1' '1') ((PRINT "YES"))((GARBAGE)))
)
\end{verbatim}
Here the parser fails because it doesn't understand \verb^GARBAGE^~:
\begin{terminaloutput}
Was expecting a Function name ?
\end{terminaloutput}
However, the interpreter never gets to the $false$ instruction since the Boolean equates to $true$ and so~:
\begin{terminaloutput}
YES
\end{terminaloutput}


\subsection*{The Formal Grammar}
\lstinputlisting[language=bash,basicstyle=\scriptsize\ttfamily,frame=none,numbers=none]{../Code/Week12/NUCLEI/nuclei.grammar}

\begin{exercise}
\begin{itemize}
\item {\bf $30\%$}
Implement a recursive descent parser - this will report
whether or not a given NUCLEI program follows the formal grammar or not.
The input file is specified via \verb^argv[1]^ - and if the file is valid the output is~:
\begin{terminaloutput}
Parsed OK
\end{terminaloutput}
Otherwise, a suitable error message is given and a non-zero \verb^exit^ is made.

\item {\bf $30\%$}
Extend the parser, so it becomes an interpreter. The instructions are
now `executed'. Do not write a new program for this, simply extend your
existing parser. To help with this, I've provided a Makefile that does
some conditional compilation - it effectively does a~:
\begin{codesnippet}
#define INTERP 
\end{codesnippet}
depending upon whether you're compiling the parser or interpreter version of the code.
In the C file, you can do conditional compilation using the \verb^#ifdef^~:
\begin{codesnippet}
#ifdef INTERP
      return Listfunc(s);
#else
      Listfunc(s);
      return;
#endif
\end{codesnippet}

\item {\bf $20\%$}
Show a testing strategy on the above in \verb^testing.txt^ - you should
give details of unit testing, white/black-box testing done on your code,
or any test-harnesses used.  Convince me that every line of your C code
has been tested, but not just by showing it running on some NUCLEI files.

\item {\bf $20\%$}
Show an extension to the project in a direction of your choice. It should
demonstrate your {\bf understanding} of some additional aspect of programming or
S/W engineering. If you extend the formal grammar make sure that you
show the new, full grammar.

\end{itemize}


\subsection*{Hints}
\begin{itemize}
\item Don't try to write the entire program in one go. Try a cut
down version of the grammar first, maybe something similar to~:
\lstinputlisting[language=bash,basicstyle=\scriptsize\ttfamily,frame=none,numbers=none]{../Code/Week12/NUCLEI/cutdown.grammar}
\item Some issues, such as what happens if you use an undefined variable,
or if you use a variable before it is set, are not explained by the formal
grammar. Use your own common-sense, and explain what you have done.
\item Once your parser works, extend it to become an interpreter. DO NOT
aim to parse the program first and then interpret it separately. Interpreting
and parsing are inseparably bound together.
\item Start testing very early - this is a complex beast to test and trying to
do it near the end won't work.
\item In NUCLEI, all variables are global i.e. they are not local to loops etc.
\end{itemize}

\subsection*{Submission}
Your testing strategy will be explained in \verb^testing.txt^, and your extension
as \verb^extension.txt^. For the parser, interpreter and extension sections, make
sure there's a \verb^Makefile^, so that I can easily build the code using \verb^make parse^,
\verb^make interp^ and \verb^make extension^. Submit a single \verb^nuceli.zip^
file, which has all the files required without sub-directories.

\end{exercise}


\input{turtle}
