\newcommand{\fivedice}[5]{\epsdice[white]{#1}\epsdice[white]{#2}\epsdice[white]{#3}\epsdice[white]{#4}\epsdice[white]{#5}}
\nsection{Yahtzee}


The game of Yahtzee is a game played with five dice, and you
try to obtain certain `hands'.
In a similar way to poker, these hands could include a {\it Full House} (two dice are the same, and another three are the same), e.g.:

\fivedice{6}{6}{1}{6}{1}
or  
\fivedice{4}{1}{1}{1}{4}

\noindent or another possible hand is {\it Four-of-a-Kind}, e.g.:
\fivedice{3}{3}{1}{3}{3}
or 
\fivedice{5}{5}{5}{2}{5}
(but not
\fivedice{3}{3}{3}{3}{3}
which is {\it Five-of-a-Kind})

A little mathematics tells use that the probability of these two hands should be $3.85\%$ and $1.93\%$ respectively.
\begin{exercise}

Complete the program \verb^yahtzee.c^ (which is in the usual place
online), by analysing a large number of random dice rolls, the probabilty
of each of these two hands.  The five dice of the hand are stored in an
array, and to facilitate deciding which hand you've got, a histogram is
computed to say how often a \epsdice[white]{1} occurs in the hand, how
often a \epsdice[white]{2} occurs and so one. A {\it Full-House} occurs
when both a $2$ and a $3$ occurs in the histogram; a {\it Four-of-a-Kind}
occurs when there is a $4$ somewhere in the histogram.

\end{exercise}
